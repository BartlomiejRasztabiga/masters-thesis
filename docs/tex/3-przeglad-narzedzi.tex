\clearpage % Rozdziały zaczynamy od nowej strony.
\section{Przegląd technologii i narzędzi}

\subsection{Modele językowe wykorzystywane w badaniu}

TODO opis modeli z ktorych korzystam (TODO a jakie to sa? sprawdzic najpierw co dziala z langchain), ich charakterystyki i architektury

podzial na open source i dostepne przez API?

napisac ze uzywam openrouter np.

roznice w sposobie korzystania z nich

parametry, fine-tuning ? i jak je bede wykorzystywac

w jaki sposob bede dokonywac inferencji (openrouter? bezposrednie api? langchain/langgraph)

w jaki sposob bede z nich korzystac - nie zwykle promptowanie, ale jako agent, opisac jak chce zeby wygladala calosc, jak dziala podejscie agentowe

generalnie moze porownac rozne metody prompotowania (zero shot, few shot, chain of thought, agent based)

ograniczenia (np ze moge tylko kilka modeli), tryb uruchamiania, licencje, kontrola danych

\subsection{Narzędzia DevOps: Docker i Kubernetes}

funkcja i zasady dzialania, krotka charakterystyka, sposob uzycia, przyklady

- Docker i jego rola (do czego konkretnie)
- Kubernetes – poziom szczegółowości tylko taki, jak potrzebny do oceny konfiguracji

na czym sie skupic? na tym co mozna popsuc przy generowaniu konfiguracji np

jakies przyklady dobrych i zlych praktyk?

\subsection{Narzędzia do oceny jakości IaC}

z jakich bede korzystac i w jaki sposob, ich funkcje i zasady dzialania

checkov, kubelinter, kube-score, kubeval, terrascan, cloudeval-yaml, iac-eval

przykladowe outputy narzedzi, ich interpretacja

\subsection{Środowisko eksperymentalne}

opis środowiska eksperymentalnego, w którym będą przeprowadzane testy

lokalnie i dlaczego

kind, docker desktop, python, langchain, openrouter itd

- Parametry techniczne (np. RAM, GPU, lokalność modeli)
- Zasady testowania przez API (np. limity, wersje)

\subsection{Technologie wspierające}

TODO a co ja tu w ogole chce opisac i po co?

python? jakies rozne biblioteki? kind? lens? 

a moze ten rozdzial zmergowac z poprzednim?


------- (dalej pisal chat)

\section{Przegląd technologii i narzędzi}

\subsection{Modele językowe wykorzystywane w badaniu}

Cel: Przedstawić modele LLM używane w eksperymencie, ich charakterystyki, sposób użycia i ograniczenia.

Proponowana zawartość:
	•	Rodzaje modeli:
	•	    Podział na modele open-source (np. Mistral, LLaMA, Nous Hermes) vs. modele komercyjne dostępne przez API (np. GPT-4, Claude).
	•	Architektury i charakterystyki:
	•	    Typowe parametry: liczba parametrów, specjalizacje, wersje, okno kontekstu (dlaczego wazne przy repozytoriach kodu)
	•	    Informacja, które modele są używane i dlaczego.
    •   Wybrane modele do badania:
    •       Krótki opis które i dlaczego z parametrami jakimiś basic
	•	Sposób użycia:
	•	    Poprzez OpenRouter, bezpośrednie API, LangChain / LangGraph.
	•	    Sposób integracji z agentami.
	•	Tryby promptowania:
	•	    Zero-shot, few-shot, chain-of-thought, agent-based.
	•	    Porównanie tych podejść i wybór odpowiednich metod.
	•	Ograniczenia i praktyczne aspekty:
	•	    Licencje, limity API, ograniczona liczba modeli.
	•	    Brak fine-tuningu – tylko inference.
	•	    Kontrola nad danymi – brak możliwości dostosowania kodu źródłowego w zamkniętych modelach.

⸻

\subsection{Narzędzia DevOps: Docker i Kubernetes}

Cel: Opisać narzędzia wykorzystywane do zarządzania środowiskiem uruchomieniowym i ich wpływ na eksperymenty.

Proponowana zawartość:
	•	Docker:
	•	    Do czego jest wykorzystywany: uruchamianie lokalnych usług, środowiska testowe, konteneryzacja narzędzi.
	•	Kubernetes:
	•	    Używany z minimalnym zakresem (np. przez Kind).
	•	    Rola w testowaniu konfiguracji i IaC.
	•	    Jak można popełnić błędy przy generowaniu YAML (niedopasowane zasoby, problemy z dostępnością, błędne konfiguracje).
	•	Dobre i złe praktyki:
	•	    Przykłady z testów (np. źle ustawione limity zasobów, niepoprawne labele).

⸻

\subsection{Narzędzia do oceny jakości IaC}

Cel: Pokazać narzędzia do analizy i walidacji konfiguracji IaC oraz ich funkcjonalności.

Proponowana zawartość:
	•	Lista narzędzi:
	•	    checkov, kube-linter, kube-score, kubeval, terrascan, cloudeval-yaml, iac-eval.
	•	Opis funkcji i zasad działania:
	•	    Statyczna analiza, reguły zgodności, walidacja schematów.
	•	Przykłady outputów:
	•	    Co zgłaszają narzędzia, jak to interpretować.
	•	    Zestawienie wyników na tych samych plikach (jeśli masz dane).

⸻

\subsection{Środowisko eksperymentalne}

Cel: Przedstawić infrastrukturę testową i środowisko, w którym uruchamiane są eksperymenty.

Proponowana zawartość:
	•	Charakterystyka środowiska:
	•	    Lokalna maszyna, Docker Desktop, Kind, Python, LangChain, OpenRouter.
	•	Parametry techniczne:
	•	    RAM, CPU, ewentualna obecność GPU, ograniczenia związane z lokalnością.
	•	Zasady testowania:
	•	    Jak wygląda testowanie przez API: limity żądań, obsługa błędów, wersje modeli.
	•	    Korzystanie z narzędzi CLI lub SDK.

⸻

\subsection{Technologie wspierające (do rozważenia jako część środowiska eksperymentalnego)}

Alternatywa 1: Zostawić jako osobny rozdział z wyraźnym celem: technologie pomocnicze ułatwiające pracę.

Alternatywa 2: Zmergować z poprzednim podrozdziałem jako „środowisko eksperymentalne i wspierające technologie”.

Zawartość:
	•	Język Python:
	•	    Używane biblioteki (np. requests, openai, langchain, pydantic, docker-py itd.).
	•	Narzędzia pomocnicze:
	•	    Kind, Lens, VSCode, Docker CLI, kubectl, LangSmith, LangFuse, Promptfoo, Kaniko.
	•	Rola:
	•	    Obsługa eksperymentów, monitorowanie, automatyzacja.

⸻
