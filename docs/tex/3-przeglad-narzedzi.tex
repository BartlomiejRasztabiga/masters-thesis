\clearpage % Rozdziały zaczynamy od nowej strony.

\section{Przegląd technologii i narzędzi}

Celem tego rozdziału jest przedstawienie najważniejszych technologii i narzędzi, stanowiących podstawę niniejszego badania. Szczególny nacisk położony zostanie na duże modele językowe, opis ich architektur oraz ich zastosowanie w generowaniu konfiguracji wdrożeniowej. Ponadto, omówione zostaną istotne narzędzia DevOps, takie jak Docker i Kubernetes, niezbędne w zarządzaniu środowiskami uruchomieniowymi, oraz narzędzia służące do walidacji i oceny jakości generowanego kodu IaC. Rozdział zawiera także opis środowiska eksperymentalnego oraz wykorzystywanych technologii wspierających automatyzację i monitorowanie eksperymentów.

\subsection{Modele językowe wykorzystywane w badaniu}

Duże modele językowe pełnią istotną rolę w prezentowanych eksperymentach, służąc jako generatory konfiguracji IaC. Ich zdolność do przetwarzania języka naturalnego oraz generowania poprawnego kodu czyni je ważnym elementem automatyzacji procesów DevOps. W tym podrozdziale przedstawiono charakterystykę wybranych modeli, opis ich architektur oraz praktyczne aspekty związane z ich wykorzystaniem.

Rynek dużych modeli językowych rozwija się dynamicznie, oferując zarówno modele komercyjne, dostępne przez API, jak i modele open-source, możliwe do uruchomienia lokalnie lub na własnej infrastrukturze. Decyzja o wyborze konkretnego rodzaju modelu często zależy od dostępności zasobów sprzętowych (zwłaszcza GPU), wymagań dotyczących licencji, kwestii związanych z prywatnością danych oraz możliwości dostosowania modeli do specyficznych zastosowań.

Modele komercyjne dostępne przez API zazwyczaj cechują się wysoką wydajnością, zaawansowanymi możliwościami oraz niezawodnością, dzięki wsparciu dużych firm technologicznych takich jak OpenAI, Anthropic czy Google. Modele te trenowane są na szerokich i zróżnicowanych zbiorach danych, co przekłada się na ich zdolność generowania wysokiej jakości tekstu oraz kodu. Ich użytkowanie wiąże się z kosztami dostępu do API oraz koniecznością przesyłania danych do zewnętrznych serwisów, co może być problematyczne w przypadku danych wrażliwych.

Wybrane do badania modele komercyjne to:

\begin{itemize}
    \item \textbf{OpenAI GPT-5} \cite{gpt5}: TODO

    \item \textbf{OpenAI GPT-5 Mini} \cite{gpt5mini}: TODO

    \item \textbf{OpenAI o3} \cite{o3}: TODO

    \item \textbf{Anthropic Claude Sonnet 4.5} \cite{claude_sonnet45}: TODO

    \item \textbf{Anthropic Claude Haiku 4.5} \cite{claude_haiku45}: TODO

    \item \textbf{Anthropic Claude Opus 4.1} \cite{claude_opus41}: TODO

    \item \textbf{Google Gemini 2.5 Pro} \cite{gemini25_pro}: TODO

    \item \textbf{Google Gemini 2.5 Flash} \cite{gemini25_flash}: TODO
\end{itemize}

Modele open-source stanowią atrakcyjną alternatywę, oferując pełną kontrolę, możliwość uruchomienia na własnej infrastrukturze oraz brak dodatkowych opłat za każde zapytanie. Ich wydajność zależy przede wszystkim od dostępności zasobów sprzętowych, a aktywny rozwój społeczności przyczynia się do ciągłej optymalizacji i pojawiania się coraz bardziej zaawansowanych wersji.

Wybrane do badania modele open-source to:

\begin{itemize}
    \item \textbf{Qwen 3} \cite{qwen3}: TODO

    \item \textbf{DeepSeek R1} \cite{deepseek_r1}: TODO

    \item \textbf{DeepSeek V3.2} \cite{deepseek_v32}: TODO

    \item \textbf{GLM 4.6} \cite{glm46}: TODO

    \item \textbf{Mistral Medium} \cite{mistral_medium}: TODO

    \item \textbf{Meta Llama 4 Maverick} \cite{llama4_maverick}: TODO

    \item \textbf{Meta Llama 4 Scout} \cite{llama4_scout}: TODO
\end{itemize}

Wszystkie wymienione modele reprezentują aktualny szczyt osiągnięć w dziedzinie dużych modeli językowych. Wybrane zostały przede wszystkim ze względu na ich potwierdzone zdolności do generowania wysokiej jakości kodu oraz wsparcie dla funkcji \textit{tool calling}. Funkcje te umożliwiają bezpośrednią integrację modeli z zewnętrznymi narzędziami lub usługami poprzez dedykowane mechanizmy API bądź platformy agentowe, takie jak LangChain \cite{langchain} czy LangGraph \cite{langgraph}. W proponowanym w tej pracy podejściu, funkcje \textit{tool calling} będą odgrywać znaczącą rolę w zarządzaniu kontekstem i procesie generowania konfiguracji.

\subsection{Architektury i charakterystyka}

Większość współczesnych dużych modeli językowych bazuje na architekturze transformera \cite{vaswani_attention_2023}, wykorzystując mechanizm uwagi (attention mechanism) do efektywnego przetwarzania sekwencji danych. Architektury te, pomimo wspólnych założeń, różnią się między sobą liczbą parametrów, rozmiarem okna kontekstowego oraz specyficznymi optymalizacjami, które wpływają na ich zdolność do generowania kodu i rozumienia złożonych instrukcji. Coraz większe znaczenie zyskują również alternatywne podejścia architektoniczne, takie jak Mixture-of-Experts (MoE), w których przy każdym zapytaniu aktywowane są tylko wybrane części modelu, co pozwala na zwiększenie efektywności obliczeniowej bez pogarszania jakości generowanych wyników.

Poniżej przedstawiono najważniejsze cechy różnicujące współczesne modele językowe, mające istotny wpływ na ich zachowanie i przydatność w zadaniach związanych z generowaniem kodu:

\begin{itemize}
    \item \textbf{Liczba parametrów:} Liczba parametrów (często liczona w miliardach) jest wskaźnikiem skali modelu i jego zdolności do uczenia się złożonych wzorców z danych. Modele z większą liczbą parametrów zazwyczaj wykazują lepszą wydajność w szerokim zakresie zadań, jednak ich uruchomienie i obsługa wymagają znacznie większych zasobów obliczeniowych. Należy jednak zaznaczyć, że sama liczba parametrów nie jest jedynym wyznacznikiem jakości; znaczenie ma również jakość danych treningowych, architektura (np. Mixture-of-Experts) oraz proces dostrajania modelu.
    \item \textbf{Okno kontekstowe:} Okno kontekstowe odnosi się do maksymalnej długości sekwencji tokenów (słów, znaków, fragmentów kodu), którą model może przetworzyć i wykorzystać do wygenerowania odpowiedzi. Jest to szczególnie istotne w kontekście generowania konfiguracji IaC z repozytoriów kodu. Duże okno kontekstowe pozwala na dostarczenie modelowi całych plików źródłowych aplikacji, fragmentów dokumentacji, wielu powiązanych ze sobą promptów, a nawet wyników walidacji z zewnętrznych narzędzi. To umożliwia agentowi LLM holistyczne zrozumienie projektu i kontekstu, co przekłada się na wyższą jakość i trafność generowanych konfiguracji. Modele takie jak Google Gemini 2.5 Pro wyróżniają się wyjątkowo dużym oknem kontekstowym (do miliona tokenów), co jest znaczącą przewagą w zadaniach wymagających głębokiej analizy kodu i dokumentacji.
    \item \textbf{Specjalizacje i optymalizacje:} Niektóre modele, takie jak Mistral Codestral, są specjalizowane lub fine-tuninguowane w kierunku generowania kodu. Oznacza to, że są one trenowane na dużych zbiorach danych zawierających kod programistyczny, co poprawia ich zdolność do generowania syntaktycznie poprawnego i semantycznie trafnego kodu. Architektury takie jak Mixture-of-Experts (MoE) stosowane np. w DeepSeek V3 pozwalają na efektywne skalowanie modeli, aktywując tylko część ekspertów dla danego zapytania, co optymalizuje zużycie zasobów przy zachowaniu wysokiej jakości odpowiedzi.
\end{itemize}

Poniższa tabela \ref{tab:llm-characteristic} przedstawia ogólne charakterystyki wybranych modeli LLM, które będą wykorzystywane w badaniu. Okno kontekstu podano w liczbie tokenów, a liczba parametrów — w miliardach, jeśli została opublikowana.

\begin{table}[!h] \centering
\caption{Charakterystyka wybranych modeli LLM}
\label{tab:llm-characteristic}
\begin{tabular}{| c | c | c | c |} \hline
\textbf{Dostawca} & \textbf{Model} & \textbf{Parametry (mld)} & \textbf{Okno kontekstu (tokeny)} \\ \hline\hline
OpenAI & GPT-5 & - & TODO \\ \hline
OpenAI & GPT-5 Mini & - & TODO \\ \hline
OpenAI & o3 & - & TODO \\ \hline
Anthropic & Claude Sonnet 4.5 & - & TODO \\ \hline
Anthropic & Claude Haiku 4.5 & - & TODO \\ \hline
Anthropic & Claude Opus 4.1 & - & TODO \\ \hline
Google & Gemini 2.5 Pro & - & TODO \\ \hline
Google & Gemini 2.5 Flash & - & TODO \\ \hline
Qwen & Qwen 3 & TODO & TODO \\ \hline
DeepSeek & DeepSeek R1 & TODO & TODO \\ \hline
DeepSeek & DeepSeek V3.2 & TODO & TODO \\ \hline
Zhipu AI & GLM 4.6 & TODO & TODO \\ \hline
Mistral AI & Mistral Medium & TODO & TODO \\ \hline
Meta & Llama 4 Maverick & TODO & TODO \\ \hline
Meta & Llama 4 Scout & TODO & TODO \\ \hline
\end{tabular}
\end{table}

\subsection{Sposoby wykorzystania modeli}

W ramach niniejszego badania modele językowe są wykorzystywane w trybie inferencji (wnioskowania), bez dodatkowego etapu ich dostrajania (fine-tuningu) na specyficznym zbiorze danych kodu infrastruktury. Oznacza to, że modele wykorzystują swoją ogólną wiedzę, nabytą podczas szerokiego treningu, do generowania konfiguracji na podstawie dostarczonego kontekstu i promptów. Taki sposób podejścia jest zgodny z koncepcją "Don't Train, Just Prompt" \cite{kratzke_dont_2024}, co pozwala na szybką iterację i wykorzystanie najnowszych, potężnych modeli bez konieczności kosztownego i czasochłonnego procesu ponownego trenowania.

Integracja z wybranymi modelami LLM odbywa się głównie za pośrednictwem ich bezpośrednich interfejsów programistycznych (API). Takie podejście gwarantuje dostęp do najnowszych wersji modeli, pełną kontrolę nad parametrami zapytania oraz optymalne czasy odpowiedzi. Należy jednak pamiętać o potencjalnych ograniczeniach wynikających z limitów API oraz braku możliwości bezpośredniej modyfikacji wewnętrznej architektury lub kodu źródłowego w przypadku zamkniętych modeli komercyjnych.

Warto zaznaczyć, że chociaż inferencję dla wielu modeli open-source (takich jak te z serii LLaMA czy Mistral) można przeprowadzać lokalnie, wymaga to znacznych zasobów sprzętowych (zwłaszcza mocnego procesora graficznego - GPU). Ze względu na ograniczenia sprzętowe środowiska eksperymentalnego, w niniejszej pracy nie zastosowano lokalnej inferencji. Zamiast tego, dla modeli komercyjnych korzystano z ich dedykowanych API, natomiast dla modeli open-source, które nie oferują własnego API, wykorzystana została platforma OpenRouter \cite{openrouter}, która agreguje dostęp do wielu modeli od różnych dostawców, oferując ujednolicony interfejs.

\subsection{Tryby promptowania}

Efektywność wykorzystania dużych modeli językowych w dużej mierze zależy od sposobu formułowania zapytań (promptów) \cite{kratzke_dont_2024}. Istnieje kilka podstawowych strategii promptowania, które mogą być stosowane w zależności od złożoności zadania i dostępności przykładów:

\begin{itemize}
	\item \textbf{Zero-shot prompting:} Jest to najprostsza forma promptowania, gdzie model otrzymuje zadanie lub pytanie bez żadnych wcześniejszych przykładów. Oczekuje się, że model, bazując na swojej wiedzy ogólnej, wygeneruje odpowiedź. W kontekście generowania IaC, model otrzymałby instrukcję typu "Wygeneruj Dockerfile dla aplikacji Node.js", bez dodatkowych danych kontekstowych poza samym opisem zadania. Jest to podejście szybkie, ale jego skuteczność jest często ograniczona, zwłaszcza w przypadku złożonych lub bardzo specyficznych wymagań.
	\item \textbf{Few-shot prompting:} Ta strategia polega na dostarczeniu modelowi kilku przykładów par (zapytanie, oczekiwana odpowiedź) przed właściwym zadaniem \cite{brown_language_2020} Przykłady te służą jako demonstracja pożądanego formatu i stylu odpowiedzi, pomagając modelowi zrozumieć intencję użytkownika i dostosować swoje generacje. W przypadku IaC, można by podać kilka par "opis aplikacji \textrightarrow{} przykładowy Dockerfile/manifest Kubernetes", aby model nauczył się wzorca i preferowanych praktyk. Zastosowanie tej metody często zwiększa jakość generacji w porównaniu do zero-shot.
	\item \textbf{Chain-of-Thought (CoT) prompting:} Jest to technika, w której prompt instruuje model, aby przed podaniem ostatecznej odpowiedzi, wygenerował sekwencję pośrednich kroków rozumowania. Model "myśli na głos", co często prowadzi do bardziej precyzyjnych i trafnych wyników, szczególnie w zadaniach wymagających złożonych rozumowań lub wieloetapowych rozwiązań. W początkowej fazie rozwoju LLM technika CoT często wymagała dedykowanych instrukcji w prompcie, aby model wygenerował te kroki. Obecnie wiele zaawansowanych modeli językowych, dzięki swojej architekturze i procesom treningowym, potrafi wykazywać tego typu "rozumowanie" w sposób bardziej intuicyjny, często nawet bez wyraźnego polecenia, jeśli kontekst zadania na to wskazuje. W kontekście IaC, model mógłby najpierw "zastanowić się" nad zależnościami aplikacji, potem nad wymaganiami środowiskowymi i optymalizacjami, a dopiero potem wygenerować kod konfiguracji. CoT poprawia transparentność i debugowalność procesu generowania, a także może zwiększyć poprawność odpowiedzi.
	\item \textbf{Agent-based prompting:} Ten tryb wykracza poza jednorazowe generowanie odpowiedzi. Model, działając jako "agent", otrzymuje zadanie i może iteracyjnie wykonywać szereg akcji: planować, używać zewnętrznych narzędzi (tzw. "tool calling"), analizować ich wyniki i modyfikować swój plan działania w celu osiągnięcia celu. Podejście agentowe można postrzegać jako ewolucję i rozszerzenie idei Chain-of-Thought, gdzie "myślenie" modelu jest przekształcane w sekwencję interakcji ze światem zewnętrznym i refleksji. Agent nie tylko generuje wewnętrzne kroki rozumowania, ale także aktywnie wykorzystuje narzędzia do zbierania danych, weryfikacji hipotez i korygowania swojego działania na podstawie rzeczywistych wyników. Jest to podejście szczególnie efektywne w przypadku złożonych problemów wymagających interakcji ze środowiskiem zewnętrznym oraz adaptacji.
\end{itemize}

W kontekście generowania konfiguracji wdrożeniowej, zastosowanie podejścia agentowego jest uzasadnione ze względu na możliwość kompleksowego zarządzania kontekstem i procesem generacji. Generowanie IaC często wymaga dostępu do wielu źródeł informacji i precyzyjnego umieszczenia danych w konfiguracji. W przeciwieństwie do jednorazowych generacji (jak w zero-shot czy few-shot), agent może, na podstawie dostarczonego mu opisu docelowej infrastruktury, wygenerować wstępną wersję konfiguracji (np. Dockerfile). Następnie, istotnym elementem jest możliwość wykorzystania zewnętrznych narzędzi do interakcji z kontekstem. Agent może na przykład czytać wybrane pliki z repozytorium kodu, aby lepiej zrozumieć strukturę projektu i zależności. Może również sterować swoim wewnętrznym przepływem rozumowania i zakresem czytanego kontekstu.

W niniejszej pracy magisterskiej zastosowana zostanie właśnie metoda agentowa. Podejście agentowe może być realizowane na różne sposoby, zarówno poprzez bezpośrednie wykorzystanie dedykowanych interfejsów API udostępnianych przez niektóre modele (np. poprzez \textit{tool calling} czy \textit{function calling}), jak i poprzez zastosowanie specjalizowanych frameworków orkiestracyjnych, takich jak \textbf{AutoGen} \cite{autogen} czy \textbf{LangChain/LangGraph}. W tym badaniu, do implementacji agentów i zarządzania złożonymi przepływami pracy z wykorzystaniem funkcji \textit{tool calling} wybrana została biblioteka LangGraph w języku Python, uzupełniona o funkcjonalności LangChain. Rola agenta koncentruje się na efektywnym dostarczaniu kontekstu i zarządzaniu procesem generacji konfiguracji IaC.

\subsection{Narzędzia DevOps: Docker i Kubernetes}

Zarządzanie środowiskami uruchomieniowymi i wdrażanie aplikacji w chmurze opiera się współcześnie na narzędziach takich jak Docker i Kubernetes. Niniejszy podrozdział opisuje te technologie, ich rolę w cyklu DevOps oraz potencjalne pułapki konfiguracyjne, które mogą być generowane przez duże modele językowe. Zrozumienie ich funkcjonalności i typowych błędów jest istotne dla oceny jakości generowanych przez LLM konfiguracji IaC.

\textbf{Docker} to platforma do tworzenia, wdrażania i uruchamiania aplikacji w kontenerach. Konteneryzacja pozwala na pakowanie aplikacji wraz z jej zależnościami i bibliotekami w izolowane środowiska, co zapewnia spójność działania niezależnie od środowiska docelowego. Docker jest wykorzystywany w wielu obszarach: od uruchamiania lokalnych usług deweloperskich i testowych, przez konteneryzację narzędzi, aż po konteneryzację produkcyjnych aplikacji w złożonych architekturach mikroserwisowych.

Tworzenie plików Dockerfile – deklaratywnych instrukcji do budowania obrazów Docker – wymaga precyzji i znajomości najlepszych praktyk. Modele LLM, generując Dockerfile, mogą popełnić szereg błędów, które wpływają na bezpieczeństwo, wydajność lub poprawne działanie kontenera. Przykłady takich błędów to:
\begin{itemize}
\item \textbf{Użycie ADD zamiast COPY:} ADD automatycznie rozpakowuje archiwa i może pobierać pliki z URL, co jest mniej transparentne i może wprowadzać nieoczekiwane zależności lub luki bezpieczeństwa. Preferowane jest jawne COPY dla lokalnych plików.
\item \textbf{Brak WORKDIR lub jego nieoptymalne użycie:} Nieokreślenie katalogu roboczego lub jego wielokrotne zmienianie może prowadzić do nieczytelnych i nieefektywnych Dockerfile.
\item \textbf{Użycie tagu latest dla obrazów bazowych:} Obrazy z tagiem latest są zmienne, co prowadzi do braku determinizmu w budowaniu aplikacji. Powoduje to, że ta sama aplikacja może nie budować się w przyszłości w ten sam sposób lub z tym samym obrazem bazowym. Zalecane jest używanie konkretnych wersji obrazów (np. node:20.10-slim).
\item \textbf{Brak usuwania zbędnych zależności i plików po instalacji:} Pozostawianie pakietów użytych tylko do budowy lub tymczasowych plików zwiększa rozmiar obrazu i powierzchnię ataku (ang. attack surface).
\item \textbf{Uruchamianie aplikacji jako root:} Domyślne uruchamianie procesów w kontenerze z uprawnieniami roota jest poważną luką bezpieczeństwa.
\item \textbf{Niewystarczające buforowanie warstw:} Nieoptymalna kolejność instrukcji w Dockerfile może prowadzić do częstego przebudowywania warstw i wydłużenia czasu budowania obrazów.
\end{itemize}

\textbf{Kubernetes (K8s)} jest systemem do automatyzacji wdrażania, skalowania i zarządzania aplikacjami kontenerowymi. Dzięki deklaratywnemu podejściu, użytkownicy definiują pożądany stan systemu (poprzez manifesty YAML), a Kubernetes dąży do jego utrzymania. W niniejszym badaniu Kubernetes służy przede wszystkim jako środowisko do \textbf{walidacji poprzez uruchomienie} wygenerowanych konfiguracji IaC. Pozwala to na weryfikację, czy manifesty Kubernetes, wygenerowane przez LLM, faktycznie prowadzą do uruchomienia działających aplikacji i czy zachowują się zgodnie z oczekiwaniami, a nie tylko są syntaktycznie poprawne.

Podobnie jak w przypadku Dockerfile, generowanie manifestów Kubernetes przez LLM może prowadzić do szeregu błędów i nieoptymalnych konfiguracji, które mogą wpłynąć na dostępność, bezpieczeństwo lub wydajność aplikacji:
\begin{itemize}
\item \textbf{Brak liveness i readiness probes:} Te sondy są kluczowe dla monitorowania stanu kontenerów i zapewnienia wysokiej dostępności. Ich brak może prowadzić do wdrażania niedziałających aplikacji lub błędnego zarządzania cyklem życia poda.
\item \textbf{Niedopasowane lub błędne zasoby (CPU/Memory limits and requests):} Niewłaściwe ustawienie limitów lub żądań zasobów może prowadzić do problemów z wydajnością aplikacji (throttling), niestabilności klastra (OOMKilled) lub nieefektywnego wykorzystania infrastruktury.
\item \textbf{Błędne konfiguracje serwisów:} Niepoprawne typy serwisów (np. użycie ClusterIP zamiast NodePort lub LoadBalancer gdy wymagany jest dostęp zewnętrzny) lub brak poprawnej konfiguracji portów może uniemożliwić dostęp do aplikacji.
\item \textbf{Brak PersistentVolumeClaim (PVC) dla aplikacji stanowych:} Aplikacje wymagające trwałego przechowywania danych (np. bazy danych) bez zdefiniowanego PVC utracą dane po restarcie poda.
\item \textbf{Niepoprawne użycie ConfigMap i Secret:} Twarde zakodowanie wrażliwych danych bezpośrednio w manifeście zamiast użycia Secret lub nieprawidłowe zamontowanie ConfigMap/Secret.
\item \textbf{Brak kontekstu bezpieczeństwa (Security Context):} Brak definicji securityContext dla poda lub kontenera, co może prowadzić do uruchamiania aplikacji z niepotrzebnie wysokimi uprawnieniami (np. jako root, z otwartymi portami lub zbyt szerokim dostępem do systemu plików).
\item \textbf{Niepoprawne etykiety (labels) i selektory (selectors):} Błędy w etykietach mogą uniemożliwić serwisom odnalezienie odpowiednich podów, co prowadzi do niedostępności aplikacji.
\end{itemize}

Analiza zdolności LLM do unikania wymienionych błędów oraz do generowania efektywnych, bezpiecznych i zoptymalizowanych konfiguracji Dockerfile i manifestów Kubernetes stanowi jeden z głównych obszarów oceny w niniejszej pracy.



\subsection{Narzędzia do oceny jakości IaC}

W kontekście automatycznego generowania konfiguracji Infrastructure as Code (IaC) przez modele językowe, ważne jest zapewnienie, że wygenerowane konfiguracje są nie tylko poprawne składniowo, ale również bezpieczne, zgodne z najlepszymi praktykami i efektywne. Do tego celu służą wyspecjalizowane narzędzia do analizy i walidacji kodu IaC. Wyniki uzyskane z tych narzędzi będą stanowić podstawę do obiektywnej oceny jakości konfiguracji generowanych przez LLM w dalszej części niniejszej pracy. W niniejszym podrozdziale opisane zostaną dwa wybrane narzędzia, które zostaną użyte w badaniu: Hadolint \cite{hadolint} (dla Dockerfile) i Kube-linter \cite{kubelinter} (dla manifestów Kubernetes).

Oba narzędzia działają na zasadzie statycznej analizy kodu, co oznacza, że analizują pliki konfiguracyjne bez konieczności ich uruchamiania lub wdrażania.

Hadolint to narzędzie do lintingu i statycznej analizy plików Dockerfile. Sprawdza zgodność z najlepszymi praktykami, wykrywa potencjalne błędy i luki bezpieczeństwa. Analizuje pliki Dockerfile, stosując zestaw wbudowanych reguł (bazujących na najlepszych praktykach Docker) oraz konfigurowalnych reguł. Wykrywa takie problemy jak błędy składniowe w Dockerfile, użycie przestarzałych lub niebezpiecznych instrukcji, brak optymalizacji warstw, czy potencjalne luki bezpieczeństwa (np. uruchamianie procesów jako root).

Kube-linter to narzędzie do analizy manifestów Kubernetes, wykrywające błędy konfiguracyjne, problemy z bezpieczeństwem i niezgodności z najlepszymi praktykami. Analizuje manifesty Kubernetes (pliki YAML), sprawdzając je pod kątem zgodności z najlepszymi praktykami Kubernetes oraz potencjalnych błędów konfiguracyjnych. Wykrywa takie problemy jak brak liveness i readiness probes, nieprawidłowe limity zasobów, problemy z konfiguracją serwisów, czy potencjalne luki bezpieczeństwa (np. brak securityContext).

Poniżej przedstawiono przykładowe fragmenty wyjścia z Hadolint i Kube-linter, ilustrujące typowe problemy wykrywane przez te narzędzia:

\begin{lstlisting}[caption={Przykład wyników Hadolint},label={lst:example-hadolint},captionpos=b,columns=fullflexible, breaklines=true]
Dockerfile:3 DL3003 Use WORKDIR to set the working directory for subsequent instructions.
Dockerfile:5 DL3018 Pin versions in apk add. Instead of `apk add <package>` use
`apk add <package>=<version>`.
Dockerfile:12 DL3005 Do not use apt-get upgrade
\end{lstlisting}

W powyższym przykładzie, przedstawionym w wycinku \ref{lst:example-hadolint}, Hadolint zgłasza brak instrukcji WORKDIR, brak przypiętych wersji pakietów w apk add oraz użycie apt-get upgrade.

\begin{lstlisting}[caption={Przykład wyników Kube-linter},label={lst:example-kubelinter},captionpos=b,columns=fullflexible, breaklines=true]
[ERROR] Pod: hello-pod, object name hello-pod: container "hello" has no livenessProbe.
[ERROR] Pod: hello-pod, object name hello-pod: container "hello" has no readinessProbe.
[WARNING] Service: hello-service, object name hello-service: service "hello-service" should
set spec.type to "ClusterIP"
\end{lstlisting}

W tym przykładzie, zaprezentowanym w wycinku \ref{lst:example-kubelinter}, Kube-linter zgłasza brak livenessProbe i readinessProbe w definicji Poda oraz sugeruje zmianę typu Serwisu na ClusterIP.

Interpretacja wyników z obu narzędzi wymaga zrozumienia specyfiki Dockerfile i manifestów Kubernetes. Ważne jest, aby skupić się na błędach o wysokim poziomie ważności (np. problemy z bezpieczeństwem, brak sond) oraz na ostrzeżeniach, które mogą prowadzić do problemów z wydajnością lub stabilnością aplikacji. Wyjścia z tych narzędzi zostaną wykorzystane w dalszej części pracy do obiektywnej oceny jakości generowanych konfiguracji IaC.

\subsection{Środowisko eksperymentalne i technologie wspierające}

Eksperymenty zostały przeprowadzone na lokalnej stacji roboczej, którą stanowi komputer MacBook Air z procesorem Apple M4 i 16 GB pamięci RAM. Ważne jest, że w środowisku tym nie wykorzystywano lokalnie żadnych kart graficznych (GPU) do inferencji modeli językowych. Wszystkie interakcje z modelami LLM odbywały się wyłącznie za pośrednictwem zewnętrznych interfejsów API.

Do zarządzania kontenerami i środowiskami uruchomieniowymi wykorzystano następujące narzędzia:
\begin{itemize}
	\item \textbf{Docker Desktop:} \cite{docker_desktop} Używany do uruchamiania kontenerów oraz do budowania obrazów Docker na podstawie generowanych plików Dockerfile. Docker Desktop służył jako podstawowe środowisko konteneryzacji.
	\item \textbf{Kind (Kubernetes in Docker):} \cite{kind} Lekki klaster Kubernetes uruchamiany w kontenerach Docker. Kind posłużył jako środowisko do walidacji wygenerowanych manifestów Kubernetes w fazie eksperymentalnej. Pozwoliło to na weryfikację, czy aplikacje faktycznie uruchamiają się i działają poprawnie w środowisku Kubernetes.
	\item \textbf{MicroK8s:} \cite{microk8s} Lekka dystrybucja Kubernetes przeznaczona do uruchamiania w środowiskach produkcyjnych oraz testowych. MicroK8s został wykorzystany jako bardziej zbliżony do środowiska produkcyjnego klaster Kubernetes, umożliwiający walidację wygenerowanych konfiguracji w warunkach bardziej odpowiadających rzeczywistym wdrożeniom.
\end{itemize}

Cała logika sterowania eksperymentami, w tym implementacja agentów, wywoływanie narzędzi, walidacja i zbieranie wyników, została zrealizowana przy użyciu języka \textbf{Python w wersji 3.13}. Wykorzystane biblioteki Python obejmowały:
\begin{itemize}
	\item \textbf{LangChain i LangGraph:} \cite{langchain} \cite{langgraph} Te frameworki stanowiły podstawę do budowy architektury agentowej, zarządzania przepływami konwersacji z LLM oraz integrowania wywołań narzędzi.
	\item \textbf{Docker SDK for Python (docker):} \cite{docker_sdk_python} Umożliwiała programistyczną interakcję z Docker Daemon, w tym budowanie obrazów i zarządzanie kontenerami.
	\item \textbf{Kubernetes Python Client (kubernetes):} \cite{kubernetes_sdk_python} Służyła do programistycznej interakcji z API Kubernetes, umożliwiając wdrażanie manifestów i sprawdzanie stanu zasobów.
	\item \textbf{Pydantic:} \cite{pydantic} Wykorzystywana do walidacji danych i definiowania schematów wejścia/wyjścia dla funkcji i narzędzi używanych przez agentów, co zwiększyło niezawodność i bezpieczeństwo interakcji.
	\item \textbf{Streamlit:} \cite{streamlit} Framework do szybkiego tworzenia aplikacji webowych w języku Python, wykorzystany do implementacji panelu sterowania eksperymentami. Umożliwił stworzenie interaktywnego interfejsu do zarządzania eksperymentami, monitorowania ich postępu oraz przeglądania wyników.
	\item Inne standardowe biblioteki Python do obsługi danych, operacji na plikach oraz generowania raportów.
\end{itemize}

Interakcje z modelami LLM odbywały się za pośrednictwem dwóch głównych ścieżek:
\begin{itemize}
	\item \textbf{Bezpośrednie API:} W przypadku modeli komercyjnych, które oferują własne dedykowane API (np. OpenAI GPT-4, Google Gemini Pro, Anthropic Claude), korzystano z ich natywnych interfejsów programistycznych.
	\item \textbf{OpenRouter:} Dla modeli open-source (np. Mistral, Meta Llama, DeepSeek), które często nie posiadają własnych komercyjnych API lub ich infrastruktura do hostingu jest zbyt kosztowna dla pojedynczego badacza, wykorzystano platformę \textbf{OpenRouter}. OpenRouter działa jako brama API (API Gateway), agregująca dostęp do wielu modeli od różnych dostawców za pośrednictwem ujednoliconego interfejsu API, co znacząco ułatwiało testowanie różnorodnych modeli bez konieczności integracji z wieloma odrębnymi API.
\end{itemize}

W trakcie prowadzenia badań i eksperymentów wykorzystano również szereg narzędzi wspomagających, które były cenne dla efektywnego zarządzania procesem badawczym i analizą wyników:
\begin{itemize}
	\item \textbf{LangSmith:} \cite{langsmith} Nrzędzie do monitorowania, debugowania i ewaluacji łańcuchów agentów zbudowanych w LangChain/LangGraph. LangSmith pozwalał na wizualizację poszczególnych kroków rozumowania agenta, jego interakcji z narzędziami oraz wywołań API do LLM. Było to kluczowe do analizy, dlaczego agent podjął taką, a nie inną decyzję, i gdzie popełnił błąd, co znacząco przyspieszyło proces optymalizacji promptów i logiki agenta.
	\item \textbf{Lens IDE:} \cite{lens_ide} Graficzne narzędzie (IDE) do zarządzania klastrami Kubernetes, ułatwiające monitorowanie stanu podów, serwisów i innych zasobów w trakcie walidacji.
\end{itemize}
Wspomniane narzędzia pomocnicze były integralną częścią środowiska badawczego, umożliwiając nie tylko uruchamianie eksperymentów, ale przede wszystkim ich efektywne monitorowanie, analizę i optymalizację.
