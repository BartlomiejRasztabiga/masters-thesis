\clearpage % Rozdziały zaczynamy od nowej strony.
\section{Analiza porównawcza modeli}

W niniejszym rozdziale przedstawiono wyniki przeprowadzonych eksperymentów, w których testowano zdolność wybranych modeli językowych do generowania konfiguracji Docker i Kubernetes na podstawie zawartości repozytoriów aplikacji. Analiza obejmuje zarówno aspekty ilościowe (metryki wydajności, poprawności i bezpieczeństwa), jak i jakościowe (zachowanie modeli, typy popełnianych błędów, deterministyczność wyników).

\subsection{Przegląd przeprowadzonych eksperymentów}

[TODO: Opis konfiguracji eksperymentów, liczba przebiegów, kombinacje modeli i repozytoriów]

Eksperymenty przeprowadzono zgodnie z metodologią opisaną w rozdziale 4, wykorzystując system orkiestracji eksperymentów umożliwiający automatyczne wykonanie serii testów dla różnych kombinacji modeli, repozytoriów i wariantów promptów. Każda konfiguracja została przetestowana wielokrotnie w celu oceny powtarzalności wyników.

Testowane repozytoria:
\begin{itemize}
    \item \textbf{poc1-fastapi} — aplikacja bezstanowa bez zależności,
    \item \textbf{poc2-fastapi} — aplikacja ze stanową bazą danych PostgreSQL,
    \item \textbf{poc3-fastapi} — architektura wielowarstwowa (frontend + backend + baza danych).
\end{itemize}

Testowane modele obejmowały zarówno komercyjne rozwiązania (OpenAI: GPT-5, GPT-5 Mini, o3; Anthropic: Claude Sonnet 4.5, Claude Haiku 4.5, Claude Opus 4.1; Google: Gemini 2.5 Pro, Gemini 2.5 Flash), jak i modele open-source (DeepSeek V3.2, GLM 4.6, Mistral Medium, Meta Llama 4 Maverick, Meta Llama 4 Scout).

\subsection{Metodyka analizy}

[TODO: Opis metryk i metod analizy wyników]

Dla każdego przebiegu eksperymentu zebrano następujące kategorie metryk:
\begin{itemize}
    \item \textbf{Metryki wydajnościowe} — liczba tokenów wejściowych i wyjściowych, czas generacji, liczba wywołań narzędzi,
    \item \textbf{Metryki poprawności} — poprawność składniowa plików Dockerfile i manifestów Kubernetes, powodzenie budowy obrazów Docker, powodzenie wdrożenia w klastrze Kubernetes,
    \item \textbf{Metryki bezpieczeństwa i jakości} — liczba i poziomy ważności błędów wykrytych przez Hadolint i Kube-linter,
    \item \textbf{Metryki walidacji uruchomieniowej} — dostępność aplikacji po wdrożeniu, poprawność działania punktów końcowych API.
\end{itemize}

Wyniki zostały zagregowane i przeanalizowane z wykorzystaniem statystyk opisowych oraz porównań między modelami dla poszczególnych typów repozytoriów.

\subsection{Wyniki testów poprawności składniowej}

[TODO: Tabela z wynikami poprawności składniowej dla każdego modelu i repozytorium]

\subsection{Wyniki testów budowy i wdrożenia}

[TODO: Analiza powodzenia budowy obrazów Docker i wdrożenia w Kubernetes]

\subsection{Analiza bezpieczeństwa i zgodności z dobrymi praktykami}

[TODO: Zestawienie liczby błędów wykrytych przez Hadolint i Kube-linter per model]

\subsection{Analiza wydajności generacji}

[TODO: Porównanie czasów generacji, zużycia tokenów i liczby wywołań narzędzi]

\subsection{Deterministyczność wyników}

[TODO: Analiza różnic między wielokrotnymi przebiegami dla tych samych konfiguracji]

\subsection{Przykładowe obserwacje i napotkane problemy}

[TODO: Opis specyficznych zachowań modeli podczas eksperymentów]

W trakcie przeprowadzania eksperymentów zaobserwowano szereg charakterystycznych zachowań oraz typowych problemów napotkanych przez modele LLM podczas generacji konfiguracji:

\begin{itemize}
    \item \textbf{Ograniczenia długości kontekstu} — niektóre modele o mniejszych oknach kontekstowych miały trudności z przetworzeniem dużych repozytoriów zawierających wiele plików,
    \item \textbf{Błędy w rozumieniu struktury repozytorium} — modele czasami niepoprawnie interpretowały zależności między komponentami lub pomijały kluczowe pliki konfiguracyjne,
    \item \textbf{Wpływ długości i precyzji prompta na jakość generacji} — bardziej szczegółowe instrukcje systemowe prowadziły do lepszej zgodności z dobrymi praktykami, jednak zbyt długie prompty mogły powodować "rozmycie uwagi" modelu,
    \item \textbf{Deterministyczność generowanych plików} — mimo ustawienia parametru \texttt{temperature = 0}, niektóre modele wykazywały niewielkie różnice w kolejnych przebiegach, szczególnie w przypadkach wymagających długich sekwencji interakcji z narzędziami,
    \item \textbf{Niekonsekwencje w nazwach plików i usług} — modele czasami stosowały różne konwencje nazewnictwa w obrębie tej samej konfiguracji, co mogło prowadzić do problemów z integracją komponentów.
\end{itemize}

\subsection{Analiza porównawcza modeli}

[TODO: Szczegółowe porównanie modeli komercyjnych vs open-source, modeli ogólnych vs wyspecjalizowanych w kodzie]

\subsection{Wnioski z analizy}

[TODO: Podsumowanie najważniejszych odkryć, identyfikacja najlepiej i najgorzej radzących sobie modeli, rekomendacje dotyczące wyboru modelu w zależności od przypadku użycia]