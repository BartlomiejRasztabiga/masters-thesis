\clearpage % Rozdziały zaczynamy od nowej strony.
\section{Projekt eksperymentów}

TODO

% kryteria oceny tutaj czy w nastepnym?

% tu chyba napisac w jaki sposob to odpalam, ze agent, ze langsmith itd



% TODO to tylko fragment o closed feedback loopie z static analyzerami:

% 4.X. Podejście agentowe z pętlą sprzężenia zwrotnego (Feedback Loop)

% W celu maksymalizacji jakości i bezpieczeństwa generowanych konfiguracji, w badaniu zostanie zaimplementowane podejście agentowe wykorzystujące mechanizm zamkniętej pętli sprzężenia zwrotnego. Koncepcja ta zakłada iteracyjne udoskonalanie wygenerowanego kodu na podstawie automatycznie pozyskiwanego feedbacku z narzędzi do statycznej analizy i walidacji IaC.

% Proces będzie przebiegał następująco:

% Generacja początkowa: Wybrany model LLM (wykorzystując techniki promptowania takie jak zero-shot lub few-shot) wygeneruje wstępną konfigurację Dockerfile lub manifestu Kubernetes na podstawie zadanego promptu.
% Walidacja i analiza: Wygenerowany kod zostanie automatycznie przekazany do zestawu narzędzi walidacyjnych i analitycznych, takich jak Checkov, Kube-score i Kube-linter. Narzędzia te ocenią kod pod kątem poprawności składniowej, zgodności z najlepszymi praktykami oraz potencjalnych luk bezpieczeństwa i niezoptymalizowanych ustawień.
% Pozyskiwanie feedbacku: Wyniki analizy (np. lista wykrytych "violations" lub zaleceń) zostaną przetworzone i sformatowane w sposób zrozumiały dla LLM.
% Iteracyjna korekta: Przetworzony feedback zostanie przekazany z powrotem do LLM jako rozbudowany prompt, instruujący model do dokonania korekt w początkowej konfiguracji. Proces ten będzie powtarzany w celu zmniejszenia liczby wykrytych problemów lub osiągnięcia zdefiniowanego progu jakości.
% Kryteria zatrzymania: Pętla sprzężenia zwrotnego zostanie przerwana po osiągnięciu maksymalnej liczby iteracji (np. 3-5 iteracji) lub gdy liczba wykrytych naruszeń spadnie poniżej ustalonego poziomu (np. brak krytycznych błędów bezpieczeństwa).
% W ramach eksperymentów, kluczowe będzie zmierzenie początkowej liczby naruszeń (violations) generowanych przez LLM bez feedbacku oraz dynamiki ich redukcji w kolejnych iteracjach z wykorzystaniem pętli sprzężenia zwrotnego. Pozwoli to na ocenę efektywności podejścia agentowego i zdolności LLM do samodzielnej korekty kodu na podstawie zewnętrznych walidatorów. Badanie to pozwoli również zidentyfikować typy błędów, które są najłatwiejsze i najtrudniejsze do skorygowania automatycznie.

% Uzasadnienie umieszczenia tu:

% Projektowanie: Opisujesz, jak zaprojektujesz swoje eksperymenty, w tym innowacyjną metodę z feedback loop.
% Klarowność: Oddzielenie tego w osobnym podrozdziale w "Projekcie eksperymentów" podkreśla jego znaczenie i pozwala na szczegółowe omówienie.
% Wprowadzenie metryk: Już na etapie projektu eksperymentów określasz, co będziesz mierzyć (początkowe violations, redukcja, liczba iteracji), co jest spójne z metodologią naukową.