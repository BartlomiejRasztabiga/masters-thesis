\clearpage % Rozdziały zaczynamy od nowej strony.
\section{Wyniki eksperymentów}

W niniejszym rozdziale przedstawiono wyniki przeprowadzonych eksperymentów badających możliwości autonomicznego generowania konfiguracji Infrastructure as Code przez agenty oparte na dużych modelach językowych. Analiza wyników zorganizowana jest zgodnie ze strukturą hipotez badawczych przedstawionych w rozdziale 4 — dla każdej z pięciu hipotez (H1–H5) zaprezentowano wyniki eksperymentalne, analizę danych oraz wnioski dotyczące weryfikacji hipotezy. Rozdział kończy synteza wyników podkreślająca kluczowe obserwacje dotyczące zdolności LLM do generowania IaC.

\subsection{Przegląd przeprowadzonych eksperymentów}

Eksperymenty przeprowadzono zgodnie z metodologią opisaną w rozdziale 4, wykorzystując system orkiestracji umożliwiający automatyczne wykonanie serii testów dla różnych kombinacji modeli, repozytoriów i wariantów konfiguracji. Szczegóły zbiorów danych, środowiska wykonawczego oraz metryk znajdują się w rozdziale 4.

\subsection{Wyniki H1: Autonomiczna generacja funkcjonalnych konfiguracji}

\subsubsection{Wyniki ilościowe}

\textbf{Współczynnik sukcesu ogólny:}

Łącznie 94/150 przebiegów zakończyło się sukcesem end-to-end, co daje 62{,}7\% (95\% CI: 54{,}7\%--70{,}0\%). Oznacza to, że w ponad 60\% przypadków agent był w stanie wygenerować konfigurację IaC umożliwiającą poprawne zbudowanie, wdrożenie i uruchomienie aplikacji testowej.

\textbf{Analiza etapowa i warunkowa:}

% TODO: Dodać wykres kaskady etapów.

\begin{table}[h]
    \centering
    \begin{tabular}{lccc}
        \textbf{Etap} & \textbf{Sukcesy} & \textbf{Skuteczność} & \textbf{95\% CI} \\
        Build & 118/150 & 78{,}7\% & 71{,}4\%--84{,}5\% \\
        K8s apply & 115/148 & 77{,}7\% & 70{,}3\%--83{,}7\% \\
        apply $|$ build & 115/118 & 97{,}5\% & 92{,}8\%--99{,}1\% \\
        Runtime & 94/150 & 62{,}7\% & 54{,}7\%--70{,}0\% \\
        runtime $|$ apply & 94/115 & 81{,}7\% & 73{,}7\%--87{,}7\% \\
    \end{tabular}
    \caption{Skuteczność etapów w H1}
    \label{tab:h1-stages}
\end{table}

\textbf{Rozbicie per model:}

\begin{table}[h]
    \centering
    \begin{tabular}{lccc}
        \textbf{Model} & \textbf{Sukcesy} & \textbf{Skuteczność} & \textbf{95\% CI} \\
        gemini-2.5-flash & 31/50 & 62{,}0\% & 48{,}2\%--74{,}1\% \\
        gpt-5-mini & 34/50 & 68{,}0\% & 54{,}2\%--79{,}2\% \\
        deepseek-chat & 29/50 & 58{,}0\% & 44{,}2\%--70{,}6\% \\
    \end{tabular}
    \caption{Skuteczność per model w H1}
    \label{tab:h1-models}
\end{table}

\textbf{Rozbicie per repozytorium:}

\begin{table}[h]
    \centering
    \begin{tabular}{lccc}
        \textbf{Repozytorium} & \textbf{Sukcesy} & \textbf{Skuteczność} & \textbf{95\% CI} \\
        math-pdf-generator-web & 6/6 & 100{,}0\% & 61{,}0\%--100{,}0\% \\
        simple-webapp & 0/6 & 0{,}0\% & 0{,}0\%--39{,}0\% \\
        banner-designer-webapp & 5/6 & 83{,}3\% & 43{,}6\%--97{,}0\% \\
        lsfont & 5/6 & 83{,}3\% & 43{,}6\%--97{,}0\% \\
        airwatch & 0/6 & 0{,}0\% & 0{,}0\%--39{,}0\% \\
        diceopt\_kcd2 & 6/6 & 100{,}0\% & 61{,}0\%--100{,}0\% \\
        parklandschooltimer & 6/6 & 100{,}0\% & 61{,}0\%--100{,}0\% \\
        realtime-chatting-webapp & 1/6 & 16{,}7\% & 3{,}0\%--56{,}4\% \\
        csv-hero & 2/6 & 33{,}3\% & 9{,}7\%--70{,}0\% \\
        metathief & 3/6 & 50{,}0\% & 18{,}8\%--81{,}2\% \\
        deeplx-app & 6/6 & 100{,}0\% & 61{,}0\%--100{,}0\% \\
        mtg-scryfall-randomizer & 6/6 & 100{,}0\% & 61{,}0\%--100{,}0\% \\
        vesen & 3/6 & 50{,}0\% & 18{,}8\%--81{,}2\% \\
        webclient & 2/6 & 33{,}3\% & 9{,}7\%--70{,}0\% \\
        chessable & 4/6 & 66{,}7\% & 30{,}0\%--90{,}3\% \\
        class-order-checker & 6/6 & 100{,}0\% & 61{,}0\%--100{,}0\% \\
        adsensedetective & 1/6 & 16{,}7\% & 3{,}0\%--56{,}4\% \\
        mario-kart-tournament & 1/6 & 16{,}7\% & 3{,}0\%--56{,}4\% \\
        tempo-sync & 5/6 & 83{,}3\% & 43{,}6\%--97{,}0\% \\
        attendence-calculator & 6/6 & 100{,}0\% & 61{,}0\%--100{,}0\% \\
        scheme-seva & 2/6 & 33{,}3\% & 9{,}7\%--70{,}0\% \\
        hello-world-war & 4/6 & 66{,}7\% & 30{,}0\%--90{,}3\% \\
        openai-hello-world & 2/6 & 33{,}3\% & 9{,}7\%--70{,}0\% \\
        todo-list & 6/6 & 100{,}0\% & 61{,}0\%--100{,}0\% \\
        simple-webapp-flask & 6/6 & 100{,}0\% & 61{,}0\%--100{,}0\% \\
    \end{tabular}
    \caption{Skuteczność per repozytorium w H1}
    \label{tab:h1-repos}
\end{table}

\subsubsection{Typowe problemy i tryby awarii}

% TODO: Lista najczęstszych przyczyn niepowodzeń + przykłady.

% \begin{itemize}
%     \item \textbf{Błędna identyfikacja punktu wejścia aplikacji} — [X]\% przypadków,
%     \item \textbf{Pominięcie zależności systemowych} — [Y]\% przypadków,
%     \item \textbf{Niepoprawna konfiguracja zmiennych środowiskowych} — [Z]\% przypadków,
%     \item \textbf{Błędy w definiowaniu portów i protokołów sieciowych} — [W]\% przypadków,
%     \item \textbf{Brak obsługi inicjalizacji bazy danych} — [V]\% przypadków.
% \end{itemize}

\subsubsection{Werdykt dla H1}

Na podstawie kryteriów z rozdziału 4 hipoteza H1 została potwierdzona. Spełnienie warunków przedstawia się następująco:
\begin{itemize}
    \item end-to-end success rate: 62{,}7\% (próg 60\%),
    \item skuteczność etapów: build 78{,}7\% (próg 75\%), apply 77{,}7\% (próg 75\%), runtime 62{,}7\% (próg 60\%),
    \item metryki warunkowe: apply $|$ build 97{,}5\% (próg 95\%), runtime $|$ apply 81{,}7\% (próg 80\%),
    \item repozytoria z sukcesem $\geq$ 50\%: 16 z 25 (wymagane $\geq$ 50\% zbioru).
\end{itemize}
W konsekwencji można uznać, że agenty LLM są w stanie autonomicznie generować funkcjonalne konfiguracje IaC w badanym zakresie.

\subsection{Wyniki H2: Ograniczenia złożonościowe}

\subsubsection{Wyniki ilościowe}

% TODO: Wykres success rate vs. poziom złożoności (repozytoria kontrolowane).

\subsubsection{Trend złożoności}

% TODO: Opis trendu spadku skuteczności i punktu załamania.

\subsubsection{Werdykt dla H2}

% TODO: Wniosek z odniesieniem do kryteriów z rozdziału 4.

\subsection{Wyniki H3: Jakość i zgodność z dobrymi praktykami}

\subsubsection{Wyniki ilościowe}

% TODO: Liczba ostrzeżeń Hadolint/Kube-linter (bazowy prompt vs. prompt best practices).

\subsubsection{Wpływ prompt engineeringu}

% TODO: Porównanie jakości konfiguracji przed/po wzmocnieniu prompta.

\subsubsection{Werdykt dla H3}

% TODO: Wniosek z odniesieniem do progów z rozdziału 4.

\subsection{Wyniki H4: Niezawodność i powtarzalność procesu agentowego}

\subsubsection{Wyniki ilościowe}

% TODO: run-to-run diff ratio + zmienność liczby kroków narzędziowych.

\subsubsection{Analiza powtarzalności}

% TODO: Wpływ deterministycznych parametrów (temperature=0, seed).

\subsubsection{Werdykt dla H4}

% TODO: Wniosek z odniesieniem do progów z rozdziału 4.

\subsection{Wyniki H5: Podatność na manipulację kontekstem}

\subsubsection{Wyniki ilościowe}

% TODO: Odsetek przebiegów z odchyleniami od oczekiwań.

\subsubsection{Przykłady manipulacji}

% TODO: Krótkie studia przypadków (prompt injection, myląca dokumentacja).

\subsubsection{Werdykt dla H5}

% TODO: Wniosek z odniesieniem do progów z rozdziału 4.

\subsection{Porównanie modeli}

% TODO: Analiza przekrojowa modeli (skuteczność, jakość, powtarzalność).

\subsection{Synteza wyników}

\subsubsection{Podsumowanie weryfikacji hipotez}

% TODO: Tabela zbiorcza z werdyktem dla H1–H5 i krótkim uzasadnieniem.

\subsubsection{Kluczowe wnioski}

% TODO: 3–6 punktów najważniejszych obserwacji.

\subsubsection{Ograniczenia badania}

% TODO: Wypunktuj ograniczenia wyników (zbieżne z threats to validity).

\subsubsection{Implikacje praktyczne}

% TODO: Zalecenia dla praktyków i twórców narzędzi DevOps.
