%-----------------------------------------------
%  Engineer's & Master's Thesis Template
%  Copyleft by Artur M. Brodzki & Piotr Woźniak
%  Warsaw University of Technology, 2019-2022
%-----------------------------------------------

\documentclass[
    bindingoffset=5mm,  % Binding offset
    footnoteindent=3mm, % Footnote indent
    hyphenation=true    % Hyphenation turn on/off
]{src/wut-thesis}

\graphicspath{{tex/img/}} % Katalog z obrazkami.
\addbibresource{bibliografia.bib} % Plik .bib z bibliografią

\usepackage{minted}

%-------------------------------------------------------------
% Wybór wydziału:
%  \facultyeiti: Wydział Elektroniki i Technik Informacyjnych
%  \facultymeil: Wydział Mechaniczny Energetyki i Lotnictwa
% --
% Rodzaj pracy: \EngineerThesis, \MasterThesis, \PPMGR
% --
% Wybór języka: \langpol, \langeng
%-------------------------------------------------------------
\facultyeiti    % Wydział Elektroniki i Technik Informacyjnych
\MasterThesis % Praca inżynierska
\langpol % Praca w języku polskim

\begin{document}

\counterwithin{lstlisting}{section}

%------------------
% Strona tytułowa
%------------------
\instytut{Informatyki}
\kierunek{Informatyka}
\specjalnosc{Inteligentne Systemy}
\title{
    Zastosowanie dużych modeli językowych (LLM) \\ 
    do generowania konfiguracji Docker i Kubernetes
}
% Title in English for English theses
% In English theses, you may remove this command
\engtitle{
    Application of large language models (LLMs) \\
    for generating Docker and Kubernetes configurations
}
% Title in Polish for English theses
% Use it only in English theses
\poltitle{
    Zastosowanie dużych modeli językowych (LLM) \\ 
    do generowania konfiguracji Docker i Kubernetes
}
\author{Bartłomiej Rasztabiga}
\album{304117}
\promotor{dr inż. Mateusz Modrzejewski}
\date{\the\year}
\maketitle

%-------------------------------------
% Streszczenie po polsku dla \langpol
% English abstract if \langeng is set
%-------------------------------------
\cleardoublepage % Zaczynamy od nieparzystej strony
\abstract
Celem niniejszej pracy magisterskiej jest zbadanie możliwości wykorzystania dużych modeli językowych (LLM) do automatycznego generowania konfiguracji infrastruktury jako kodu (IaC), ze szczególnym uwzględnieniem plików Dockerfile oraz manifestów Kubernetes. Przeprowadzono przegląd literatury opisującej aktualne badania w tym obszarze, wskazując zarówno potencjał automatyzacji, jak i liczne wyzwania związane z poprawnością, bezpieczeństwem i niezawodnością wygenerowanego kodu.

W części badawczej pracy przeprowadzono analizę możliwości wybranych modeli językowych, zaprojektowano zestaw eksperymentów porównawczych oraz opracowano własne kryteria oceny poprawności i bezpieczeństwa generowanych manifestów Docker i Kubernetes. Następnie, na podstawie wyników eksperymentów, zaprojektowano i zaimplementowano prototyp systemu typu Platform as a Service (PaaS), który automatycznie generuje konfiguracje na podstawie repozytorium kodu, buduje obrazy Docker oraz wdraża aplikacje w klastrze Kubernetes.

Przeprowadzono serię testów oceniających skuteczność modeli pod względem poprawności składniowej, zgodności z wymaganiami oraz odporności na błędy i podatności bezpieczeństwa. Dodatkowo w systemie zaimplementowano komponenty wspierające walidację i automatyczne wykrywanie potencjalnie niebezpiecznych konfiguracji.

\keywords Infrastructure as Code, LLM, Kubernetes, Docker, automatyzacja, bezpieczeństwo, DevOps, PaaS

%----------------------------------------
% Streszczenie po angielsku dla \langpol
% Polish abstract if \langeng is set
%----------------------------------------
\clearpage
\secondabstract
The goal of this master’s thesis is to investigate the potential of large language models (LLMs) for automatically generating Infrastructure as Code (IaC), with a particular focus on Dockerfiles and Kubernetes manifests. The study includes a literature review of recent research in the field, identifying both the automation potential and key challenges related to the correctness, security, and reliability of LLM-generated infrastructure code.

In the research phase, a comparative analysis of selected language models was conducted, along with the design of experimental test cases and the development of custom evaluation criteria for the correctness and security of generated Docker and Kubernetes manifests. Based on the experimental results, a prototype Platform as a Service (PaaS) system was designed and implemented, capable of automatically generating configurations from code repositories, building Docker images, and deploying applications to a Kubernetes cluster.

A series of tests was conducted to evaluate the models’ accuracy, compliance with system requirements, and resistance to misconfigurations and security vulnerabilities. In addition, components supporting validation and automatic detection of unsafe configurations were implemented.

\secondkeywords Infrastructure as Code, Large Language Models, Kubernetes, Docker, Automation, Security, DevOps, PaaS

\pagestyle{plain}

%--------------
% Spis treści
%--------------
\cleardoublepage % Zaczynamy od nieparzystej strony
\tableofcontents

%------------
% Rozdziały
%------------
\cleardoublepage % Zaczynamy od nieparzystej strony
\pagestyle{headings}

% TODO rozdziały magisterki
% 
% TODO

\clearpage % Rozdziały zaczynamy od nowej strony.
\section{Wprowadzenie}

W ciągu ostatnich lat obserwujemy dynamiczny rozwój dużych modeli językowych (LLM) \cite{zhao_survey_2025} oraz równoległy wzrost znaczenia technologii konteneryzacji i orkiestracji. Modele takie jak GPT \cite{brown_language_2020}, LLaMA \cite{touvron_llama_2023}, Falcon \cite{almazrouei_falcon_2023} czy Claude \cite{anthropic_claude} wykazują zdolność generowania złożonego kodu, w tym konfiguracji infrastruktury \cite{srivatsa_survey_2024}, podczas gdy Docker \cite{merkel_docker_nodate} i Kubernetes \cite{burns_borg_2016} stały się standardami wdrażania aplikacji w środowiskach produkcyjnych \cite{kratzke_dont_2024}. Niniejsza praca bada potencjał wykorzystania dużych modeli językowych do pełnej automatyzacji procesu generowania i wdrażania konfiguracji Docker oraz Kubernetes, ze szczególnym uwzględnieniem aspektów poprawności, bezpieczeństwa oraz odporności na ataki manipulacyjne.

\subsection{Cel i zakres pracy}

Celem niniejszej pracy magisterskiej jest analiza możliwości zastosowania dużych modeli językowych (LLM) do automatycznego generowania konfiguracji typu Infrastructure as Code (IaC, infrastruktura jako kod) \cite{pahl_infrastructure_nodate}, ze szczególnym uwzględnieniem plików Dockerfile oraz manifestów Kubernetes.
W ramach pracy zostanie przeprowadzona ocena jakości, poprawności i bezpieczeństwa generowanych konfiguracji, ze szczególnym uwzględnieniem zagrożeń wynikających z automatyzacji, takich jak podatności na ataki typu "prompt injection" czy manipulację zawartością repozytoriów. Finalnym celem jest zaprojektowanie oraz implementacja prototypowego systemu typu Platform as a Service (PaaS), który na podstawie kodu źródłowego repozytorium automatycznie generuje, buduje i wdraża aplikacje kontenerowe bez ingerencji człowieka.

Praca ma charakter badawczo-prototypowy i obejmuje następujące zagadnienia:
\begin{itemize}
\item przegląd aktualnego stanu wiedzy na temat wykorzystania LLM w generowaniu kodu i konfiguracji infrastruktury,
\item identyfikację braków i zagrożeń związanych z automatyzacją generacji IaC, w tym problemów bezpieczeństwa i odporności na ataki (np. "prompt injection", manipulacja repozytorium),
\item analizę i porównanie skuteczności różnych modeli językowych (m.in. GPT, LLaMA, Falcon, Claude) w zadaniach generacji plików Dockerfile i manifestów Kubernetes,
\item przeprowadzenie eksperymentów obejmujących testy poprawności, jakości, bezpieczeństwa i deterministyczności generowanych konfiguracji,
\item zaprojektowanie architektury i implementację prototypowego systemu typu PaaS, który na podstawie repozytorium kodu automatycznie generuje, buduje i wdraża aplikacje kontenerowe bez ingerencji człowieka.
\end{itemize}

\subsection{Motywacja}

Motywacja pracy wynika z kilku kluczowych czynników. Po pierwsze, rośnie znaczenie metodyk DevOps oraz automatyzacji zarządzania infrastrukturą. IaC umożliwia spójną, powtarzalną konfigurację środowisk i redukcję błędów ludzkich \cite{low_repairing_2024}. Jednak ręczne tworzenie skryptów IaC dla złożonych środowisk chmurowych jest czasochłonne i wymaga specjalistycznej wiedzy. Rozwój dużych modeli językowych (LLM) stwarza możliwość automatycznego generowania kodu konfiguracyjnego na podstawie opisów w języku naturalnym, co może obniżyć barierę wejścia dla mniej doświadczonych programistów i przyspieszyć procesy wdrożeniowe \cite{hu_llm-based_2025}.

Jednocześnie automatyzacja generowania IaC rodzi pytania o poprawność i bezpieczeństwo tworzonych konfiguracji. Modele językowe mogą popełniać błędy lub „halucynować”, generując nieistniejące lub niezalecane elementy konfiguracji \cite{malul_genkubesec_2024}. Istotne jest więc zbadanie wiarygodności LLM w kontekście infrastruktury krytycznej oraz zgodności generowanych plików z najlepszymi praktykami bezpieczeństwa. Automatyzacja tego procesu ma również praktyczne znaczenie w platformach typu PaaS, gdzie może znacząco skrócić czas wdrażania aplikacji oraz poprawić niezawodność i bezpieczeństwo świadczonych usług.

\subsection{Struktura pracy}

Praca składa się z dziewięciu rozdziałów, odzwierciedlających pełny cykl badawczy:
\begin{itemize}
    \item \textbf{Rozdział 1 – Wprowadzenie}: przedstawienie tematu, celów, motywacji oraz układu pracy,
    \item \textbf{Rozdział 2 – Przegląd literatury}: analiza aktualnych badań nad wykorzystaniem LLM w generowaniu konfiguracji IaC (Docker, Kubernetes) oraz identyfikacja luk w istniejącej wiedzy,
    \item \textbf{Rozdział 3 – Przegląd technologii i narzędzi}: opis wybranych modeli językowych, technologii infrastrukturalnych oraz porównanie podejść API vs open-source,
    \item \textbf{Rozdział 4 – Projekt eksperymentów}: opis przypadków testowych, strategii promptowania oraz procesu generowania i testowania konfiguracji; przedstawienie kryteriów oceny,
    \item \textbf{Rozdział 5 – Analiza porównawcza modeli LLM}: przedstawienie wyników testów oraz wniosków dotyczących jakości działania modeli,
    \item \textbf{Rozdział 6 – Bezpieczeństwo konfiguracji}: analiza zagrożeń związanych z automatyczną generacją IaC, technik ataku na LLM oraz metod oceny bezpieczeństwa wygenerowanych plików,
    \item \textbf{Rozdział 7 – Projekt systemu PaaS}: opis architektury systemu oraz procesu automatycznego wdrażania aplikacji na podstawie repozytorium kodu,
    \item \textbf{Rozdział 8 – Implementacja i wdrożenie prototypu}: opis realizacji systemu PaaS, wykorzystanych technologii oraz przykładowych wdrożeń,
    \item \textbf{Rozdział 9 – Wnioski i dalsze kierunki rozwoju}: podsumowanie wyników badań, ocena przydatności LLM w kontekście DevOps oraz wskazanie możliwych obszarów dalszych badań.
\end{itemize}

\clearpage % Rozdziały zaczynamy od nowej strony.
\section{Przegląd literatury}

\subsection{Zastosowanie LLM w generowaniu kodu infrastruktury}

Podejście Infrastructure as Code polega na zarządzaniu infrastrukturą IT poprzez definiowanie jej konfiguracji w postaci czytelnego dla maszyny kodu. Zapewnia to automatyzację, spójność środowisk, kontrolę wersji oraz skalowalność zarządzania infrastrukturą. Tworzenie i utrzymanie takiej infrastruktury bywa jednak wymagające, stąd coraz większe zainteresowanie automatyzacją tych zadań z użyciem AI. Duże modele językowe, trenowane na ogromnych zbiorach danych tekstowych i kodu, wykazały zdolność do generowania i rozumienia kodu, co czyni je obiecującym narzędziem do automatycznego tworzenia konfiguracji IaC. W ostatnich latach pojawiło się wiele prac badających zastosowanie LLM do różnych platform i języków IaC. Srivatsa i in. (2024) \cite{srivatsa_survey_2024} przedstawili przekrojowe spojrzenie na ten temat w swojej pracy przeglądowej, podkreślając, że automatyzacja IaC za pomocą LLM może zredukować nakład pracy ludzkiej i błędy, jeśli uda się odpowiednio ukierunkować model. W pracy tej omówiono również wyzwania stojące przed takimi rozwiązaniami oraz perspektywy dalszych badań.

Aktualne badania integrujące LLM z narzędziami chmurowymi pokazują obiecujące rezultaty, choć są to dopiero początki. Lanciano i in. (2023) \cite{lanciano_analyzing_2023} zaproponowali wyspecjalizowany model analizujący pliki wdrożeniowe Kubernetes, aby wspomóc mniej doświadczonych użytkowników w projektowaniu konfiguracji i zapewnieniu ich jakości. Z kolei Xu i in. (2023) \cite{ueno_migrating_2024} wprowadzili benchmark CloudEval-YAML do oceny zdolności LLM w generowaniu konfiguracji aplikacji cloud-native (w formacie YAML), udostępniając zestaw testów jednostkowych do weryfikacji poprawności wygenerowanych manifestów. Inni badacze sugerują zastosowanie LLM również do automatycznego monitorowania i utrzymania systemów – np. Malul i in. (2024) \cite{malul_genkubesec_2024} zaproponowali pipeline wykorzystujący LLM do detekcji anomalii i autoremediacji w środowisku Kubernetes, w celu zwiększenia stabilności systemu. Podejścia te często bazują na trenowaniu dedykowanych modeli językowych dopasowanych do domeny konfiguracji (np. z użyciem specjalistycznych korpusów danych). Alternatywą dla kosztownego trenowania od podstaw jest wykorzystanie istniejących, ogólnych modeli (GPT-3.5, GPT-4, Llama2 itp.) i dostrajanie \cite{liu_pre-train_2023} ich za pomocą odpowiednich promptów \cite{kratzke_dont_2024}. Niniejsza praca wpisuje się w ten nurt – badano, jak standardowe modele językowe mogą być wykorzystane (bez dodatkowego trenowania) do generowania plików Dockerfile i manifestów Kubernetes zgodnych z wymaganiami DevOps, poprzez odpowiednie techniki inżynierii promptów.

Ważnym kierunkiem rozwoju są również benchmarki oceny jakości konfiguracji IaC generowanych przez modele językowe. Kon i in. zaprezentowali benchmark IaC-Eval, który służy do systematycznej oceny jakości konfiguracji dla usług chmurowych AWS \cite{kon_iac-eval_nodate}. Narzędzie to umożliwia bardziej obiektywną ocenę możliwości modeli językowych w kontekście generowania poprawnych, zgodnych z intencją użytkownika konfiguracji.

Pujar i in. (2023) \cite{pujar_invited_2023} zaproponowali natomiast wykorzystanie LLM do generowania konfiguracji IT w języku Ansible YAML, opracowując narzędzie Ansible Wisdom. Ich praca pokazuje, że techniki prompt engineering można z powodzeniem zastosować również do innych języków deklaratywnych, nie tylko Dockerfile czy manifestów Kubernetes.

Reasumując, liczne prace badawcze wykazały potencjał LLM w automatyzacji tworzenia konfiguracji IaC – zarówno poprzez integrację z narzędziami chmurowymi, jak i dostosowanie technik promptowania do specyfiki deklaratywnych języków konfiguracyjnych. Jednocześnie zauważono szereg wyzwań, takich jak błędne generacje (halucynacje), aspekty bezpieczeństwa czy zgodność z zamierzeniami użytkownika \cite{malul_genkubesec_2024, kratzke_dont_2024}.

\subsection{Generowanie konfiguracji Docker przy użyciu LLM}

Docker stał się standardem w dziedzinie konteneryzacji aplikacji, umożliwiając tworzenie przenośnych, izolowanych środowisk wykonawczych. Jednak tworzenie efektywnych i bezpiecznych plików Dockerfile wymaga specjalistycznej wiedzy i doświadczenia.

Blog Docker \cite{docker_genai} przedstawia innowacyjne podejście do generowania plików Dockerfile przy użyciu GenAI. Autorzy demonstrują, jak wyposażenie asystenta AI w narzędzia do analizy projektu pozwala na generowanie bardziej adekwatnych plików Dockerfile. Zamiast polegać wyłącznie na ogólnej wiedzy modelu, asystent może bezpośrednio analizować kod źródłowy projektu, określać jego typ (np. projekt NPM) i na tej podstawie generować odpowiedni Dockerfile. Co więcej, autorzy podkreślają znaczenie inkorporowania najlepszych praktyk do procesu generowania, co pozwala uniknąć typowych problemów, takich jak używanie przestarzałych obrazów bazowych czy pomijanie mechanizmów buforowania.

\begin{lstlisting}[caption={Dockerfile wygenerowany przez LLM},label={lst:example-dockerfile},captionpos=b,numbers=left]
# Stage 1 - Downloading dependencies
FROM node:22-slim AS deps
WORKDIR /usr/src/app
COPY package*.json ./
RUN --mount=type=cache,target=/root/.npm npm ci --omit=dev

# Stage 2 - Building application
FROM deps AS build
RUN --mount=type=cache,target=/root/.npm npm ci && npm build

# Stage 3 - Using a recommended base image from Scout
FROM node:22-slim
WORKDIR /usr/src/app
COPY --from=deps /usr/src/app/node_modules ./node_modules
COPY --from=build /usr/src/app/dist ./dist
CMD [ "npm", "start" ]    
\end{lstlisting}

Powyższy listing \ref{lst:example-dockerfile} wygenerowanego przez LLM pliku Dockerfile ilustruje, jak model może zastosować najlepsze praktyki, takie jak budowa wieloetapowa, używanie rekomendowanych obrazów bazowych oraz mechanizmów cache.

Jednym z pionierskich rozwiązań praktycznych integrujących LLM z DevOps jest projekt Repo2Run – agent oparty na LLM, zaprojektowany do automatycznego konfigurowania środowisk i generowania plików Dockerfile dla dowolnych repozytoriów Python. Narzędzie to zostało zaprezentowane przez Hu i in. (2025) \cite{hu_llm-based_2025} i stanowi próbę pełnej automatyzacji procesu tworzenia kontenera na podstawie kodu aplikacji. Problem, jaki adresuje Repo2Run, to złożoność ręcznego przygotowania Dockerfile dla istniejącego projektu – np. ustalenie zależności, właściwej bazy obrazu, komend instalacji pakietów – a także ryzyko błędów podczas konfiguracji środowiska (np. niepowodzenie jednej komendy może zostawić środowisko w nieokreślonym stanie). Repo2Run rozwiązuje to poprzez dwupoziomową architekturę środowiska uruchomieniowego. Model LLM działa wewnątrz odizolowanego kontenera (środowisko wewnętrzne) i wykonuje kolejne polecenia konfiguracyjne, podczas gdy otaczające go środowisko zewnętrzne monitoruje przebieg i w razie potrzeby modyfikuje kontekst (np. zmienia bazowy obraz Docker). Wprowadzono mechanizm rollback, który przy każdym błędzie cofa stan kontenera do poprzedniego stabilnego punktu, zapobiegając tzw. „zanieczyszczeniu” środowiska nieudanymi komendami. Równolegle działa komponent generujący Dockerfile – na podstawie zestawu wykonanych z sukcesem poleceń buduje on finalny plik Dockerfile, pomijając te instrukcje, które powodowały błędy, aby wynikowy obraz dawał się zbudować bez błędów. Według autorów, Repo2Run jest pierwszym podejściem wykorzystującym agenta LLM do automatycznego wygenerowania Dockerfile z istniejącego repozytorium kodu w pełni samodzielnie. Skuteczność rozwiązania oceniono na zbiorze 420 popularnych projektów Pythona z testami jednostkowymi – w 86\% przypadków Repo2Run zdołał poprawnie skonfigurować środowisko i wygenerować działający Dockerfile dla danego projektu. Tak wysoki odsetek pokazuje potencjał LLM w automatyzacji tworzenia kontenerów, choć warto zauważyć, że około 14\% przypadków nadal wymaga interwencji (co może wynikać z nietypowych zależności lub ograniczeń modelu w rozumieniu specyfiki danego projektu).

\subsection{Zastosowanie LLM w generowaniu i zarządzaniu konfiguracjami Kubernetes}

Kubernetes, jako standardowa platforma do orkiestracji kontenerów w środowiskach chmurowych, oferuje potężne możliwości w zakresie zarządzania aplikacjami kontenerowymi, ale jednocześnie charakteryzuje się wysokim poziomem złożoności. Tworzenie poprawnych manifestów Kubernetes – złożonych plików YAML opisujących obiekty takie jak wdrożenia, usługi czy konfiguracje – stanowi wyzwanie nawet dla doświadczonych specjalistów DevOps. Ze względu na deklaratywny charakter manifestów i ich szczegółowe wymagania, każdy błąd lub pominięcie może prowadzić do nieprawidłowego działania aplikacji. Poprawne przygotowanie konfiguracji wymaga dobrej znajomości API Kubernetes oraz najlepszych praktyk projektowych.

Kratzke i Drews (2024) \cite{kratzke_dont_2024} zaproponowali podejście “Don’t Train, Just Prompt”, w którym wykorzystali istniejące modele językowe (m.in. GPT-3.5, GPT-4 oraz otwarte modele Llama2, Mistral) do generowania manifestów Kubernetes na podstawie opisów tekstowych, bez trenowania modeli od zera. Zastosowali różne techniki inżynierii promptów, w tym tryb zero-shot (model generuje manifest tylko na podstawie pojedynczej instrukcji), few-shot (model otrzymuje kilka przykładów poprawnych manifestów jako kontekst) oraz prompt-chaining (łańcuch promptów, gdzie wynik pierwszego zapytania jest iteracyjnie korygowany w kolejnym).

Wyniki ich badań są zachęcające: nawet bez dostrajania, duże modele były w stanie wygenerować poprawne manifesty spełniające podane wymagania konfiguracyjne, choć czasem wymagały drobnych poprawek przez człowieka. W szczególności modele GPT-4 oraz GPT-3.5 okazały się na tyle potężne, że umożliwiły w pewnych przypadkach w pełni automatyczne wdrożenie aplikacji na Kubernetes – od specyfikacji słownej do działającego klastru – bez konieczności ręcznej ingerencji DevOps. Co ciekawe, autorzy zauważyli, że nie zawsze „większy znaczy lepszy” – mniejsze modele (jak np. Llama2 13B) przy odpowiednim przygotowaniu promptu potrafiły dorównać, a nawet przewyższyć dokładnością większe modele w generowaniu niektórych fragmentów manifestu. Kwestionuje to powszechne założenie, że tylko największe modele są użyteczne, wskazując jednocześnie na dużą rolę właściwej konfiguracji zapytania (promptu) w osiągnięciu dobrych wyników.

W podsumowaniu pracy autorzy podkreślają, że opracowanie skutecznych promptów jest kluczowe dla uzyskania wysokiej jakości konfiguracji z modelu językowego oraz że podejście to stanowi obiecujący kierunek rozwoju automatyzacji DevOps. Sugestie te potwierdzają, iż z odpowiednimi technikami LLM może znacząco ułatwić tworzenie manifestów, czyniąc zarządzanie Kubernetes bardziej intuicyjnym i mniej obarczonym ryzykiem błędu ludzkiego.

W praktyce migracji aplikacji do chmury istotny jest również scenariusz konwersji istniejących konfiguracji kontenerów do formatu Kubernetes. Ueno i Uchiumi (2024) \cite{ueno_migrating_2024} zaproponowali wykorzystanie LLM do automatycznej migracji z Docker Compose do Kubernetes. Docker Compose jest prostszym narzędziem do definiowania wielokontenerowych aplikacji, często używanym przez programistów lokalnie, podczas gdy Kubernetes wymaga bardziej szczegółowych manifestów do wdrożenia w środowisku produkcyjnym. Różnica poziomu abstrakcji między Compose a Kubernetes sprawia, że ręczna migracja bywa czasochłonna i podatna na pomyłki.

Autorzy opracowali benchmark pozwalający ocenić, jak dobrze model językowy (np. ChatGPT) radzi sobie z tłumaczeniem pliku docker-compose.yml na odpowiadające mu manifesty Kubernetes. Ocena obejmowała trzy kryteria: (1) czytelność i zrozumiałość wygenerowanego kodu dla programisty (np. czy zachowano przejrzystą strukturę i komentarze), (2) zgodność ze specyfikacją wejściową (czy manifest Kubernetes rzeczywiście odpowiada usługom/zależnościom z oryginalnego Compose) oraz (3) spójność i kompletność wygenerowanej konfiguracji.

Wyniki eksperymentów pokazały, że LLM potrafią zasadniczo poprawnie konwertować typowe pliki Compose – w wielu przypadkach dodając nawet brakujące informacje, które nie były jawnie podane, a są wymagane przez Kubernetes (np. tworząc domyślne definicje dla elementów, które Compose traktuje w uproszczony sposób). Innymi słowy, model uzupełniał „luki” w specyfikacji, dostarczając działający manifest Kubernetes nawet z niepełnych danych wejściowych.

Zauważono jednak dwa znaczące ograniczenia. Po pierwsze, brak komentarzy – wygenerowane manifesty zwykle nie zawierały żadnych adnotacji ani objaśnień, przez co traciły na czytelności z punktu widzenia dewelopera. Ręcznie pisane konfiguracje często zawierają komentarze wyjaśniające nietypowe ustawienia; model językowy pomijał tę warstwę, skupiając się wyłącznie na kodzie. Po drugie, przy nietypowych lub niejasnych specyfikacjach wejściowych jakość wyników znacząco spadała. Jeśli intencje użytkownika nie były jednoznaczne lub plik Compose zawierał niekonwencjonalne konstrukcje, LLM miewał trudności z poprawnym wygenerowaniem odpowiedników w Kubernetes.

Mimo tych wad, praca Ueno i Uchiumi potwierdza, że LLM mogą wspomóc proces migracji do chmury, automatyzując dużą część pracy konfiguracyjnej. Wskazuje jednocześnie na potrzebę dalszych usprawnień – np. integracji mechanizmów dodających objaśnienia do kodu, czy lepszego radzenia sobie z niejednoznacznymi przypadkami poprzez dopytywanie użytkownika lub bardziej zaawansowane prompty.

\begin{lstlisting}[caption={Manifest Kubernetes wygenerowany przez LLM},label={lst:example-k8s},captionpos=b,numbers=left]
apiVersion: apps/v1
kind: Deployment
metadata:
  name: example-app
spec:
  replicas: 3
  selector:
    matchLabels:
      app: example-app
  template:
    metadata:
      labels:
        app: example-app
    spec:
      containers:
      - name: example-app
        image: example-image:latest
        ports:
        - containerPort: 8080
        resources:
          limits:
            cpu: "500m"
            memory: "512Mi"
          requests:
            cpu: "200m"
            memory: "256Mi"
\end{lstlisting}

Powyższy listing \ref{lst:example-k8s} przedstawia prosty manifest Kubernetes wygenerowany przez LLM, który definiuje wdrożenie aplikacji z trzema replikami, ograniczeniami zasobów i ekspozycją portu.

Po omówieniu możliwości generowania kodu, warto przyjrzeć się kwestiom jego jakości i dalszego utrzymania.

\subsection{Wykrywanie i naprawa błędnych konfiguracji przy użyciu LLM}

Oprócz generowania nowych konfiguracji, istotnym polem badań jest wykorzystanie LLM do wykrywania i naprawy błędów w istniejących skryptach IaC. Błędne lub nieoptymalne konfiguracje mogą prowadzić do poważnych problemów – od awarii usług, po luki bezpieczeństwa narażające system na ataki. Tradycyjne narzędzia (skanery statyczne, lintery) potrafią wykryć pewne klasy błędów konfiguracyjnych, lecz ich naprawa zwykle spoczywa na inżynierach. Zainteresowanie budzi więc pytanie, czy model językowy mógłby automatycznie poprawiać wykryte nieprawidłowości IaC.

Low i in. (2024) \cite{low_repairing_2024} przeprowadzili badania nad automatycznym naprawianiem skryptów Terraform (IaC dla chmury) z wykorzystaniem modeli GPT-3.5 oraz GPT-4. Zaprezentowali oni podejście, w którym LLM otrzymuje jako wejście kod infrastruktury oraz raport z narzędzia skanującego (np. listę wykrytych podatności czy niezgodności z politykami bezpieczeństwa) i na tej podstawie generuje poprawioną wersję kodu. Co ważne, autorzy wprowadzili strategię "human-in-the-loop", w której proces naprawy odbywa się w dwóch etapach: w pierwszym model stara się samodzielnie usunąć większość problemów, a w drugim programista może dostarczyć dodatkowych informacji lub kontekstu, aby pomóc modelowi naprawić trudniejsze błędy. Taka iteracyjna pętla ma na celu uzupełnienie brakującej wiedzy – wiele błędów IaC wymaga bowiem informacji o zewnętrznym kontekście (np. identyfikatorów zasobów w chmurze), które nie są zawarte bezpośrednio w kodzie.

Eksperymenty autorów wykazały, że model GPT-4 znacząco przewyższa GPT-3.5 w skuteczności naprawy błędów konfiguracyjnych. Już po pierwszym przejściu GPT-4 usuwał średnio 18–34\% więcej wykrytych niezgodności niż GPT-3.5, a po drugim przejściu różnica ta sięgała nawet 57\% na korzyść GPT-4 (dla niektórych narzędzi skanujących). W najlepszym scenariuszu model GPT-4 był w stanie zredukować liczbę alarmów o błędnej konfiguracji aż o 84,7\% w stosunku do stanu początkowego, znacząco odciążając programistę. Uwzględnienie interwencji człowieka (czyli przekazanie modelowi dodatkowego kontekstu w drugiej iteracji) podniosło odsetek poprawnie usuniętych błędów łącznie o dalsze 22–60\%, osiągając maksymalnie 87,4\% naprawionych mis-konfiguracji (dla problemów wykrywanych przez skaner Checkov \cite{checkov}).

Mimo tych imponujących wyników, analiza jakości wygenerowanych poprawek ujawniła pewne ograniczenia i wyzwania. Około 20,4\% zaproponowanych przez model poprawek okazało się nietrafnych – zawierały one błędy składniowe lub nie rozwiązywały faktycznego problemu, a jedynie „uciszały” ostrzeżenie skanera powierzchowną zmianą. Stwierdzono, że model ma tendencję do halucynowania pól konfiguracji, tzn. dodawania ustawień, które syntaktycznie pasują do kodu i powodują zniknięcie ostrzeżenia, lecz w rzeczywistości nie mają pokrycia w dokumentacji (nie istnieją) bądź nie poprawiają bezpieczeństwa. Przykładowo, LLM potrafił zaproponować pole, które „naprawia” błąd według skanera, ale to pole nie jest obsługiwane przez dany zasób, przez co wynikowy kod nie przejdzie walidacji schematu lub nadal nie spełnia zamierzonej polityki bezpieczeństwa.

Takie zachowanie jest szczególnie niebezpieczne, gdyż generuje fałszywe poczucie poprawy – konfiguracja wygląda na poprawioną (skanery nie zgłaszają błędu), podczas gdy problem nadal istnieje lub pojawiają się inne błędy. Autorzy zaproponowali kilka potencjalnych rozwiązań tych problemów, w tym bardziej restrykcyjne prompty oraz dodatkowe automatyczne testy wygenerowanych poprawek (np. walidacja syntaktyczna i porównanie z intencją oryginalnego kodu). Praca Low i in. stanowi ważną wskazówkę, że choć LLM potrafią znacznie przyspieszyć naprawę infrastruktury jako kodu, to nie można polegać na nich bezkrytycznie – konieczne jest włączenie mechanizmów weryfikujących i korygujących ich propozycje.

Kolejną kluczową kwestią badaną w literaturze jest bezpieczeństwo konfiguracji Kubernetes generowanych przez LLM. Manifesty K8s mogą zawierać subtelne błędy konfiguracyjne (np. brak ograniczeń zasobów, użycie uprzywilejowanych kontenerów, ekspozycja usług bez autoryzacji itp.), które nie wpływają od razu na działanie aplikacji, ale stanowią poważne ryzyko bezpieczeństwa. Malul i in. (2024) \cite{malul_genkubesec_2024} zajęli się problemem wykrywania i automatycznej naprawy niepoprawnych konfiguracji Kubernetes przy użyciu LLM, przedstawiając kompleksowe rozwiązanie o nazwie GenKubeSec.

W odróżnieniu od wcześniej opisanych podejść, skupionych na generowaniu nowych konfiguracji, GenKubeSec zakłada istnienie pewnej bazy wiedzy o znanych błędach Kubernetes i wykorzystuje model językowy do odnajdywania tych błędów w dostarczonych plikach oraz sugerowania poprawek. Architektura GenKubeSec składa się z trzech głównych komponentów: (1) modułu przygotowania danych, gdzie zebrano obszerny zestaw znanych przypadków błędnych konfiguracji Kubernetes (Kubernetes Configuration Faults, KCF) i zunifikowano ich kategoryzację za pomocą ujednoliconego indeksu mis-konfiguracji (Unified Misconfig Index); (2) GenKubeDetect, czyli modelu LLM dostrojonego (fine-tuned) na zebranych danych, zdolnego do wykrywania różnorodnych błędów w konfiguracji Kubernetes; oraz (3) GenKubeResolve, który wykorzystuje pretrenowany model (bez dodatkowego trenowania) wraz z technikami prompt engineering i few-shot learning, aby dla każdego wykrytego błędu podać jego lokalizację w pliku, zrozumiałe wyjaśnienie oraz zaproponować konkretną poprawkę.

Takie podejście pozwala na kompleksową obsługę problemów – od detekcji po naprawę – w ramach jednego systemu, podczas gdy wcześniejsze prace zazwyczaj koncentrowały się albo na samym wykrywaniu, albo tylko na sugerowaniu poprawek. Efektywność GenKubeSec oceniono porównawczo z istniejącymi narzędziami do statycznej analizy manifestów K8s (m.in. Checkov, KubeLinter \cite{kubelinter}, Terrascan \cite{terrascan}). Wyniki są bardzo obiecujące: dostrojony model GenKubeDetect osiągnął precyzję na poziomie 0,990 oraz czułość (recall) 0,999 w wykrywaniu błędów KCF. Oznacza to, że niemal wszystkie rzeczywiste błędy zostały wykryte (tylko 0,1\% umknęło uwadze modelu), przy znikomym odsetku fałszywych alarmów.

Dla porównania, sumaryczna precyzja trzech popularnych narzędzi opartych na regułach była zbliżona (również około 0,99), ale każde z nich z osobna miało luki (błędy wykryte przez GenKubeSec wykraczały poza ich pokrycie). Świadczy to o ogólnej zdolności modelu do uogólniania wiedzy – jest w stanie wykryć także te niepoprawności, których nie obejmują zdefiniowane statycznie reguły. Co więcej, przy pomocy modułu GenKubeResolve system generował dla każdego błędu klarowne wyjaśnienie i propozycję naprawy, które zostały zweryfikowane przez ekspertów Kubernetes jako trafne i poprawne.

Przykładowo, jeśli w manifeście użyto domyślnej przestrzeni nazw (co jest uznawane za złą praktykę), GenKubeSec wskazywał dokładnie tę linię, wyjaśniał dlaczego domyślna przestrzeń nie powinna być używana i sugerował dodanie konkretnej deklaracji namespace lub zmianę na dedykowaną przestrzeń.

Istotnym elementem wdrożeniowym GenKubeSec jest to, że działa on w oparciu o lokalnie uruchomiony model LLM, a nie poprzez zapytania do zewnętrznego API modelu. Dzięki temu rozwiązanie minimalizuje ryzyko wycieku danych wrażliwych – pliki konfiguracyjne (które mogą zawierać np. informacje o architekturze systemu, nazwy usług, a nawet niektóre dane dostępowe) nie są nigdzie wysyłane, co mogłoby stwarzać zagrożenie bezpieczeństwa lub prywatności. Twórcy podkreślają tę zaletę, zauważając że wiele firm obawia się korzystać z chmurowych API LLM właśnie z uwagi na konieczność przesyłania swojej konfiguracji na zewnątrz. GenKubeSec pokazuje, że możliwe jest skuteczne połączenie technik uczenia maszynowego z praktycznymi wymaganiami bezpieczeństwa w środowisku korporacyjnym.

\subsection{Bezpieczeństwo zastosowań LLM w IaC}

Bezpieczeństwo jest jednym z kluczowych aspektów przy wdrażaniu dużych modeli językowych do praktyk DevOps i Infrastructure as Code. W literaturze wskazuje się szereg zagrożeń, które mogą pojawić się zarówno po stronie jakości generowanego kodu, jak i podatności samych aplikacji wykorzystujących LLM.

Fu i in. (2025) \cite{fu_security_2025} przeprowadzili analizę bezpieczeństwa kodu generowanego przez LLM na podstawie repozytoriów GitHub. Wykazali oni, że znaczący odsetek fragmentów kodu zawiera podatności bezpieczeństwa, w tym takie, które zostały zaklasyfikowane do listy CWE Top-25 \cite{cwetop25}. Wyniki te wskazują na potrzebę weryfikacji nie tylko konfiguracji infrastruktury, ale również ogólnej jakości kodu produkowanego przez LLM, który może zostać wdrożony bez uprzedniego audytu.

Liu i in. (2024) \cite{liu_prompt_2024} z kolei zaprezentowali technikę ataku prompt injection w aplikacjach wykorzystujących LLM, wskazując na potencjalne wektory zagrożeń, które mogą zostać wykorzystane do manipulacji generowanymi wynikami. Jest to szczególnie istotne w kontekście aplikacji DevOps, gdzie złośliwy użytkownik mógłby wpłynąć na wygenerowaną konfigurację systemu, np. poprzez wstrzyknięcie niepożądanych instrukcji do promptu lub danych wejściowych.

Opisane ryzyka podkreślają konieczność uzupełnienia procesu integracji LLM w IaC o dodatkowe mechanizmy zabezpieczające – takie jak lokalna walidacja, sandboxing, inspekcja promptów, czy użycie oddzielnych mechanizmów inspekcji i audytu konfiguracji po ich wygenerowaniu.

\subsection{Wyzwania i ograniczenia}

Na podstawie przeglądu literatury można stwierdzić, że zastosowanie dużych modeli językowych w dziedzinie Infrastructure as Code niesie ze sobą znaczące korzyści, ale także wiąże się z istotnymi wyzwaniami. Do głównych zalet zaliczymy:

\textbf{Automatyzacja i oszczędność czasu:} LLM są w stanie automatycznie wygenerować znaczną część kodu konfiguracyjnego, począwszy od definicji obrazu Docker, po złożone manifesty Kubernetes. Umożliwia to przyspieszenie wdrożeń – konfiguracje, które normalnie zajęłyby godziny ręcznego pisania i debugowania, mogą zostać uzyskane w ciągu minut. Przykładowo, projekt Repo2Run pokazał, że LLM może samodzielnie przygotować Dockerfile dla skomplikowanego projektu, eliminując dziesiątki potencjalnych kroków manualnych \cite{hu_llm-based_2025}. Automatyzacja przekłada się również na odciążenie specjalistów: deweloper bez głębokiej wiedzy o Kubernetes może uzyskać działający manifest, skupiając się na wysokopoziomowych wymaganiach, podczas gdy szczegółowy kod wygeneruje za niego model. To obniża próg wejścia – zespoły mogą szybciej adoptować technologie chmurowe bez długiego szkolenia z każdego narzędzia.

\textbf{Poprawa produktywności i jakości:} Dzięki LLM możliwe jest wygenerowanie szablonów konfiguracji zgodnych z ogólnie przyjętymi wzorcami i dobrymi praktykami. Dobrze skonstruowany prompt może spowodować, że model uwzględni zalecane ustawienia bezpieczeństwa czy optymalizacji (np. doda limity zasobów w manifeście, utworzy oddzielne sieci dla kontenerów, itp.). W badaniach odnotowano, że LLM potrafią uzupełniać brakujące elementy konfiguracji, zmniejszając ryzyko pominięcia istotnego parametru \cite{kratzke_dont_2024}. Ponadto modele takie jak GPT-4 wykazały zdolność do korygowania istniejącego kodu – co oznacza, że mogą nie tylko generować, ale i naprawiać konfiguracje, redukując liczbę błędów przed wdrożeniem \cite{low_repairing_2024}. To wszystko przekłada się na wyższą jakość infrastruktury: bardziej spójne, przemyślane i bezpieczne konfiguracje.

\textbf{Ujednolicenie i dokumentacja wiedzy:} LLM trenują się na ogromnych zbiorach danych, które zawierają również wiedzę ekspercką rozsianą po dokumentacjach, blogach, forach. Wykorzystując model, możemy niejako skondensować tę wiedzę w generowanych konfiguracjach. Modele mogą wskazywać rozwiązania podpatrzone w wielu źródłach, co jest korzystne zwłaszcza w małych zespołach bez dedykowanych specjalistów od DevOps. Dodatkowo, niektóre prace sugerują, że generowane przez LLM wyjaśnienia (np. moduł GenKubeResolve) mogą służyć jako dokumentacja edukacyjna – model tłumaczy, dlaczego coś jest błędem i jak to naprawić \cite{malul_genkubesec_2024}. To może pomóc zespołom lepiej zrozumieć własną infrastrukturę i uczyć się na bieżąco dobrych praktyk.

Z drugiej strony, pojawia się szereg wyzwań i ryzyk, które należy rozważyć wdrażając LLM do generowania IaC:

\textbf{Halucynacje i błędne konfiguracje:} Duże modele językowe czasem z dużą pewnością podają informacje nieprawdziwe lub nieadekwatne. W kontekście IaC może to oznaczać wygenerowanie parametrów, które wydają się poprawne, ale w rzeczywistości są błędne lub nieistnieją w danej technologii (np. wymyślone pola w manifeście Kubernetes czy atrybuty zasobów Terraform). Tego typu halucynacje są groźne, ponieważ mogą nie zostać natychmiast wychwycone – plik konfiguracyjny może przejść prostą walidację składni, ale dopiero przy wdrożeniu ujawni się problem, albo co gorsza, błąd pozostanie ukryty jako „tykająca bomba”. Rozwiązaniem może być wzbogacenie procesu generowania o walidatory, testy lub drugi model weryfikujący, jednak zwiększa to złożoność całego potoku wytwórczego \cite{low_repairing_2024}.

\textbf{Brak gwarancji spełnienia wymagań:} Obecne modele działają jak „czarne skrzynki” – trudno przewidzieć, czy wygenerowany kod w pełni spełni założenia użytkownika. Ueno i Uchiumi \cite{ueno_migrating_2024} wskazali, że brak jest mechanizmu pewnej weryfikacji, czy output modelu pokrywa wszystkie elementy specyfikacji wejściowej i czyni to we właściwy sposób. W praktyce oznacza to konieczność ręcznego przeglądu i testów. Nadal jednak odpowiedzialność za ostateczną poprawność spoczywa na człowieku. LLM mogą nie uwzględnić pewnych niuansów biznesowych czy kontekstowych, których nie było w opisie – np. mogą domyślnie otworzyć port usługi, mimo że według wymagań bezpieczeństwa powinien on być ograniczony.

\textbf{Bezpieczeństwo i zgodność:} Paradoksalnie, narzędzia służące poprawie bezpieczeństwa (jak GenKubeSec) same muszą być zaadresowane pod kątem bezpieczeństwa. Jedno zagrożenie to kwestia poufności danych – jeśli korzystamy z zewnętrznej usługi LLM (np. API chmurowego), musimy wysłać do niej nasze pliki konfiguracyjne, które mogą zawierać wrażliwe informacje o systemie. W środowiskach korporacyjnych często jest to nieakceptowalne, dlatego preferowane może być użycie lokalnych instancji modeli \cite{malul_genkubesec_2024}. Kolejna sprawa to bezpieczeństwo generowanych konfiguracji: model może nieświadomie zaproponować ustawienia z lukami (np. kontener uruchamiany z uprawnieniami root, brak szyfrowania komunikacji, itp.). Wygenerowany kod koniecznie musi przejść przez standardowe procedury audytu bezpieczeństwa tak samo, jak kod pisany ręcznie.

\textbf{Niezawodność i deterministyczność:} Modele językowe mogą dawać różne wyniki na podstawie nawet subtelnych różnic w promptach. Oznacza to, że proces generowania konfiguracji może być nie zawsze powtarzalny – dwie iteracje mogą wygenerować nieco inny manifest. W kontekście IaC oczekujemy deterministycznych rezultatów. Konieczne jest więc odpowiednie strojenie parametrów modelu (np. temperature=0) oraz standaryzacja promptów, by wyniki były jak najbardziej spójne i przewidywalne. Dodatkowo, jeśli integrujemy LLM w potoki CI/CD, musimy brać pod uwagę jego dostępność i czas odpowiedzi – model pracujący na dużym manifeście może mieć opóźnienia lub być ograniczony limitami API.

\subsection{Podsumowanie}

Podsumowując, literatura wskazuje, że duże modele językowe posiadają już zdolności umożliwiające znaczne usprawnienie procesu tworzenia i utrzymania infrastruktury jako kodu. Udane demonstracje – takie jak automatyczne wygenerowanie działającego Dockerfile dla setek projektów \cite{hu_llm-based_2025}, czy wygenerowanie manifestu Kubernetes pozwalającego na wdrożenie aplikacji bez ręcznej konfiguracji \cite{kratzke_dont_2024} – dowodzą, że automatyzacja DevOps z pomocą AI jest możliwa i może przynieść realne oszczędności czasu oraz redukcję błędów.

Jednocześnie jednak żadne z rozwiązań nie jest pozbawione wad. Wyzwania związane z halucynacjami modeli \cite{low_repairing_2024}, koniecznością walidacji wyników \cite{ueno_migrating_2024, kon_iac-eval_nodate} i zapewnieniem bezpieczeństwa \cite{malul_genkubesec_2024, fu_security_2025} wskazują, że rola człowieka (eksperta DevOps) wciąż pozostaje ważna jako nadzorcy i korektora działania AI. Dlatego bieżące badania kierują się ku metodom łączenia silnych stron LLM z mechanizmami kontrolnymi – np. stosowanie podejścia human-in-the-loop \cite{low_repairing_2024}, rozwijanie benchmarków oceny jakości (aby móc mierzyć postępy i porównywać modele) \cite{ueno_migrating_2024, kon_iac-eval_nodate}, eksplorowanie technik prompt engineering \cite{kratzke_dont_2024, pujar_invited_2023}, a także opracowanie metod ochrony przed atakami prompt injection \cite{liu_prompt_2024}, które zmniejszą skłonność modeli do popełniania typowych błędów lub podatności.

Niniejsza praca magisterska wpisuje się w ten nurt, koncentrując się na zbudowaniu automatycznego narzędzia do generowania konfiguracji Docker i Kubernetes wyłącznie na podstawie repozytorium kodu źródłowego, bez udziału człowieka. W ramach pracy zostanie przeprowadzona analiza i porównanie skuteczności różnych dużych modeli językowych w tym zadaniu. Szczególny nacisk położony będzie na bezpieczeństwo generowanej konfiguracji, poprawność składniową i funkcjonalną manifestów oraz na analizę potencjalnych zagrożeń, takich jak ataki typu prompt injection czy przez złośliwie spreparowane repozytoria. Ostatecznie projekt zostanie rozszerzony o prototyp systemu typu PaaS, umożliwiającego automatyczne wdrażanie aplikacji na podstawie kodu źródłowego, bez konieczności ingerencji człowieka.
\clearpage % Rozdziały zaczynamy od nowej strony.

\section{Przegląd technologii i narzędzi}

Celem tego rozdziału jest przedstawienie najważniejszych technologii i narzędzi, stanowiących podstawę niniejszego badania. Szczególny nacisk położony zostanie na duże modele językowe, opis ich architektur oraz ich zastosowanie w generowaniu konfiguracji wdrożeniowej. Ponadto, omówione zostaną istotne narzędzia DevOps, takie jak Docker i Kubernetes, niezbędne w zarządzaniu środowiskami uruchomieniowymi, oraz narzędzia służące do walidacji i oceny jakości generowanego kodu IaC. Rozdział zawiera także opis środowiska eksperymentalnego oraz wykorzystywanych technologii wspierających automatyzację i monitorowanie eksperymentów.

\subsection{Modele językowe wykorzystywane w badaniu}

Duże modele językowe pełnią istotną rolę w prezentowanych eksperymentach, służąc jako generatory konfiguracji IaC. Ich zdolność do przetwarzania języka naturalnego oraz generowania poprawnego kodu czyni je ważnym elementem automatyzacji procesów DevOps. W tym podrozdziale przedstawiono charakterystykę wybranych modeli, opis ich architektur oraz praktyczne aspekty związane z ich wykorzystaniem.

Rynek dużych modeli językowych rozwija się dynamicznie, oferując zarówno modele komercyjne, dostępne przez API, jak i modele open-source, możliwe do uruchomienia lokalnie lub na własnej infrastrukturze. Decyzja o wyborze konkretnego rodzaju modelu często zależy od dostępności zasobów sprzętowych (zwłaszcza GPU), wymagań dotyczących licencji, kwestii związanych z prywatnością danych oraz możliwości dostosowania modeli do specyficznych zastosowań.

Modele komercyjne dostępne przez API zazwyczaj cechują się wysoką wydajnością, zaawansowanymi możliwościami oraz niezawodnością, dzięki wsparciu dużych firm technologicznych takich jak OpenAI, Anthropic czy Google. Modele te trenowane są na szerokich i zróżnicowanych zbiorach danych, co przekłada się na ich zdolność generowania wysokiej jakości tekstu oraz kodu. Ich użytkowanie wiąże się z kosztami dostępu do API oraz koniecznością przesyłania danych do zewnętrznych serwisów, co może być problematyczne w przypadku danych wrażliwych.

Wybrane do badania modele komercyjne to:

\begin{itemize}
    \item \textbf{OpenAI GPT-5} \cite{gpt5}: TODO

    \item \textbf{OpenAI GPT-5 Mini} \cite{gpt5mini}: TODO

    \item \textbf{OpenAI o3} \cite{o3}: TODO

    \item \textbf{Anthropic Claude Sonnet 4.5} \cite{claude_sonnet45}: TODO

    \item \textbf{Anthropic Claude Haiku 4.5} \cite{claude_haiku45}: TODO

    \item \textbf{Anthropic Claude Opus 4.1} \cite{claude_opus41}: TODO

    \item \textbf{Google Gemini 2.5 Pro} \cite{gemini25_pro}: TODO

    \item \textbf{Google Gemini 2.5 Flash} \cite{gemini25_flash}: TODO
\end{itemize}

Modele open-source stanowią atrakcyjną alternatywę, oferując pełną kontrolę, możliwość uruchomienia na własnej infrastrukturze oraz brak dodatkowych opłat za każde zapytanie. Ich wydajność zależy przede wszystkim od dostępności zasobów sprzętowych, a aktywny rozwój społeczności przyczynia się do ciągłej optymalizacji i pojawiania się coraz bardziej zaawansowanych wersji.

Wybrane do badania modele open-source to:

\begin{itemize}
    \item \textbf{Qwen 3} \cite{qwen3}: TODO

    \item \textbf{DeepSeek R1} \cite{deepseek_r1}: TODO

    \item \textbf{DeepSeek V3.2} \cite{deepseek_v32}: TODO

    \item \textbf{GLM 4.6} \cite{glm46}: TODO

    \item \textbf{Mistral Medium} \cite{mistral_medium}: TODO

    \item \textbf{Meta Llama 4 Maverick} \cite{llama4_maverick}: TODO

    \item \textbf{Meta Llama 4 Scout} \cite{llama4_scout}: TODO
\end{itemize}

Wszystkie wymienione modele reprezentują aktualny szczyt osiągnięć w dziedzinie dużych modeli językowych. Wybrane zostały przede wszystkim ze względu na ich potwierdzone zdolności do generowania wysokiej jakości kodu oraz wsparcie dla funkcji \textit{tool calling}. Funkcje te umożliwiają bezpośrednią integrację modeli z zewnętrznymi narzędziami lub usługami poprzez dedykowane mechanizmy API bądź platformy agentowe, takie jak LangChain \cite{langchain} czy LangGraph \cite{langgraph}. W proponowanym w tej pracy podejściu, funkcje \textit{tool calling} będą odgrywać znaczącą rolę w zarządzaniu kontekstem i procesie generowania konfiguracji.

\subsection{Architektury i charakterystyka}

Większość współczesnych dużych modeli językowych bazuje na architekturze transformera \cite{vaswani_attention_2023}, wykorzystując mechanizm uwagi (attention mechanism) do efektywnego przetwarzania sekwencji danych. Architektury te, pomimo wspólnych założeń, różnią się między sobą liczbą parametrów, rozmiarem okna kontekstowego oraz specyficznymi optymalizacjami, które wpływają na ich zdolność do generowania kodu i rozumienia złożonych instrukcji. Coraz większe znaczenie zyskują również alternatywne podejścia architektoniczne, takie jak Mixture-of-Experts (MoE), w których przy każdym zapytaniu aktywowane są tylko wybrane części modelu, co pozwala na zwiększenie efektywności obliczeniowej bez pogarszania jakości generowanych wyników.

Poniżej przedstawiono najważniejsze cechy różnicujące współczesne modele językowe, mające istotny wpływ na ich zachowanie i przydatność w zadaniach związanych z generowaniem kodu:

\begin{itemize}
    \item \textbf{Liczba parametrów:} Liczba parametrów (często liczona w miliardach) jest wskaźnikiem skali modelu i jego zdolności do uczenia się złożonych wzorców z danych. Modele z większą liczbą parametrów zazwyczaj wykazują lepszą wydajność w szerokim zakresie zadań, jednak ich uruchomienie i obsługa wymagają znacznie większych zasobów obliczeniowych. Należy jednak zaznaczyć, że sama liczba parametrów nie jest jedynym wyznacznikiem jakości; znaczenie ma również jakość danych treningowych, architektura (np. Mixture-of-Experts) oraz proces dostrajania modelu.
    \item \textbf{Okno kontekstowe:} Okno kontekstowe odnosi się do maksymalnej długości sekwencji tokenów (słów, znaków, fragmentów kodu), którą model może przetworzyć i wykorzystać do wygenerowania odpowiedzi. Jest to szczególnie istotne w kontekście generowania konfiguracji IaC z repozytoriów kodu. Duże okno kontekstowe pozwala na dostarczenie modelowi całych plików źródłowych aplikacji, fragmentów dokumentacji, wielu powiązanych ze sobą promptów, a nawet wyników walidacji z zewnętrznych narzędzi. To umożliwia agentowi LLM holistyczne zrozumienie projektu i kontekstu, co przekłada się na wyższą jakość i trafność generowanych konfiguracji. Modele takie jak Google Gemini 2.5 Pro wyróżniają się wyjątkowo dużym oknem kontekstowym (do miliona tokenów), co jest znaczącą przewagą w zadaniach wymagających głębokiej analizy kodu i dokumentacji.
    \item \textbf{Specjalizacje i optymalizacje:} Niektóre modele, takie jak Mistral Codestral, są specjalizowane lub fine-tuninguowane w kierunku generowania kodu. Oznacza to, że są one trenowane na dużych zbiorach danych zawierających kod programistyczny, co poprawia ich zdolność do generowania syntaktycznie poprawnego i semantycznie trafnego kodu. Architektury takie jak Mixture-of-Experts (MoE) stosowane np. w DeepSeek V3 pozwalają na efektywne skalowanie modeli, aktywując tylko część ekspertów dla danego zapytania, co optymalizuje zużycie zasobów przy zachowaniu wysokiej jakości odpowiedzi.
\end{itemize}

Poniższa tabela \ref{tab:llm-characteristic} przedstawia ogólne charakterystyki wybranych modeli LLM, które będą wykorzystywane w badaniu. Okno kontekstu podano w liczbie tokenów, a liczba parametrów — w miliardach, jeśli została opublikowana.

\begin{table}[!h] \centering
\caption{Charakterystyka wybranych modeli LLM}
\label{tab:llm-characteristic}
\begin{tabular}{| c | c | c | c |} \hline
\textbf{Dostawca} & \textbf{Model} & \textbf{Parametry (mld)} & \textbf{Okno kontekstu (tokeny)} \\ \hline\hline
OpenAI & GPT-5 & - & TODO \\ \hline
OpenAI & GPT-5 Mini & - & TODO \\ \hline
OpenAI & o3 & - & TODO \\ \hline
Anthropic & Claude Sonnet 4.5 & - & TODO \\ \hline
Anthropic & Claude Haiku 4.5 & - & TODO \\ \hline
Anthropic & Claude Opus 4.1 & - & TODO \\ \hline
Google & Gemini 2.5 Pro & - & TODO \\ \hline
Google & Gemini 2.5 Flash & - & TODO \\ \hline
Qwen & Qwen 3 & TODO & TODO \\ \hline
DeepSeek & DeepSeek R1 & TODO & TODO \\ \hline
DeepSeek & DeepSeek V3.2 & TODO & TODO \\ \hline
Zhipu AI & GLM 4.6 & TODO & TODO \\ \hline
Mistral AI & Mistral Medium & TODO & TODO \\ \hline
Meta & Llama 4 Maverick & TODO & TODO \\ \hline
Meta & Llama 4 Scout & TODO & TODO \\ \hline
\end{tabular}
\end{table}

\subsection{Sposoby wykorzystania modeli}

W ramach niniejszego badania modele językowe są wykorzystywane w trybie inferencji (wnioskowania), bez dodatkowego etapu ich dostrajania (fine-tuningu) na specyficznym zbiorze danych kodu infrastruktury. Oznacza to, że modele wykorzystują swoją ogólną wiedzę, nabytą podczas szerokiego treningu, do generowania konfiguracji na podstawie dostarczonego kontekstu i promptów. Taki sposób podejścia jest zgodny z koncepcją "Don't Train, Just Prompt" \cite{kratzke_dont_2024}, co pozwala na szybką iterację i wykorzystanie najnowszych, potężnych modeli bez konieczności kosztownego i czasochłonnego procesu ponownego trenowania.

Integracja z wybranymi modelami LLM odbywa się głównie za pośrednictwem ich bezpośrednich interfejsów programistycznych (API). Takie podejście gwarantuje dostęp do najnowszych wersji modeli, pełną kontrolę nad parametrami zapytania oraz optymalne czasy odpowiedzi. Należy jednak pamiętać o potencjalnych ograniczeniach wynikających z limitów API oraz braku możliwości bezpośredniej modyfikacji wewnętrznej architektury lub kodu źródłowego w przypadku zamkniętych modeli komercyjnych.

Warto zaznaczyć, że chociaż inferencję dla wielu modeli open-source (takich jak te z serii LLaMA czy Mistral) można przeprowadzać lokalnie, wymaga to znacznych zasobów sprzętowych (zwłaszcza mocnego procesora graficznego - GPU). Ze względu na ograniczenia sprzętowe środowiska eksperymentalnego, w niniejszej pracy nie zastosowano lokalnej inferencji. Zamiast tego, dla modeli komercyjnych korzystano z ich dedykowanych API, natomiast dla modeli open-source, które nie oferują własnego API, wykorzystana została platforma OpenRouter \cite{openrouter}, która agreguje dostęp do wielu modeli od różnych dostawców, oferując ujednolicony interfejs.

\subsection{Tryby promptowania}

Efektywność wykorzystania dużych modeli językowych w dużej mierze zależy od sposobu formułowania zapytań (promptów) \cite{kratzke_dont_2024}. Istnieje kilka podstawowych strategii promptowania, które mogą być stosowane w zależności od złożoności zadania i dostępności przykładów:

\begin{itemize}
	\item \textbf{Zero-shot prompting:} Jest to najprostsza forma promptowania, gdzie model otrzymuje zadanie lub pytanie bez żadnych wcześniejszych przykładów. Oczekuje się, że model, bazując na swojej wiedzy ogólnej, wygeneruje odpowiedź. W kontekście generowania IaC, model otrzymałby instrukcję typu "Wygeneruj Dockerfile dla aplikacji Node.js", bez dodatkowych danych kontekstowych poza samym opisem zadania. Jest to podejście szybkie, ale jego skuteczność jest często ograniczona, zwłaszcza w przypadku złożonych lub bardzo specyficznych wymagań.
	\item \textbf{Few-shot prompting:} Ta strategia polega na dostarczeniu modelowi kilku przykładów par (zapytanie, oczekiwana odpowiedź) przed właściwym zadaniem \cite{brown_language_2020} Przykłady te służą jako demonstracja pożądanego formatu i stylu odpowiedzi, pomagając modelowi zrozumieć intencję użytkownika i dostosować swoje generacje. W przypadku IaC, można by podać kilka par "opis aplikacji \textrightarrow{} przykładowy Dockerfile/manifest Kubernetes", aby model nauczył się wzorca i preferowanych praktyk. Zastosowanie tej metody często zwiększa jakość generacji w porównaniu do zero-shot.
	\item \textbf{Chain-of-Thought (CoT) prompting:} Jest to technika, w której prompt instruuje model, aby przed podaniem ostatecznej odpowiedzi, wygenerował sekwencję pośrednich kroków rozumowania. Model "myśli na głos", co często prowadzi do bardziej precyzyjnych i trafnych wyników, szczególnie w zadaniach wymagających złożonych rozumowań lub wieloetapowych rozwiązań. W początkowej fazie rozwoju LLM technika CoT często wymagała dedykowanych instrukcji w prompcie, aby model wygenerował te kroki. Obecnie wiele zaawansowanych modeli językowych, dzięki swojej architekturze i procesom treningowym, potrafi wykazywać tego typu "rozumowanie" w sposób bardziej intuicyjny, często nawet bez wyraźnego polecenia, jeśli kontekst zadania na to wskazuje. W kontekście IaC, model mógłby najpierw "zastanowić się" nad zależnościami aplikacji, potem nad wymaganiami środowiskowymi i optymalizacjami, a dopiero potem wygenerować kod konfiguracji. CoT poprawia transparentność i debugowalność procesu generowania, a także może zwiększyć poprawność odpowiedzi.
	\item \textbf{Agent-based prompting:} Ten tryb wykracza poza jednorazowe generowanie odpowiedzi. Model, działając jako "agent", otrzymuje zadanie i może iteracyjnie wykonywać szereg akcji: planować, używać zewnętrznych narzędzi (tzw. "tool calling"), analizować ich wyniki i modyfikować swój plan działania w celu osiągnięcia celu. Podejście agentowe można postrzegać jako ewolucję i rozszerzenie idei Chain-of-Thought, gdzie "myślenie" modelu jest przekształcane w sekwencję interakcji ze światem zewnętrznym i refleksji. Agent nie tylko generuje wewnętrzne kroki rozumowania, ale także aktywnie wykorzystuje narzędzia do zbierania danych, weryfikacji hipotez i korygowania swojego działania na podstawie rzeczywistych wyników. Jest to podejście szczególnie efektywne w przypadku złożonych problemów wymagających interakcji ze środowiskiem zewnętrznym oraz adaptacji.
\end{itemize}

W kontekście generowania konfiguracji wdrożeniowej, zastosowanie podejścia agentowego jest uzasadnione ze względu na możliwość kompleksowego zarządzania kontekstem i procesem generacji. Generowanie IaC często wymaga dostępu do wielu źródeł informacji i precyzyjnego umieszczenia danych w konfiguracji. W przeciwieństwie do jednorazowych generacji (jak w zero-shot czy few-shot), agent może, na podstawie dostarczonego mu opisu docelowej infrastruktury, wygenerować wstępną wersję konfiguracji (np. Dockerfile). Następnie, istotnym elementem jest możliwość wykorzystania zewnętrznych narzędzi do interakcji z kontekstem. Agent może na przykład czytać wybrane pliki z repozytorium kodu, aby lepiej zrozumieć strukturę projektu i zależności. Może również sterować swoim wewnętrznym przepływem rozumowania i zakresem czytanego kontekstu.

W niniejszej pracy magisterskiej zastosowana zostanie właśnie metoda agentowa. Podejście agentowe może być realizowane na różne sposoby, zarówno poprzez bezpośrednie wykorzystanie dedykowanych interfejsów API udostępnianych przez niektóre modele (np. poprzez \textit{tool calling} czy \textit{function calling}), jak i poprzez zastosowanie specjalizowanych frameworków orkiestracyjnych, takich jak \textbf{AutoGen} \cite{autogen} czy \textbf{LangChain/LangGraph}. W tym badaniu, do implementacji agentów i zarządzania złożonymi przepływami pracy z wykorzystaniem funkcji \textit{tool calling} wybrana została biblioteka LangGraph w języku Python, uzupełniona o funkcjonalności LangChain. Rola agenta koncentruje się na efektywnym dostarczaniu kontekstu i zarządzaniu procesem generacji konfiguracji IaC.

\subsection{Narzędzia DevOps: Docker i Kubernetes}

Zarządzanie środowiskami uruchomieniowymi i wdrażanie aplikacji w chmurze opiera się współcześnie na narzędziach takich jak Docker i Kubernetes. Niniejszy podrozdział opisuje te technologie, ich rolę w cyklu DevOps oraz potencjalne pułapki konfiguracyjne, które mogą być generowane przez duże modele językowe. Zrozumienie ich funkcjonalności i typowych błędów jest istotne dla oceny jakości generowanych przez LLM konfiguracji IaC.

\textbf{Docker} to platforma do tworzenia, wdrażania i uruchamiania aplikacji w kontenerach. Konteneryzacja pozwala na pakowanie aplikacji wraz z jej zależnościami i bibliotekami w izolowane środowiska, co zapewnia spójność działania niezależnie od środowiska docelowego. Docker jest wykorzystywany w wielu obszarach: od uruchamiania lokalnych usług deweloperskich i testowych, przez konteneryzację narzędzi, aż po konteneryzację produkcyjnych aplikacji w złożonych architekturach mikroserwisowych.

Tworzenie plików Dockerfile – deklaratywnych instrukcji do budowania obrazów Docker – wymaga precyzji i znajomości najlepszych praktyk. Modele LLM, generując Dockerfile, mogą popełnić szereg błędów, które wpływają na bezpieczeństwo, wydajność lub poprawne działanie kontenera. Przykłady takich błędów to:
\begin{itemize}
\item \textbf{Użycie ADD zamiast COPY:} ADD automatycznie rozpakowuje archiwa i może pobierać pliki z URL, co jest mniej transparentne i może wprowadzać nieoczekiwane zależności lub luki bezpieczeństwa. Preferowane jest jawne COPY dla lokalnych plików.
\item \textbf{Brak WORKDIR lub jego nieoptymalne użycie:} Nieokreślenie katalogu roboczego lub jego wielokrotne zmienianie może prowadzić do nieczytelnych i nieefektywnych Dockerfile.
\item \textbf{Użycie tagu latest dla obrazów bazowych:} Obrazy z tagiem latest są zmienne, co prowadzi do braku determinizmu w budowaniu aplikacji. Powoduje to, że ta sama aplikacja może nie budować się w przyszłości w ten sam sposób lub z tym samym obrazem bazowym. Zalecane jest używanie konkretnych wersji obrazów (np. node:20.10-slim).
\item \textbf{Brak usuwania zbędnych zależności i plików po instalacji:} Pozostawianie pakietów użytych tylko do budowy lub tymczasowych plików zwiększa rozmiar obrazu i powierzchnię ataku (ang. attack surface).
\item \textbf{Uruchamianie aplikacji jako root:} Domyślne uruchamianie procesów w kontenerze z uprawnieniami roota jest poważną luką bezpieczeństwa.
\item \textbf{Niewystarczające buforowanie warstw:} Nieoptymalna kolejność instrukcji w Dockerfile może prowadzić do częstego przebudowywania warstw i wydłużenia czasu budowania obrazów.
\end{itemize}

\textbf{Kubernetes (K8s)} jest systemem do automatyzacji wdrażania, skalowania i zarządzania aplikacjami kontenerowymi. Dzięki deklaratywnemu podejściu, użytkownicy definiują pożądany stan systemu (poprzez manifesty YAML), a Kubernetes dąży do jego utrzymania. W niniejszym badaniu Kubernetes służy przede wszystkim jako środowisko do \textbf{walidacji poprzez uruchomienie} wygenerowanych konfiguracji IaC. Pozwala to na weryfikację, czy manifesty Kubernetes, wygenerowane przez LLM, faktycznie prowadzą do uruchomienia działających aplikacji i czy zachowują się zgodnie z oczekiwaniami, a nie tylko są syntaktycznie poprawne.

Podobnie jak w przypadku Dockerfile, generowanie manifestów Kubernetes przez LLM może prowadzić do szeregu błędów i nieoptymalnych konfiguracji, które mogą wpłynąć na dostępność, bezpieczeństwo lub wydajność aplikacji:
\begin{itemize}
\item \textbf{Brak liveness i readiness probes:} Te sondy są kluczowe dla monitorowania stanu kontenerów i zapewnienia wysokiej dostępności. Ich brak może prowadzić do wdrażania niedziałających aplikacji lub błędnego zarządzania cyklem życia poda.
\item \textbf{Niedopasowane lub błędne zasoby (CPU/Memory limits and requests):} Niewłaściwe ustawienie limitów lub żądań zasobów może prowadzić do problemów z wydajnością aplikacji (throttling), niestabilności klastra (OOMKilled) lub nieefektywnego wykorzystania infrastruktury.
\item \textbf{Błędne konfiguracje serwisów:} Niepoprawne typy serwisów (np. użycie ClusterIP zamiast NodePort lub LoadBalancer gdy wymagany jest dostęp zewnętrzny) lub brak poprawnej konfiguracji portów może uniemożliwić dostęp do aplikacji.
\item \textbf{Brak PersistentVolumeClaim (PVC) dla aplikacji stanowych:} Aplikacje wymagające trwałego przechowywania danych (np. bazy danych) bez zdefiniowanego PVC utracą dane po restarcie poda.
\item \textbf{Niepoprawne użycie ConfigMap i Secret:} Twarde zakodowanie wrażliwych danych bezpośrednio w manifeście zamiast użycia Secret lub nieprawidłowe zamontowanie ConfigMap/Secret.
\item \textbf{Brak kontekstu bezpieczeństwa (Security Context):} Brak definicji securityContext dla poda lub kontenera, co może prowadzić do uruchamiania aplikacji z niepotrzebnie wysokimi uprawnieniami (np. jako root, z otwartymi portami lub zbyt szerokim dostępem do systemu plików).
\item \textbf{Niepoprawne etykiety (labels) i selektory (selectors):} Błędy w etykietach mogą uniemożliwić serwisom odnalezienie odpowiednich podów, co prowadzi do niedostępności aplikacji.
\end{itemize}

Analiza zdolności LLM do unikania wymienionych błędów oraz do generowania efektywnych, bezpiecznych i zoptymalizowanych konfiguracji Dockerfile i manifestów Kubernetes stanowi jeden z głównych obszarów oceny w niniejszej pracy.



\subsection{Narzędzia do oceny jakości IaC}

W kontekście automatycznego generowania konfiguracji Infrastructure as Code (IaC) przez modele językowe, ważne jest zapewnienie, że wygenerowane konfiguracje są nie tylko poprawne składniowo, ale również bezpieczne, zgodne z najlepszymi praktykami i efektywne. Do tego celu służą wyspecjalizowane narzędzia do analizy i walidacji kodu IaC. Wyniki uzyskane z tych narzędzi będą stanowić podstawę do obiektywnej oceny jakości konfiguracji generowanych przez LLM w dalszej części niniejszej pracy. W niniejszym podrozdziale opisane zostaną dwa wybrane narzędzia, które zostaną użyte w badaniu: Hadolint \cite{hadolint} (dla Dockerfile) i Kube-linter \cite{kubelinter} (dla manifestów Kubernetes).

Oba narzędzia działają na zasadzie statycznej analizy kodu, co oznacza, że analizują pliki konfiguracyjne bez konieczności ich uruchamiania lub wdrażania.

Hadolint to narzędzie do lintingu i statycznej analizy plików Dockerfile. Sprawdza zgodność z najlepszymi praktykami, wykrywa potencjalne błędy i luki bezpieczeństwa. Analizuje pliki Dockerfile, stosując zestaw wbudowanych reguł (bazujących na najlepszych praktykach Docker) oraz konfigurowalnych reguł. Wykrywa takie problemy jak błędy składniowe w Dockerfile, użycie przestarzałych lub niebezpiecznych instrukcji, brak optymalizacji warstw, czy potencjalne luki bezpieczeństwa (np. uruchamianie procesów jako root).

Kube-linter to narzędzie do analizy manifestów Kubernetes, wykrywające błędy konfiguracyjne, problemy z bezpieczeństwem i niezgodności z najlepszymi praktykami. Analizuje manifesty Kubernetes (pliki YAML), sprawdzając je pod kątem zgodności z najlepszymi praktykami Kubernetes oraz potencjalnych błędów konfiguracyjnych. Wykrywa takie problemy jak brak liveness i readiness probes, nieprawidłowe limity zasobów, problemy z konfiguracją serwisów, czy potencjalne luki bezpieczeństwa (np. brak securityContext).

Poniżej przedstawiono przykładowe fragmenty wyjścia z Hadolint i Kube-linter, ilustrujące typowe problemy wykrywane przez te narzędzia:

\begin{lstlisting}[caption={Przykład wyników Hadolint},label={lst:example-hadolint},captionpos=b,columns=fullflexible, breaklines=true]
Dockerfile:3 DL3003 Use WORKDIR to set the working directory for subsequent instructions.
Dockerfile:5 DL3018 Pin versions in apk add. Instead of `apk add <package>` use
`apk add <package>=<version>`.
Dockerfile:12 DL3005 Do not use apt-get upgrade
\end{lstlisting}

W powyższym przykładzie, przedstawionym w wycinku \ref{lst:example-hadolint}, Hadolint zgłasza brak instrukcji WORKDIR, brak przypiętych wersji pakietów w apk add oraz użycie apt-get upgrade.

\begin{lstlisting}[caption={Przykład wyników Kube-linter},label={lst:example-kubelinter},captionpos=b,columns=fullflexible, breaklines=true]
[ERROR] Pod: hello-pod, object name hello-pod: container "hello" has no livenessProbe.
[ERROR] Pod: hello-pod, object name hello-pod: container "hello" has no readinessProbe.
[WARNING] Service: hello-service, object name hello-service: service "hello-service" should
set spec.type to "ClusterIP"
\end{lstlisting}

W tym przykładzie, zaprezentowanym w wycinku \ref{lst:example-kubelinter}, Kube-linter zgłasza brak livenessProbe i readinessProbe w definicji Poda oraz sugeruje zmianę typu Serwisu na ClusterIP.

Interpretacja wyników z obu narzędzi wymaga zrozumienia specyfiki Dockerfile i manifestów Kubernetes. Ważne jest, aby skupić się na błędach o wysokim poziomie ważności (np. problemy z bezpieczeństwem, brak sond) oraz na ostrzeżeniach, które mogą prowadzić do problemów z wydajnością lub stabilnością aplikacji. Wyjścia z tych narzędzi zostaną wykorzystane w dalszej części pracy do obiektywnej oceny jakości generowanych konfiguracji IaC.

\subsection{Środowisko eksperymentalne i technologie wspierające}

Eksperymenty zostały przeprowadzone na lokalnej stacji roboczej, którą stanowi komputer MacBook Air z procesorem Apple M4 i 16 GB pamięci RAM. Ważne jest, że w środowisku tym nie wykorzystywano lokalnie żadnych kart graficznych (GPU) do inferencji modeli językowych. Wszystkie interakcje z modelami LLM odbywały się wyłącznie za pośrednictwem zewnętrznych interfejsów API.

Do zarządzania kontenerami i środowiskami uruchomieniowymi wykorzystano następujące narzędzia:
\begin{itemize}
	\item \textbf{Docker Desktop:} \cite{docker_desktop} Używany do uruchamiania kontenerów oraz do budowania obrazów Docker na podstawie generowanych plików Dockerfile. Docker Desktop służył jako podstawowe środowisko konteneryzacji.
	\item \textbf{Kind (Kubernetes in Docker):} \cite{kind} Lekki klaster Kubernetes uruchamiany w kontenerach Docker. Kind posłużył jako środowisko do walidacji wygenerowanych manifestów Kubernetes w fazie eksperymentalnej. Pozwoliło to na weryfikację, czy aplikacje faktycznie uruchamiają się i działają poprawnie w środowisku Kubernetes.
	\item \textbf{MicroK8s:} \cite{microk8s} Lekka dystrybucja Kubernetes przeznaczona do uruchamiania w środowiskach produkcyjnych oraz testowych. MicroK8s został wykorzystany jako bardziej zbliżony do środowiska produkcyjnego klaster Kubernetes, umożliwiający walidację wygenerowanych konfiguracji w warunkach bardziej odpowiadających rzeczywistym wdrożeniom.
\end{itemize}

Cała logika sterowania eksperymentami, w tym implementacja agentów, wywoływanie narzędzi, walidacja i zbieranie wyników, została zrealizowana przy użyciu języka \textbf{Python w wersji 3.13}. Wykorzystane biblioteki Python obejmowały:
\begin{itemize}
	\item \textbf{LangChain i LangGraph:} \cite{langchain} \cite{langgraph} Te frameworki stanowiły podstawę do budowy architektury agentowej, zarządzania przepływami konwersacji z LLM oraz integrowania wywołań narzędzi.
	\item \textbf{Docker SDK for Python (docker):} \cite{docker_sdk_python} Umożliwiała programistyczną interakcję z Docker Daemon, w tym budowanie obrazów i zarządzanie kontenerami.
	\item \textbf{Kubernetes Python Client (kubernetes):} \cite{kubernetes_sdk_python} Służyła do programistycznej interakcji z API Kubernetes, umożliwiając wdrażanie manifestów i sprawdzanie stanu zasobów.
	\item \textbf{Pydantic:} \cite{pydantic} Wykorzystywana do walidacji danych i definiowania schematów wejścia/wyjścia dla funkcji i narzędzi używanych przez agentów, co zwiększyło niezawodność i bezpieczeństwo interakcji.
	\item \textbf{Streamlit:} \cite{streamlit} Framework do szybkiego tworzenia aplikacji webowych w języku Python, wykorzystany do implementacji panelu sterowania eksperymentami. Umożliwił stworzenie interaktywnego interfejsu do zarządzania eksperymentami, monitorowania ich postępu oraz przeglądania wyników.
	\item Inne standardowe biblioteki Python do obsługi danych, operacji na plikach oraz generowania raportów.
\end{itemize}

Interakcje z modelami LLM odbywały się za pośrednictwem dwóch głównych ścieżek:
\begin{itemize}
	\item \textbf{Bezpośrednie API:} W przypadku modeli komercyjnych, które oferują własne dedykowane API (np. OpenAI GPT-4, Google Gemini Pro, Anthropic Claude), korzystano z ich natywnych interfejsów programistycznych.
	\item \textbf{OpenRouter:} Dla modeli open-source (np. Mistral, Meta Llama, DeepSeek), które często nie posiadają własnych komercyjnych API lub ich infrastruktura do hostingu jest zbyt kosztowna dla pojedynczego badacza, wykorzystano platformę \textbf{OpenRouter}. OpenRouter działa jako brama API (API Gateway), agregująca dostęp do wielu modeli od różnych dostawców za pośrednictwem ujednoliconego interfejsu API, co znacząco ułatwiało testowanie różnorodnych modeli bez konieczności integracji z wieloma odrębnymi API.
\end{itemize}

W trakcie prowadzenia badań i eksperymentów wykorzystano również szereg narzędzi wspomagających, które były cenne dla efektywnego zarządzania procesem badawczym i analizą wyników:
\begin{itemize}
	\item \textbf{LangSmith:} \cite{langsmith} Nrzędzie do monitorowania, debugowania i ewaluacji łańcuchów agentów zbudowanych w LangChain/LangGraph. LangSmith pozwalał na wizualizację poszczególnych kroków rozumowania agenta, jego interakcji z narzędziami oraz wywołań API do LLM. Było to kluczowe do analizy, dlaczego agent podjął taką, a nie inną decyzję, i gdzie popełnił błąd, co znacząco przyspieszyło proces optymalizacji promptów i logiki agenta.
	\item \textbf{Lens IDE:} \cite{lens_ide} Graficzne narzędzie (IDE) do zarządzania klastrami Kubernetes, ułatwiające monitorowanie stanu podów, serwisów i innych zasobów w trakcie walidacji.
\end{itemize}
Wspomniane narzędzia pomocnicze były integralną częścią środowiska badawczego, umożliwiając nie tylko uruchamianie eksperymentów, ale przede wszystkim ich efektywne monitorowanie, analizę i optymalizację.

\clearpage % Rozdziały zaczynamy od nowej strony.
\section{Projekt eksperymentów}

Cel: Szczegółowe przedstawienie projektu eksperymentów mających na celu ocenę zdolności dużych modeli językowych (LLM) do automatycznego generowania konfiguracji Infrastructure as Code (IaC) dla środowisk Docker i Kubernetes.

Proponowana zawartość:

\textbf{Cel i hipotezy badawcze:}
Sformułowanie głównego celu eksperymentów.
Wypunktowanie konkretnych hipotez badawczych do weryfikacji.
\textbf{Zestawienie przypadków testowych:}
Opis 4 spreparowanych repozytoriów, reprezentujących różne scenariusze wdrożeniowe:
Aplikacja bezstanowa bez zależności.
Aplikacja wykorzystująca stanową bazę danych postawioną obok.
Frontend + Backend + Baza Danych.
Prosty system mikroserwisowy.
Zostawienie miejsca na wspomnienie o ewentualnym wykorzystaniu dodatkowych repozytoriów z serwisu GitHub w celu przetestowania działania na "niespreparowanych" aplikacjach.
\textbf{Metodyka generacji i promptowania:}
Szczegółowy opis, że metodą promptowania będzie agent napisany w LangGraph.
Miejsce na wykorzystywany prompt (listing lub szczegółowy opis struktury).
Miejsce na opis narzędzi (funkcji), które będą dostępne dla agenta i które będzie mógł wykorzystać (np. do czytania plików, walidacji składni, itp.).
\textbf{Proces testowy:}
Szczegółowy schemat działania agenta i następującego po nim procesu oceny: \begin{enumerate} \item Przygotowanie środowiska roboczego (working directory). \item Sklonowanie repozytorium aplikacji. \item Generowanie Dockerfile (lub wielu, w zależności od architektury) przez agenta LLM. \item Generowanie manifestów Kubernetes przez agenta LLM. \end{enumerate}
W tym momencie kończy się rola generowania przez agenta, a zaczyna rola automatycznej oceny modelu (jakość generacji): \begin{enumerate}[resume] \item Weryfikacja syntaktycznej poprawności wygenerowanego Dockerfile. \item Statyczna analiza Dockerfile za pomocą Hadolint (sprawdzenie "prucia się" Hadolint). \item Próba budowy obrazu Docker na podstawie wygenerowanego Dockerfile. \item Weryfikacja syntaktycznej poprawności wygenerowanych manifestów Kubernetes. \item Statyczna analiza manifestów Kubernetes za pomocą Kube-linter (sprawdzenie "prucia się" Kube-linter). \item Próba zaaplikowania manifestów Kubernetes w środowisku Kind. \item Walidacja działania aplikacji w środowisku Kubernetes (walidacja runtime). \item Ocena jakości konfiguracji pod kątem dobrych praktyk (np. poprzez interpretację wyników statycznej analizy). \end{enumerate}
\textbf{Kryteria oceny i metryki:}
Definicja kryteriów jakości IaC (poprawność składniowa/funkcjonalna, bezpieczeństwo, kompletność, deterministyczność, odporność na manipulacje, wydajność generacji).
Wskazanie, jakie metryki będą używane do pomiaru każdego kryterium.
\textbf{Przykładowe obserwacje i napotkane problemy:}
Omówienie ogólnego zachowania modeli LLM podczas generacji.
Analiza typowych problemów napotkanych podczas eksperymentów (np. ograniczenia tokenów, błędy w kontekście, wpływ długości promptów, deterministyczność).
Wyniki szczegółowe zostaną zaprezentowane w kolejnym rozdziale.

TODO napisac ze wszystkie modele dostaja ten sam prompt i sa testowane na tych samych repozytoriach, a wyniki sa porownywane miedzy nimi.

TODO co bedzie mierzone i jak? (np langsmith), np. liczba tokenow wejsciowych, wyjsciowych, liczba przeczytanych plikow, liczba wylistowan plikow, liczbe bledow statycznej analizy itd

TODO
% \clearpage % Rozdziały zaczynamy od nowej strony.
\section{Analiza porównawcza}

TODO
% \input{tex/6-bezpieczenstwo}
% \input{tex/7-projekt-systemu}
% \input{tex/8-implementacja}
% \input{tex/9-wnioski-i-dalsze-kierunki}

%---------------
% Bibliografia
%---------------
\cleardoublepage % Zaczynamy od nieparzystej strony
\printbibliography
\clearpage

% Wykaz symboli i skrótów.
% Pamiętaj, żeby posortować symbole alfabetycznie
% we własnym zakresie. Makro \acronymlist
% generuje właściwy tytuł sekcji, w zależności od języka.
% Makro \acronym dodaje skrót/symbol do listy,
% zapewniając podstawowe formatowanie.

\acronymlist
\acronym{LLM}{ang. \emph{Large Language Model}} - duży model językowy
\acronym{IaC}{ang. \emph{Infrastructure as Code} – infrastruktura jako kod}
\acronym{K8s}{ang. \emph{Kubernetes} – system orkiestracji kontenerów}
\acronym{PaaS}{ang. \emph{Platform as a Service} – platforma jako usługa}
\acronym{CI/CD}{ang. \emph{Continuous Integration / Continuous Delivery} – ciągła integracja i dostarczanie}
\acronym{YAML}{ang. \emph{YAML Ain't Markup Language} – czytelny dla człowieka format danych tekstowych}
\acronym{API}{ang. \emph{Application Programming Interface} – interfejs programistyczny}
\acronym{KCF}{ang. \emph{Kubernetes Configuration Fault} – błąd konfiguracyjny w manifestach Kubernetes}
\acronym{CWE}{ang. \emph{Common Weakness Enumeration} – klasyfikacja podatności w oprogramowaniu}
\acronym{GPT}{ang. \emph{Generative Pre-trained Transformer} – architektura modelu językowego wykorzystywana w LLM}
\vspace{0.8cm}

%--------------------------------------
% Spisy: rysunków, tabel, załączników
%--------------------------------------
\pagestyle{plain}

\listoffigurestoc    % Spis rysunków.
\vspace{1cm}         % vertical space
\listoftablestoc     % Spis tabel.
\vspace{1cm}         % vertical space
\listofappendicestoc % Spis załączników
\vspace{1cm}         % vertical space
\listoflistingstoc   % Spis listingów

%-------------
% Załączniki
%-------------

% Obrazki i tabele w załącznikach nie trafiają do spisów

% Używając powyższych spisów jako szablonu,
% możesz dodać również swój własny wykaz,
% np. spis algorytmów.

% Załącznik A
\clearpage
\appendix{Prompt wykorzystywany w eksperymencie}
\label{att:prompt}

Poniżej przedstawiono pełną treść prompta używanego przez agenta LangGraph podczas generacji konfiguracji Infrastructure as Code (IaC).

\begin{minted}[breaklines=true, fontsize=\small, linenos, frame=lines]{text}
You are a helpful assistant specialized in working with Git repositories.
You have access to tools that can help you with these tasks. When given a repository URL, you can:
1. Clone the repository and remove confusing files
2. Analyze the repository structure to identify important files
3. Retrieve the content of files you determine are necessary to understand the application
4. Write or modify files in the repository (e.g., Dockerfile, Kubernetes manifests)
5. List directory contents within the repository

You should use the clone_repo tool to clone a repository. The repository name can be extracted from the repository URL by taking the last part of the URL, removing the .git extension, and replacing dots with hyphens.
For example, for the URL \"https://github.com/run-rasztabiga-me/poc1-fastapi.git\", the repository name would be \"poc1-fastapi\".

You can use the prepare_repo_tree tool to get an overview of the repository structure if needed, but you should focus on identifying and examining files that are most relevant to understanding the application and creating the required outputs.

Use the get_file_content tool to retrieve the content of specific files that you determine are important. This tool requires the file path relative to the repository root.

You can use the write_file tool to create new files or modify existing ones in the repository. This tool requires the file path relative to the repository root and the content to write to the file. This is particularly useful for creating files like Dockerfile or Kubernetes manifests.

You can use the ls tool to list the contents of a directory within the cloned repository. This tool requires the directory path relative to the repository root. You can use an empty string or "." to list the contents of the repository root directory. The tool will display directories and files separately, with directories having a trailing slash and files showing their sizes in bytes. This is useful for exploring the repository structure in a more focused way than the prepare_repo_tree tool.


Your objective is to create:

1. A Dockerfile that properly containerizes the application. When creating the Dockerfile, carefully analyze the application code to ensure that any health check endpoint you specify actually exists in the application.

2. Kubernetes manifests for the application. These manifests should:
   - Include all required resources (Deployments, Services, Ingresses, and Volumes if necessary)
   - Match exposed ports precisely as specified in the Dockerfile
   - Set replicas default to 1 unless otherwise stated
   - For ingress host, use "<repository-name>.rasztabiga.me" (e.g., repository "app1" -> domain "app1.rasztabiga.me")
   - Follow Kubernetes best practices and ensure security measures
   - If external dependencies (e.g., databases like PostgreSQL, Redis, MySQL) are identified, generate appropriate Kubernetes resources for those dependencies as well
   - Deploy stateful dependencies using StatefulSets with appropriate PersistentVolumeClaims
   - Deploy stateless applications using Deployments
   - Use Services to expose applications internally and externally as necessary
   - Ensure all Kubernetes secrets are in base64 format
   - Include appropriate resource limits/requests
   - When configuring health checks (liveness and readiness probes), verify that the specified endpoints actually exist in the application code first
   - DO NOT create or include a namespace in the manifests

Given a repository URL from the user, you should automatically:
1. Clone the repository
2. Analyze the repository structure and find important files to understand the application
3. Create a Dockerfile for the application
4. Generate appropriate Kubernetes manifests for the application 

The user will only provide the repository URL. You must handle all the remaining steps automatically without requesting additional information from the user.

IMPORTANT: Your task is only to analyze the repository and generate the required files (Dockerfile and Kubernetes manifests). You should NOT build Docker images, run containers, or apply Kubernetes manifests. After you have successfully generated a Dockerfile and Kubernetes manifests, respond with a message that includes the word "DONE" to indicate that you have completed the task.

\end{minted}

\end{document} % Dobranoc.
