%-----------------------------------------------
%  Engineer's & Master's Thesis Template
%  Copyleft by Artur M. Brodzki & Piotr Woźniak
%  Warsaw University of Technology, 2019-2022
%-----------------------------------------------

\documentclass[
    bindingoffset=5mm,  % Binding offset
    footnoteindent=3mm, % Footnote indent
    hyphenation=true    % Hyphenation turn on/off
]{src/wut-thesis}

\graphicspath{{tex/img/}} % Katalog z obrazkami.
\addbibresource{bibliografia.bib} % Plik .bib z bibliografią

%-------------------------------------------------------------
% Wybór wydziału:
%  \facultyeiti: Wydział Elektroniki i Technik Informacyjnych
%  \facultymeil: Wydział Mechaniczny Energetyki i Lotnictwa
% --
% Rodzaj pracy: \EngineerThesis, \MasterThesis, \PPMGR
% --
% Wybór języka: \langpol, \langeng
%-------------------------------------------------------------
\facultyeiti    % Wydział Elektroniki i Technik Informacyjnych
\MasterThesis % Praca inżynierska
\langpol % Praca w języku polskim

\begin{document}

\counterwithin{lstlisting}{section}

%------------------
% Strona tytułowa
%------------------
\instytut{Informatyki}
\kierunek{Informatyka}
\specjalnosc{Inteligentne Systemy}
\title{
    Zastosowanie dużych modeli językowych (LLM) \\ 
    do generowania konfiguracji Docker i Kubernetes
}
% Title in English for English theses
% In English theses, you may remove this command
\engtitle{
    Application of large language models (LLMs) \\
    for generating Docker and Kubernetes configurations
}
% Title in Polish for English theses
% Use it only in English theses
\poltitle{
    Zastosowanie dużych modeli językowych (LLM) \\ 
    do generowania konfiguracji Docker i Kubernetes
}
\author{Bartłomiej Rasztabiga}
\album{304117}
\promotor{dr inż. Mateusz Modrzejewski}
\date{\the\year}
\maketitle

%-------------------------------------
% Streszczenie po polsku dla \langpol
% English abstract if \langeng is set
%-------------------------------------
\cleardoublepage % Zaczynamy od nieparzystej strony
\abstract
Celem niniejszej pracy magisterskiej jest zbadanie możliwości wykorzystania dużych modeli językowych (LLM) do automatycznego generowania konfiguracji infrastruktury jako kodu (IaC), ze szczególnym uwzględnieniem plików Dockerfile oraz manifestów Kubernetes. Przeprowadzono przegląd literatury opisującej aktualne badania w tym obszarze, wskazując zarówno potencjał automatyzacji, jak i liczne wyzwania związane z poprawnością, bezpieczeństwem i niezawodnością wygenerowanego kodu.

W części badawczej pracy przeprowadzono analizę możliwości wybranych modeli językowych, zaprojektowano zestaw eksperymentów porównawczych oraz opracowano własne kryteria oceny poprawności i bezpieczeństwa generowanych manifestów Docker i Kubernetes. Następnie, na podstawie wyników eksperymentów, zaprojektowano i zaimplementowano prototyp systemu typu Platform as a Service (PaaS), który automatycznie generuje konfiguracje na podstawie repozytorium kodu, buduje obrazy Docker oraz wdraża aplikacje w klastrze Kubernetes.

Przeprowadzono serię testów oceniających skuteczność modeli pod względem poprawności składniowej, zgodności z wymaganiami oraz odporności na błędy i podatności bezpieczeństwa. Dodatkowo w systemie zaimplementowano komponenty wspierające walidację i automatyczne wykrywanie potencjalnie niebezpiecznych konfiguracji.

\keywords Infrastructure as Code, LLM, Kubernetes, Docker, automatyzacja, bezpieczeństwo, DevOps, PaaS

%----------------------------------------
% Streszczenie po angielsku dla \langpol
% Polish abstract if \langeng is set
%----------------------------------------
\clearpage
\secondabstract
The goal of this master’s thesis is to investigate the potential of large language models (LLMs) for automatically generating Infrastructure as Code (IaC), with a particular focus on Dockerfiles and Kubernetes manifests. The study includes a literature review of recent research in the field, identifying both the automation potential and key challenges related to the correctness, security, and reliability of LLM-generated infrastructure code.

In the research phase, a comparative analysis of selected language models was conducted, along with the design of experimental test cases and the development of custom evaluation criteria for the correctness and security of generated Docker and Kubernetes manifests. Based on the experimental results, a prototype Platform as a Service (PaaS) system was designed and implemented, capable of automatically generating configurations from code repositories, building Docker images, and deploying applications to a Kubernetes cluster.

A series of tests was conducted to evaluate the models’ accuracy, compliance with system requirements, and resistance to misconfigurations and security vulnerabilities. In addition, components supporting validation and automatic detection of unsafe configurations were implemented.

\secondkeywords Infrastructure as Code, Large Language Models, Kubernetes, Docker, Automation, Security, DevOps, PaaS

\pagestyle{plain}

%--------------
% Spis treści
%--------------
\cleardoublepage % Zaczynamy od nieparzystej strony
\tableofcontents

%------------
% Rozdziały
%------------
\cleardoublepage % Zaczynamy od nieparzystej strony
\pagestyle{headings}

% TODO rozdziały magisterki
% 
% TODO

\clearpage % Rozdziały zaczynamy od nowej strony.
\section{Wprowadzenie}

W ciągu ostatnich lat obserwujemy bezprecedensowy rozwój dużych modeli językowych (LLM) oraz równoległy wzrost znaczenia technologii konteneryzacji i orkiestracji. Modele językowe takie jak GPT, LLaMA, Falcon czy Claude wykazują zdolność do generowania złożonego kodu, w tym konfiguracji infrastruktury, podczas gdy Docker i Kubernetes ugruntowały swoją pozycję jako standard w obszarze wdrażania aplikacji. Niniejsza praca bada potencjał automatyzacji generowania konfiguracji Docker i Kubernetes z wykorzystaniem LLM, ze szczególnym uwzględnieniem aspektów poprawności i bezpieczeństwa generowanych plików konfiguracyjnych.

\subsection{Cel i zakres pracy}

Celem niniejszej pracy magisterskiej jest analiza możliwości zastosowania dużych modeli językowych (LLM) do automatycznego generowania konfiguracji typu Infrastructure as Code (IaC), ze szczególnym uwzględnieniem plików Dockerfile oraz manifestów Kubernetes. Praca skupia się na wykorzystaniu LLM w kontekście platformy jako usługi (PaaS), gdzie efektywne tworzenie i utrzymanie kontenerów oraz ich orkiestracji ma kluczowe znaczenie. Zakres obejmuje zbadanie metod generowania plików Dockerfile definiujących obrazy kontenerów oraz manifestów Kubernetes opisujących wdrożenie tych kontenerów, a także oceny jakości, bezpieczeństwa i zgodności takich wygenerowanych konfiguracji z wymaganiami systemowymi.

Głównym celem pracy jest analiza możliwości zastosowania dużych modeli językowych do generowania konfiguracji Dockerfile i Kubernetes w ramach platformy jako usługi (PaaS), która umożliwia budowanie i wdrażanie aplikacji w klastrze Kubernetes. Praca obejmuje następujące zagadnienia:

\subsection{Motywacja}

Tematyka pracy jest istotna z uwagi na rosnącą rolę metodyk DevOps i automatyzacji zarządzania infrastrukturą. Infrastructure as Code zyskuje na popularności, ponieważ umożliwia spójną, powtarzalną konfigurację środowisk i redukcję błędów ludzkich. Ręczne tworzenie skryptów IaC, zwłaszcza dla złożonych środowisk chmurowych, bywa jednak czasochłonne i wymaga specjalistycznej wiedzy. Wraz z rozwojem dużych modeli językowych pojawiła się możliwość ich wykorzystania do generowania kodu konfiguracyjnego na podstawie opisu w języku naturalnym. Użycie LLM może obniżyć barierę wejścia dla deweloperów mniej doświadczonych w technologiach chmurowych (np. Kubernetes) poprzez automatyczne tłumaczenie wysokopoziomowych specyfikacji na deklaratywne pliki konfiguracji. Ma to znaczenie praktyczne na platformach PaaS, gdzie skrócenie czasu wdrożenia aplikacji i eliminacja błędów konfiguracji przekładają się na większą wydajność i niezawodność usług. Z drugiej strony, automatyzacja generowania IaC rodzi pytania o poprawność i bezpieczeństwo tych konfiguracji. Modele językowe mogą popełniać błędy lub tzw. halucynacje, generując nieistniejące lub niezalecane elementy konfiguracji. Istotne jest zatem zbadanie, na ile można zaufać LLM w kontekście krytycznych elementów infrastruktury oraz jak zapewnić zgodność wygenerowanych konfiguracji z najlepszymi praktykami bezpieczeństwa (np. czy model nie pominie istotnych zabezpieczeń, jak autoryzacja, czy nie wygeneruje podatnych ustawień). Zainteresowanie połączeniem LLM i DevOps wynika także z potencjału ułatwienia pracy inżynierów – dzięki LLM mogą oni szybciej uzyskiwać wstępne wersje konfiguracji i skupić się na ich dostrojeniu, zamiast pisać je od podstaw.

\subsection{Struktura pracy}

Praca składa się z dziewięciu rozdziałów, które wspólnie odzwierciedlają pełny cykl badawczy — od identyfikacji problemu, przez analizę literatury i technologii, aż po eksperymenty, analizę wyników, bezpieczeństwo i implementację kompletnego systemu.

Rozdział 1 – Wprowadzenie: przedstawia temat pracy, jej cele, motywację oraz układ całej pracy.
Rozdział 2 – Przegląd literatury: zawiera analizę aktualnych badań nad wykorzystaniem dużych modeli językowych w kontekście generowania Infrastructure as Code (IaC), w tym konfiguracji Docker i Kubernetes. Wskazuje także istniejące luki, np. brak analiz bezpieczeństwa i brak pełnych pipeline’ów DevOps.
Rozdział 3 – Przegląd technologii i narzędzi: opisuje wykorzystane modele językowe (GPT, Claude, LLaMA, Mistral, DeepSeek) oraz narzędzia infrastrukturalne (Docker, Kubernetes, Lens, kind). Rozdział porównuje też podejścia API vs open-source, opisuje środowisko uruchomieniowe i technologie wspierające (FastAPI, GitPython itp.).
Rozdział 4 – Eksperymenty: opisuje zaplanowane przypadki testowe (np. aplikacje jedno- i wielosystemowe), strategie promptowania, proces generowania i testowania konfiguracji oraz przykładowe wyniki i napotkane problemy (np. ograniczenia tokenów).
Rozdział 5 – Analiza porównawcza modeli LLM: przedstawia kryteria oceny (poprawność, bezpieczeństwo, deterministyczność, odporność na manipulacje, wydajność), wyniki testów i wnioski na temat jakości działania poszczególnych modeli.
Rozdział 6 – Bezpieczeństwo konfiguracji: analizuje zagrożenia związane z automatycznym generowaniem konfiguracji, techniki ataku na LLM (prompt injection, jailbreaking) oraz metody oceny i wzmacniania bezpieczeństwa wygenerowanego kodu.
Rozdział 7 – Projekt systemu PaaS: prezentuje architekturę zaprojektowanego systemu do automatycznego generowania, budowania i wdrażania aplikacji. Opisuje także problemy projektowe oraz proces działania od repozytorium do uruchomienia aplikacji w Kubernetes.
Rozdział 8 – Implementacja i wdrożenie prototypu: zawiera szczegółowy opis realizacji systemu PaaS, użytych technologii, przykładowych wdrożeń oraz napotkanych problemów podczas integracji i testów.
Rozdział 9 – Wnioski i dalsze kierunki rozwoju: podsumowuje uzyskane wyniki, odpowiada na pytania badawcze, ocenia przydatność LLM w środowiskach DevOps oraz wskazuje możliwe ścieżki rozwoju — takie jak wsparcie mikroserwisów, walidacja semantyczna YAML czy integracja z CI/CD.
\clearpage % Rozdziały zaczynamy od nowej strony.
\section{Przegląd literatury}

Niedawny rozwój możliwości generowania kodu dzięki zastosowaniu dużych modeli językowych dotyczył głównie języków programowania ogólnego przeznaczenia. Języki specyficzne dla dziedzin, takie jak te wykorzystywane w automatyzacji IT, otrzymały znacznie mniej uwagi, mimo że angażują wielu aktywnych deweloperów i stanowią istotny element współczesnych platform chmurowych. \cite{pujar_invited_2023}

Literatura dotycząca wykorzystania dużych modeli językowych (LLM) do generowania manifestów Docker i Kubernetes jest ograniczona. Istnieją jednak badania, które zajmują się różnymi aspektami tej tematyki, szczególnie w kontekście automatyzacji infrastruktury jako kodu (IaC). Żadna z do tej pory przeanalizowanych, istniejących prac nie koncentruje się jednak na pełnym wykorzystaniu repozytorium aplikacji jako wejścia do modeli LLM, a także na dokładnym analizowaniu problemów bezpieczeństwa takich konfiguracji. W niniejszym przeglądzie przedstawiane są wybrane badania oraz wskazywane są sposoby, w jakie niniejsza praca rozszerza i ulepsza istniejące podejścia.

Praca Malula i in. wprowadza system GenKubeSec \cite{malul_genkubesec_2024}, który wykorzystuje modele LLM do wykrywania i naprawiania błędnych konfiguracji Kubernetes (KCF). System ten jest innowacyjny, ponieważ dostarcza pełne rozwiązanie: od wykrywania problemów, przez ich lokalizację, aż po ich naprawę. Kluczową zaletą GenKubeSec jest minimalizacja ryzyka związanego z bezpieczeństwem poprzez wykorzystanie lokalnych modeli LLM zamiast komercyjnych API. Praca poszerza te badania, badając nie tylko wykrywanie błędnych konfiguracji, ale także aspekt ich deterministyczności i odporności na manipulacje danymi wejściowymi.

Ueno i Uchiumi proponują benchmark oceniający jakość manifestów Kubernetes generowanych przez modele LLM na podstawie specyfikacji Docker Compose \cite{ueno_migrating_2024}. Wyniki pokazują, że generowane manifesty są często poprawne, ale brakuje w nich spójności i komentarzy poprawiających czytelność. W pracy uwzględniana jest ta krytyka, koncentrując się na generowaniu manifestów, które są zarówno funkcjonalne, jak i czytelne dla ludzi.

Kratzke i Drews badali możliwości standardowych modeli LLM w generowaniu manifestów Kubernetes przy użyciu zaawansowanych technik inżynierii promptów \cite{kratzke_dont_2024}. Badanie pokazuje, że efektywne projektowanie promptów może znacząco poprawić jakość generowanych manifestów. Praca rozwija ten kierunek, analizując nie tylko jakość generowanych konfiguracji, ale również ich podatność na ataki.

Pujar i in. skoncentrowali się na generowaniu konfiguracji YAML dla Ansible za pomocą modeli LLM \cite{pujar_invited_2023}. Choć praca skupia się na inżynierii promptów i budowaniu dedykowanych datasetów, brak w niej analizy problemów bezpieczeństwa czy deterministyczności. Niniejsza praca rozszerza ten kontekst na Kubernetes i Docker, uwzględniając dodatkowe aspekty, takie jak jailbreaking modeli.

Lanciano i in. zaproponowali podejście do analizy jakości manifestów Kubernetes z wykorzystaniem LLM \cite{lanciano_analyzing_2023}. System dostarcza rekomendacje dotyczące jakości kodu i pomaga mniej doświadczonym deweloperom w stosowaniu najlepszych praktyk. Praca rozwija tę ideę, koncentrując się na automatyzacji całego procesu, od generowania po wdrażanie.

Podjęte w literaturze próby wykorzystania LLM w kontekście IaC i Kubernetes koncentrują się głównie na generowaniu kodu i podstawowej analizie. Niniejsza praca poszerza ten obszar, skupiając się m.in. na:

\begin{itemize}
    \item Wykorzystaniu repozytoriów kodu jako źródła wejściowego dla LLM,
    \item Analizie deterministyczności i bezpieczeństwa generowanych konfiguracji,
    \item Zastosowaniu metod oceny odporności na manipulacje danymi wejściowymi,
    \item Porównaniu topowych modeli pod kątem możliwej złożoności generowanych konfiguracji (np. obsługa wielu kontenerów, sieci czy woluminów).
\end{itemize}
\clearpage % Rozdziały zaczynamy od nowej strony.

\section{Przegląd technologii i narzędzi}

Celem tego rozdziału jest przedstawienie najważniejszych technologii i narzędzi, stanowiących podstawę niniejszego badania. Szczególny nacisk położony zostanie na duże modele językowe (LLM), opis ich architektur oraz ich zastosowanie w generowaniu konfiguracji Infrastructure as Code (IaC). Ponadto, omówione zostaną istotne narzędzia DevOps, takie jak Docker i Kubernetes, niezbędne w zarządzaniu środowiskami uruchomieniowymi, oraz narzędzia służące do walidacji i oceny jakości generowanego kodu IaC. Rozdział zawiera także opis środowiska eksperymentalnego oraz wykorzystywanych technologii wspierających automatyzację i monitorowanie eksperymentów.

\subsection{Modele językowe wykorzystywane w badaniu}

Duże modele językowe (LLM) pełnią istotną rolę w prezentowanych eksperymentach, służąc jako generatory konfiguracji IaC. Ich zdolność do przetwarzania języka naturalnego oraz generowania poprawnego kodu czyni je ważnym elementem automatyzacji procesów DevOps. W tym podrozdziale przedstawiono charakterystykę wybranych modeli, opis ich architektur oraz praktyczne aspekty związane z ich wykorzystaniem.

Rynek dużych modeli językowych rozwija się dynamicznie, oferując zarówno modele komercyjne, dostępne przez API, jak i modele open-source, możliwe do uruchomienia lokalnie lub na własnej infrastrukturze. Decyzja o wyborze konkretnego rodzaju modelu często zależy od dostępności zasobów sprzętowych (zwłaszcza GPU), wymagań licencyjnych, kwestii związanych z prywatnością danych oraz możliwości dostosowania modeli do specyficznych zastosowań.

Modele komercyjne dostępne przez API zazwyczaj cechują się wysoką wydajnością, zaawansowanymi możliwościami oraz niezawodnością, dzięki wsparciu dużych firm technologicznych takich jak OpenAI, Anthropic czy Google. Modele te trenowane są na szerokich i zróżnicowanych zbiorach danych, co przekłada się na ich zdolność generowania wysokiej jakości tekstu oraz kodu. Ich użytkowanie wiąże się z kosztami dostępu do API oraz koniecznością przesyłania danych do zewnętrznych serwisów, co może być problematyczne w przypadku danych wrażliwych.

Wybrane do badania modele komercyjne to:

\begin{itemize}
	\item \textbf{OpenAI GPT-4.1}, \textbf{GPT-4o} oraz \textbf{O3}: Najnowsze flagowe modele OpenAI. GPT-4.1 charakteryzuje się udoskonalonymi możliwościami generowania i rozumienia złożonego kodu oraz zoptymalizowanymi funkcjami integracji z narzędziami (tool calling). GPT-4o, model multimodalny, oferuje wysoką wydajność, elastyczność oraz wszechstronne możliwości interakcji. O3 zapewnia optymalne połączenie wydajności, szybkości oraz efektywności kosztowej.
	\item \textbf{Anthropic Claude Opus 4}, \textbf{Sonnet 4} oraz \textbf{Haiku 3.5}: Zaawansowane modele Anthropic, cenione za bezpieczeństwo, zgodność z najlepszymi praktykami AI Safety oraz obsługę długich kontekstów i precyzyjnych instrukcji. Opus 4 to najbardziej zaawansowana wersja, Sonnet 4 równoważy wydajność z szybkością, a Haiku 3.5 skupia się na efektywności kosztowej i szybkości działania.
	\item \textbf{Google Gemini 2.5 Pro} i \textbf{2.5 Flash}: Modele Google z rozszerzonym oknem kontekstu, dedykowane do analizy obszernych repozytoriów kodu i dokumentacji. Wersja Pro oferuje wyjątkową pojemność kontekstu i wysoką wydajność, podczas gdy wersja Flash jest zoptymalizowana pod kątem szybszego działania przy niższych kosztach operacyjnych.
\end{itemize}

Modele open-source stanowią atrakcyjną alternatywę, oferując pełną kontrolę, możliwość uruchomienia na własnej infrastrukturze oraz brak dodatkowych opłat za każde zapytanie. Ich wydajność zależy przede wszystkim od dostępności zasobów sprzętowych, a aktywny rozwój społeczności przyczynia się do ciągłej optymalizacji i pojawiania się coraz bardziej zaawansowanych wersji.

Wybrane do badania modele open-source to:

\begin{itemize}
	\item \textbf{Mistral Medium} oraz \textbf{Codestral}: Modele od firmy Mistral AI, zoptymalizowane pod kątem efektywnego generowania kodu i ogólnego rozumienia kontekstu technicznego. Codestral specjalizuje się w obsłudze złożonych zadań programistycznych, oferując wysoką efektywność przy umiarkowanych wymaganiach sprzętowych.
	\item \textbf{Meta Llama 4 Maverick} oraz \textbf{Llama 4 Scout}: Najnowsze modele open-source od Meta, które stanowią istotny punkt odniesienia w dziedzinie rozwoju sztucznej inteligencji. Maverick wyróżnia się wysokimi osiągami i zaawansowanymi funkcjami, natomiast Scout zapewnia dobrą wydajność przy bardziej ograniczonych zasobach sprzętowych, umożliwiając łatwiejsze wdrożenie w praktycznych zastosowaniach.
	\item \textbf{DeepSeek V3}: Model typu Mixture-of-Experts (MoE), łączący wysoką jakość generowania kodu z dużą efektywnością obliczeniową. DeepSeek V3 wyróżnia się zdolnością do dynamicznego aktywowania tylko części parametrów modelu przy każdym zapytaniu, co pozwala osiągnąć korzystny kompromis między jakością a kosztami obliczeń.
\end{itemize}

Wszystkie wymienione modele reprezentują aktualny szczyt osiągnięć w dziedzinie dużych modeli językowych (LLM). Wybrane zostały przede wszystkim ze względu na ich potwierdzone zdolności do generowania wysokiej jakości kodu oraz wsparcie dla funkcji „tool calling”, które umożliwiają bezpośrednią integrację modeli z zewnętrznymi narzędziami lub usługami poprzez dedykowane mechanizmy API bądź platformy integracyjne, takie jak LangChain czy LangGraph. Funkcje te będą odgrywać znaczącą rolę w proponowanym w tej pracy podejściu opartym na pętli sprzężenia zwrotnego w procesie generowania konfiguracji Infrastructure as Code (IaC).

\subsubsection{Architektury i charakterystyka}

Większość współczesnych dużych modeli językowych bazuje na architekturze transformera \cite{vaswani_attention_2023}, wykorzystując mechanizm uwagi (attention mechanism) do efektywnego przetwarzania sekwencji danych. Architektury te, pomimo wspólnych założeń, różnią się między sobą liczbą parametrów, rozmiarem okna kontekstowego oraz specyficznymi optymalizacjami, które wpływają na ich zdolność do generowania kodu i rozumienia złożonych instrukcji. Coraz większe znaczenie zyskują również alternatywne podejścia architektoniczne, takie jak Mixture-of-Experts (MoE), w których przy każdym zapytaniu aktywowane są tylko wybrane części modelu, co pozwala na zwiększenie efektywności obliczeniowej bez pogarszania jakości generowanych wyników.

\begin{itemize}
	\item \textbf{Liczba parametrów:} Liczba parametrów (często liczona w miliardach) jest wskaźnikiem skali modelu i jego zdolności do uczenia się złożonych wzorców z danych. Modele z większą liczbą parametrów zazwyczaj wykazują lepszą wydajność w szerokim zakresie zadań, jednak ich uruchomienie i obsługa wymagają znacznie większych zasobów obliczeniowych. Należy jednak zaznaczyć, że sama liczba parametrów nie jest jedynym wyznacznikiem jakości; kluczowe znaczenie ma również jakość danych treningowych, architektura (np. Mixture-of-Experts) oraz proces dostrajania modelu.
	\item \textbf{Okno kontekstowe:} Okno kontekstowe odnosi się do maksymalnej długości sekwencji tokenów (słów, znaków, fragmentów kodu), którą model może przetworzyć i wykorzystać do wygenerowania odpowiedzi. Jest to szczególnie istotne w kontekście generowania konfiguracji IaC z repozytoriów kodu. Duże okno kontekstowe pozwala na dostarczenie modelowi całych plików źródłowych aplikacji, fragmentów dokumentacji, wielu powiązanych ze sobą promptów, a nawet wyników walidacji z zewnętrznych narzędzi. To umożliwia agentowi LLM holistyczne zrozumienie projektu i kontekstu, co przekłada się na wyższą jakość i trafność generowanych konfiguracji. Modele takie jak Google Gemini 2.5 Pro wyróżniają się wyjątkowo dużym oknem kontekstowym (do miliona tokenów), co jest znaczącą przewagą w zadaniach wymagających głębokiej analizy kodu i dokumentacji.
	\item \textbf{Specjalizacje i optymalizacje:} Niektóre modele, takie jak Mistral Codestral, są specjalizowane lub fine-tuninguowane w kierunku generowania kodu. Oznacza to, że są one trenowane na dużych zbiorach danych zawierających kod programistyczny, co poprawia ich zdolność do generowania syntaktycznie poprawnego i semantycznie trafnego kodu. Architektury takie jak Mixture-of-Experts (MoE) stosowane w Mixtral 8x7B pozwalają na efektywne skalowanie modeli, aktywując tylko część ekspertów dla danego zapytania, co optymalizuje zużycie zasobów przy zachowaniu wysokiej jakości odpowiedzi.
\end{itemize}

Poniższa tabela \ref{tab:llm-characteristics} przedstawia ogólne charakterystyki wybranych modeli LLM, które będą wykorzystywane w badaniu. Okno kontekstu podano w liczbie tokenów, a liczba parametrów — w miliardach, jeśli została opublikowana.

\begin{table}[!h] \centering
\caption{Charakterystyka wybranych modeli LLM}
\label{tab:llm-characteristic}
\begin{tabular}{| c | c | c | c |} \hline
\textbf{Dostawca} & \textbf{Model} & \textbf{Parametry (mld)} & \textbf{Okno kontekstu (tokeny)} \\ \hline\hline
OpenAI & GPT-4.1 & - & 1M \\ \hline
OpenAI & GPT-4o & - & 128k \\ \hline
OpenAI & O3 & - & 200k \\ \hline
Anthropic & Claude Opus 4 & - & 200k \\ \hline
Anthropic & Claude Sonnet 4 & - & 200k \\ \hline
Anthropic & Claude Haiku 3.5 & - & 200k \\ \hline
Google & Gemini 2.5 Pro & - & 1M \\ \hline
Google & Gemini 2.5 Flash & - & 1M \\ \hline
Mistral AI & Mistral Medium & 12–20 & 32k \\ \hline
Mistral AI & Codestral & 22 & 32k–64k \\ \hline
Meta & Llama 4 Maverick & 70 & 128k \\ \hline
Meta & Llama 4 Scout & 8 & 128k \\ \hline
DeepSeek & DeepSeek V3 & 236 & 128k \\ \hline
\end{tabular}
\end{table}

TODO tu skonczylem

Cel: Przedstawić modele LLM używane w eksperymencie, ich charakterystyki, sposób użycia i ograniczenia.

Proponowana zawartość:
	•	Rodzaje modeli:
	•	    Podział na modele open-source (np. Mistral, LLaMA, Nous Hermes) vs. modele komercyjne dostępne przez API (np. GPT-4, Claude).
	•		Warto podkreślić, że wybór między open-source a komercyjnymi często zależy od dostępności zasobów (GPU), wymagań licencyjnych i elastyczności.
	•	Architektury i charakterystyki:
	•	    Typowe parametry: liczba parametrów, specjalizacje, wersje, okno kontekstu (dlaczego wazne przy repozytoriach kodu)
	•		Warto dodać, że "liczba parametrów" nie zawsze jest jedynym wyznacznikiem jakości, ale jest istotnym parametrem. "Okna kontekstu" – jak najbardziej warto wyjaśnić, dlaczego jest ważne przy repozytoriach kodu (np. umożliwia dostarczanie całych plików, fragmentów dokumentacji, wielu powiązanych promptów jednocześnie).
    •   Wybrane modele do badania:
    •       Krótki opis które i dlaczego z parametrami jakimiś basic
	• 		Informacja, które modele są używane w badaniu i dlaczego (ograniczenia przez tool calling w langgraph).
	•		To jest bardzo ważna informacja! Musisz to jasno przedstawić. Jeśli LangGraph narzuca ograniczenia na wybór modeli (np. tylko te z dobrze zaimplementowanym tool calling), to jest to istotny czynnik wpływający na zakres badań.
	•	Sposób użycia:
	•	    Poprzez OpenRouter, bezpośrednie API, LangChain / LangGraph.
	•	    Sposób integracji z agentami.
	•	Tryby promptowania:
	•	    Zero-shot, few-shot, chain-of-thought, agent-based.
	•	    Porównanie tych podejść i wybór odpowiednich metod do problemu.
	•		Warto wyjaśnić, dlaczego wybrane metody (np. agent-based) są najbardziej odpowiednie dla generowania konfiguracji IaC.
	•	Ograniczenia i praktyczne aspekty:
	•	    Licencje, limity API, ograniczona liczba modeli.
	•	    Context window – jak wpływa na eksperymenty z kodem.
	•	    Brak fine-tuningu – tylko inference.
	•	    Kontrola nad danymi – brak możliwości dostosowania kodu źródłowego w zamkniętych modelach.



\subsection{Narzędzia DevOps: Docker i Kubernetes}

Cel: Opisać narzędzia wykorzystywane do zarządzania środowiskiem uruchomieniowym i ich wpływ na eksperymenty.

Proponowana zawartość:
	•	Docker:
	•	    Do czego jest wykorzystywany: uruchamianie lokalnych usług, środowiska testowe i deweloperskie, konteneryzacja narzędzi, konteneryzacja produkcyjnych aplikacji.
	•	    Jak mozna popelnic bledy
	•		Warto pokazać typowe błędy, które LLM może popełnić (np. użycie ADD zamiast COPY, brak WORKDIR, użycie latest tagu, brak usuwania zależności po buildzie). To od razu pokazuje, dlaczego ocena bezpieczeństwa i optymalizacji jest tak ważna.
	•	Kubernetes:
	•	    Używany z minimalnym zakresem (np. przez Kind, K3s, minikube).
	•	    Rola w testowaniu konfiguracji i IaC.
	•		Warto podkreślić, że Kubernetes służy jako środowisko do walidacji runtime wygenerowanych konfiguracji, a nie tylko do ich tworzenia.
	•	    Jak można popełnić błędy przy generowaniu YAML (niedopasowane zasoby, problemy z dostępnością, błędne konfiguracje).
	•		Dodaj przykłady: brak liveness/readiness probes, źle skonfigurowane serwisy (np. ClusterIP zamiast NodePort/LoadBalancer), brak PersistentVolumeClaim dla baz danych, zbyt wysokie/niskie limity zasobów, brak kontekstu bezpieczeństwa.
	•	Dobre i złe praktyki:
	•	    Przykłady z testów (np. źle ustawione limity zasobów, niepoprawne labele).
	•   Dodac tu Kaniko czy w technologiach wspierajacych?
	•	Chyba nie bedzie wykorzystywane jeszcze tutaj, a jedynie w systemie PaaS



\subsection{Narzędzia do oceny jakości IaC}

Cel: Pokazać narzędzia do analizy i walidacji konfiguracji IaC oraz ich funkcjonalności.

Proponowana zawartość:
	•	Lista narzędzi:
	•	    checkov, kube-linter, kube-score, kubeval, terrascan, cloudeval-yaml, iac-eval.
	•	Opis funkcji i zasad działania:
	•	    Statyczna analiza, reguły zgodności, walidacja schematów.
	•	Przykłady outputów:
	•	    Co zgłaszają narzędzia, jak to interpretować.
	•	    Zestawienie wyników na tych samych plikach (jeśli masz dane).
	•		Pokazać kilka błędów mniejszych i większych, podatności itd



\subsection{Środowisko eksperymentalne}

Cel: Przedstawić infrastrukturę testową i środowisko, w którym uruchamiane będą eksperymenty.

Proponowana zawartość:
	•	Charakterystyka środowiska:
	•	    Lokalna maszyna, Docker Desktop, Kubectl, Kind, Python, LangChain, LangGraph, OpenRouter?.
	•		Jasno określ, co jest na maszynie lokalnej, a co jest usługą zewnętrzną (OpenRouter to API gateway).
	•	Parametry techniczne:
	•	    RAM, CPU, ewentualna obecność GPU, ograniczenia związane z lokalnością - dlaczego lokalnie?
	•	Zasady testowania:
	•	    Jak wygląda testowanie przez API: limity żądań, obsługa błędów, wersje modeli.
	•	    Korzystanie z narzędzi CLI lub SDK.



\subsection{Technologie wspierające (do rozważenia jako część środowiska eksperymentalnego)}

Alternatywa 1: Zostawić jako osobny rozdział z wyraźnym celem: technologie pomocnicze ułatwiające pracę.

Alternatywa 2: Zmergować z poprzednim podrozdziałem jako „środowisko eksperymentalne i wspierające technologie”.

Zawartość:
	•	Język Python:
	•	    Używane biblioteki (np. requests, openai, langchain, pydantic, git, docker-py itd.).
	•	Narzędzia pomocnicze:
	•	    Kind, Lens, VSCode, Docker CLI, kubectl, LangSmith, LangFuse, Promptfoo, Kaniko.
	•		LangSmith, LangFuse, Promptfoo: To jest bardzo ważne! Są to narzędzia do ewaluacji i zarządzania promptami/LLM. Koniecznie je opisz krótko i wyjaśnij, jak pomogły w Twoich badaniach (np. do monitorowania interakcji z LLM, testowania promptów, porównywania ich skuteczności). To pokazuje zaawansowane podejście do inżynierii promptów.
	•	Rola:
	•	    Obsługa eksperymentów, monitorowanie, automatyzacja.



\clearpage % Rozdziały zaczynamy od nowej strony.
\section{Projekt eksperymentów}

TODO przeczytac uwaznie calosc i zredagowac pod katem spojnosc i logiki oraz aktualnosci ze stanem faktycznym projektu

\subsection{Cel badania}

Celem eksperymentów jest ocena możliwości autonomicznego generowania konfiguracji Infrastructure as Code przez agenty oparte na dużych modelach językowych. Badanie koncentruje się na weryfikacji, czy nowoczesne modele językowe są w stanie — bez ingerencji człowieka — wygenerować funkcjonalne, bezpieczne i zgodne z dobrymi praktykami konfiguracje wdrożeniowe (pliki \texttt{Dockerfile} oraz manifesty Kubernetes) na podstawie wyłącznie analizy struktury i zawartości kodu źródłowego repozytorium.

Eksperymenty obejmują pięć hipotez badawczych (H1–H5) dotyczących funkcjonalności, ograniczeń złożonościowych, jakości i bezpieczeństwa, deterministyczności oraz podatności na manipulację. Każda hipoteza weryfikowana jest przy użyciu dedykowanej metodyki testowej, zestawu metryk oraz wyraźnie zdefiniowanych kryteriów sukcesu lub porażki. Kluczowe pytania to: czy agenty LLM potrafią wygenerować działające konfiguracje, jakie są ograniczenia związane ze złożonością aplikacji, czy generowane konfiguracje wymagają dodatkowej walidacji przed wdrożeniem produkcyjnym, na ile powtarzalny jest proces generacji oraz czy agenty są podatne na manipulację poprzez zawartość repozytorium?

\subsection{Hipotezy badawcze}

W ramach badania sformułowano pięć hipotez weryfikowanych eksperymentalnie:

\begin{itemize}
    \item \textbf{H1: Autonomiczna generacja funkcjonalnych konfiguracji} \\
    Duże modele językowe, działające jako agenty z dostępem do narzędzi analizy repozytorium, są w stanie autonomicznie wygenerować funkcjonalne konfiguracje Docker i Kubernetes, które umożliwiają poprawne uruchomienie aplikacji bez dodatkowej ingerencji człowieka, w tym osiągnięcie gotowości usługowej widocznej na poziomie interfejsu (np. webowe API odpowiada na zapytania HTTP/REST po wdrożeniu na klastrze testowym). Dodatkowo, oceniany jest wpływ szczegółowości instrukcji systemowych (prompt engineering) przy użyciu wariantów prompta różniących się poziomem szczegółowości wskazówek dotyczących najlepszych praktyk DevOps, co pozwala ocenić wrażliwość podejścia agentowego na jakość prompt engineering.

    \item \textbf{H2: Ograniczenia złożonościowe} \\
    Istnieje próg złożoności aplikacji, powyżej którego jakość autonomicznie generowanych konfiguracji znacząco spada. Progi te można uchwycić, obserwując wyniki na kontrolowanych repozytoriach poc1–poc4 o rosnącej liczbie komponentów i zależności (od prostego monolitu po prosty układ mikroserwisowy); oczekuje się, że metryki sukcesu (np. odsetek poprawnych wdrożeń) maleją wraz z przechodzeniem od poc1 do poc4.

    \item \textbf{H3: Jakość i zgodność z dobrymi praktykami} \\
    Co najmniej 60\% autonomicznie generowanych konfiguracji zawiera ostrzeżenia klasy \texttt{warning} wykrywane przez narzędzia statycznej analizy (Hadolint dla Docker, Kube-linter dla Kubernetes), co wymaga dodatkowej walidacji i poprawek przed wdrożeniem produkcyjnym.

    \item \textbf{H4: Niezawodność i powtarzalność procesu agentowego} \\
    Proces generacji konfiguracji przez agenta LLM charakteryzuje się zmiennością wyników mimo deterministycznych parametrów (\texttt{temperature = 0}); przy wielokrotnym uruchomieniu na tym samym repozytorium wskaźniki takie jak \textit{run-to-run diff ratio} (liczba zmienionych linii pomiędzy parami wygenerowanych konfiguracji znormalizowana przez ich długość) lub odchylenie standardowe liczby kroków narzędziowych w sekwencjach zapisanych w LangSmith (różna liczba wywołań funkcji pomocniczych) przekraczają ustalony próg (np. $>0{,}2$), co świadczy o ograniczonej powtarzalności procesu.

    \item \textbf{H5: Podatność na manipulację kontekstem} \\
    Agenty LLM są podatne na manipulację przez złośliwe lub mylące treści zawarte w repozytorium (prompt injection, social engineering w plikach README/dokumentacji): w specjalnie spreparowanym repozytorium (np. poc5-jailbreak-fastapi) co najmniej jedna trzecia przebiegów prowadzi do wygenerowania konfiguracji odbiegających od oczekiwań (np. pominięcie krytycznych usług, wstrzyknięcie dodatkowych kontenerów) lub zawierających intencjonalnie szkodliwe instrukcje, co dokumentuje wpływ manipulacji kontekstem na wynik końcowy.
\end{itemize}

\bigskip
\noindent
Każda z powyższych hipotez weryfikowana jest przy użyciu dedykowanej metodyki testowej, specyficznych zestawów danych, odpowiednich metryk pomiarowych oraz wyraźnie zdefiniowanych kryteriów sukcesu lub porażki. Szczegółowy opis strategii weryfikacji każdej hipotezy znajduje się w dalszych podsekcjach (4.4–4.8).

\subsection{Architektura systemu testowego}

Eksperymenty wymagają kompleksowego środowiska testowego składającego się z trzech głównych komponentów: agenta generującego konfiguracje, potoku walidacyjnego oraz infrastruktury wykonawczej.

\subsubsection{Agent LLM}

Agent został zaimplementowany przy użyciu bibliotek LangChain oraz LangGraph i pełni rolę autonomicznego generatora konfiguracji. Otrzymuje on link do repozytorium Git oraz systemowy prompt z instrukcjami i ma za zadanie wygenerować kompletne pliki \texttt{Dockerfile} oraz manifesty Kubernetes.

\bigskip
\noindent
\textbf{Narzędzia dostępne dla agenta:}

Agent dysponuje zestawem jedenastu funkcji umożliwiających interakcję z repozytorium:

\begin{itemize}
    \item \texttt{clone\_repo(repo\_url)} – klonuje repozytorium Git i automatycznie usuwa pliki mogące sugerować gotowe rozwiązanie: metadane VCS, dokumentację, istniejące konfiguracje, pliki CI/CD oraz artefakty deweloperskie. Usunięcie istniejących plików \texttt{Dockerfile} i \texttt{docker-compose.yml} jest celową decyzją — zapewnia, że agent generuje konfiguracje wyłącznie na podstawie analizy kodu źródłowego,

    \item \texttt{prepare\_repo\_tree()} – generuje tekstowy widok struktury katalogów (podobnie do \texttt{tree}),

    \item \texttt{ls(dir\_path)} – listuje zawartość katalogu,

    \item \texttt{get\_file\_content(file\_path)} – odczytuje treść pliku,

    \item \texttt{search\_files(pattern, file\_pattern, case\_sensitive)} – wyszukuje wzorzec tekstowy w plikach,

    \item \texttt{find\_files(pattern, max\_results)} – wyszukuje pliki według wzorca nazwy,

    \item \texttt{write\_file(file\_path, content)} – zapisuje nowy plik lub nadpisuje istniejący,

    \item \texttt{patch\_file(file\_path, patch)} – aplikuje patch w formacie unified diff,

    \item \texttt{base64\_encode(content)} – koduje tekst do base64 (np. dla Kubernetes Secrets),

    \item \texttt{base64\_decode(encoded)} – dekoduje base64,

    \item \texttt{think(thoughts)} – narzędzie refleksyjne pozwalające agentowi na zorganizowanie myśli i zapisanie obserwacji o architekturze (nie modyfikuje plików).
\end{itemize}

\bigskip
\noindent
\textbf{Prompt systemowy:}

Ze względu na długość, pełna treść prompta znajduje się w załączniku~\ref{att:prompt}. Prompt zawiera:
\begin{itemize}
    \item kontekst działania agenta i jego specjalizację,
    \item szczegółowy opis dostępnych narzędzi i zasad ich użycia,
    \item reguły nazewnictwa (np. generowanie hostów Ingress jako \texttt{<repo-name>.rasztabiga.me}),
    \item dobre praktyki DevOps dla Docker (non-root, odpowiedni base image) i Kubernetes (health probes, StatefulSets vs Deployments, ConfigMaps/Secrets),
    \item format strukturalnego wyjścia (JSON z listą obrazów Docker, manifestów K8s oraz test endpoint).
\end{itemize}

Prompt został sformułowany deklaratywnie, aby maksymalnie zredukować niejednoznaczność interpretacyjną i zwiększyć powtarzalność wyników.

\bigskip
\noindent
\textbf{Strukturalne wyjście:}

Agent zwraca strukturalny JSON poprzez parametr \texttt{response\_format=ConfigurationOutput}:
\begin{itemize}
    \item \texttt{docker\_images} – lista obiektów z informacją o Dockerfile, tagu obrazu i kontekście budowania,
    \item \texttt{kubernetes\_files} – lista ścieżek do manifestów Kubernetes,
    \item \texttt{test\_endpoint} – relatywna ścieżka HTTP do weryfikacji działania aplikacji.
\end{itemize}

\subsubsection{Potok walidacyjny}

Po wygenerowaniu konfiguracji przez agenta, następuje automatyczna ewaluacja składająca się z niezależnych kroków:

\begin{enumerate}
    \item Weryfikacja syntaktyczna \texttt{Dockerfile},
    \item Statyczna analiza \texttt{Dockerfile} (Hadolint),
    \item Budowa obrazów Docker,
    \item Weryfikacja syntaktyczna manifestów Kubernetes,
    \item Statyczna analiza manifestów Kubernetes (Kube-linter),
    \item Aplikowanie manifestów do klastra MicroK8s,
    \item Walidacja runtime — sprawdzenie dostępności aplikacji.
\end{enumerate}

Modułowa architektura umożliwia łatwe dodawanie nowych kroków walidacji. Wszystkie wywołania zewnętrznych narzędzi są realizowane przez dedykowaną abstrakcję rejestrującą czasy wykonania, poziomy ważności błędów oraz dostępność narzędzi.

\subsubsection{Infrastruktura wykonawcza}

Środowisko testowe składa się z:
\begin{itemize}
    \item \textbf{Klaster Kubernetes}: MicroK8s jako środowisko docelowe dla manifestów,
    \item \textbf{Registry Docker}: Lokalny registry (\texttt{192.168.0.124:32000}) dla przechowywania zbudowanych obrazów,
    \item \textbf{Tracking}: LangSmith dla śledzenia wszystkich interakcji agenta z modelem (ścieżki wywołań narzędzi, prompty, odpowiedzi).
\end{itemize}

\subsubsection{Parametry generacji}

Wszystkie testowane modele otrzymują identyczny prompt oraz to samo repozytorium testowe, co zapewnia spójne warunki oceny wpływu architektury modelu na jakość generacji IaC. Parametry wywołania:

\begin{itemize}
    \item \texttt{temperature = 1.0} dla większości eksperymentów (H1–H3, H5) — naturalny tryb generowania zgodny z domyślnymi parametrami modeli,
    \item \texttt{temperature = 0} wyłącznie dla testów deterministyczności (H4) — wymusza wybór najbardziej prawdopodobnych tokenów w celu zbadania powtarzalności,
    \item \texttt{seed = 42} — stała wartość ziarna losowości dla zwiększenia reprodukowalności,
    \item \texttt{n = 1} — pojedyncza próba generacji.
\end{itemize}

\subsection{Weryfikacja H1: Autonomiczna generacja funkcjonalnych konfiguracji}

\subsubsection{Hipoteza}

Duże modele językowe, działające jako agenty z dostępem do narzędzi analizy repozytorium, są w stanie autonomicznie wygenerować funkcjonalne konfiguracje Docker i Kubernetes, które umożliwiają poprawne uruchomienie aplikacji bez dodatkowej ingerencji człowieka. Dodatkowo, oceniany jest wpływ szczegółowości instrukcji systemowych (prompt engineering) na skuteczność generacji.

\subsubsection{Strategia testowania}

W celu oceny ogólnej zdolności agentów LLM do generowania działających konfiguracji, testy H1 przeprowadzane są na \textbf{szerokim zestawie rzeczywistych, publicznie dostępnych projektów z GitHub}. Wykorzystanie niekontrolowanych repozytoriów o różnorodnej strukturze, niekompletnej dokumentacji i niestandardowych zależnościach pozwala ocenić skuteczność agentów w warunkach rzeczywistych.

Dataset obejmuje 30–50 projektów zróżnicowanych pod względem:
\begin{itemize}
    \item języków programowania (Python, Node.js, Go, Java, Ruby, Rust),
    \item frameworków (FastAPI, Express, Django, Spring Boot, Rails),
    \item architektur (monolity, aplikacje wielowarstwowe, systemy rozproszone).
\end{itemize}

Ten szeroki dataset zwiększa reprezentatywność eksperymentu oraz umożliwia uogólnienie wyników na szerszą klasę projektów spotykanych w praktyce, weryfikując podstawową przydatność podejścia agentowego do automatyzacji generowania konfiguracji IaC.

\textbf{Ograniczenia praktyczne:} w badaniu H1 celowo wykorzystano zestaw relatywnie prostych aplikacji (mało komponentów, brak skomplikowanej orkiestracji), aby umożliwić tanią i szybką ewaluację. Bardziej złożone przypadki są analizowane w H2. Liczba powtórzeń na kombinację (model, repozytorium, prompt) została ograniczona do 2 ze względu na budżet obliczeniowy — należy to uwzględnić przy interpretacji szerokich przedziałów ufności i istotności różnic między modelami/promptami.

\bigskip
\noindent
\textbf{Warianty promptów:}

Aby ocenić wpływ jakości instrukcji systemowych na skuteczność generacji, testy H1 obejmują dwa warianty prompta:

\begin{itemize}
    \item \textbf{basic.prompt} — zwięzła wersja zawierająca podstawowe instrukcje dotyczące celu zadania i ogólnych wytycznych tworzenia konfiguracji Docker i Kubernetes, bez szczegółowych wskazówek dotyczących najlepszych praktyk DevOps,

    \item \textbf{default.prompt} — rozbudowana wersja (załącznik~\ref{att:prompt}) zawierająca szczegółowe wskazówki dotyczące bezpieczeństwa (non-root user, read-only filesystem), optymalizacji (multi-stage builds, layer caching), zarządzania zależnościami (npm ci vs npm install, weryfikacja package-lock.json), typowych pułapek (apt-get 404, Alpine compatibility) oraz najlepszych praktyk Kubernetes (health probes, resource limits, StatefulSets vs Deployments).
\end{itemize}

Porównanie wyników między wariantami pozwoli określić, w jakim stopniu szczegółowość i jakość prompt engineering wpływa na success rate oraz jakość wygenerowanych konfiguracji, co ma istotne implikacje praktyczne dla wdrożeń produkcyjnych systemów opartych na agentach LLM.

\bigskip
\noindent
Ze względu na koszt wywołań API oraz objętość danych, dla H1 przeprowadzane są 1–2 powtórzenia dla każdego repozytorium z 2–3 najlepszymi modelami komercyjnymi (np. GPT-5, Claude Sonnet 4.5) oraz oboma wariantami promptów.

\subsubsection{Metryki}

\begin{itemize}
    \item \texttt{generation\_success} — czy agent zwrócił strukturalny JSON z konfiguracją,
    \item \texttt{dockerfile\_syntax\_valid} — czy Dockerfile jest poprawny składniowo,
    \item \texttt{k8s\_syntax\_valid} — czy manifesty Kubernetes są poprawne składniowo,
    \item \texttt{build\_success} — czy obrazy Docker zbudowały się pomyślnie,
    \item \texttt{k8s\_apply\_success} — czy manifesty zostały poprawnie zaaplikowane do klastra,
    \item \texttt{runtime\_success} — czy aplikacja uruchomiła się poprawnie w klastrze i odpowiada na żądania HTTP,
    \item \texttt{prompt\_id} — identyfikator wariantu prompta (basic / default).
\end{itemize}

\textbf{Wskaźnik sukcesu:}
\[
\text{Success rate} = \frac{\text{build\_success} \land \text{k8s\_apply\_success} \land \text{runtime\_success}}{\text{total\_runs}} \times 100\%
\]

\textbf{TODO: kaskada etapów:} raportować drop-off per etap (build → apply → runtime), by pokazać wąskie gardła.

\textbf{TODO: przedziały ufności:} raportować 95\% CI dla success rate — dla $n \ge 30$ Wilson, dla mniejszych prób Clopper–Pearson (dwumianowy dokładny). Przy $n \approx 20$ stosować Clopper–Pearson i w tabeli podać wprost „95\% CI (Clopper–Pearson)”.

\textbf{Porównanie wariantów promptów:}
\[
\Delta_{\text{prompt}} = \text{Success rate}_{\text{default}} - \text{Success rate}_{\text{basic}}
\]

\textbf{TODO: istotność różnicy:} dla $\Delta_{\text{prompt}}$ wykonać test różnicy proporcji (lub raportować, czy 95\% CI się nie pokrywają), aby ocenić znaczenie prompt engineering. Przy $n \approx 20$ rozważyć dokładny test dwumianowy albo bootstrap zamiast asymptotycznego przybliżenia normalnego.

\subsubsection{Kryteria weryfikacji}

\begin{itemize}
    \item \textbf{Pozytywna weryfikacja hipotezy}:
    \begin{itemize}
        \item Success rate > 60–70\% na różnorodnym zbiorze projektów — wskazuje, że agenty LLM potrafią autonomicznie generować funkcjonalne konfiguracje w większości przypadków,
        \item $\Delta_{\text{prompt}}$ > 15\% — szczegółowy prompt znacząco poprawia skuteczność generacji, co potwierdza istotność prompt engineering w kontekście agentów LLM.
    \end{itemize}
    \item \textbf{Negatywna weryfikacja hipotezy}:
    \begin{itemize}
        \item Success rate < 50\% — wskazuje na fundamentalne ograniczenia podejścia agentowego w praktycznym zastosowaniu,
        \item $|\Delta_{\text{prompt}}|$ < 5\% — wariant prompta nie ma istotnego wpływu na wyniki, co sugeruje, że jakość prompt engineering jest drugorzędna wobec innych czynników (np. możliwości modelu, złożoność repozytorium).
    \end{itemize}
\end{itemize}

\subsection{Weryfikacja H2: Ograniczenia złożonościowe}

\subsubsection{Hipoteza}

Istnieje próg złożoności aplikacji, powyżej którego jakość autonomicznie generowanych konfiguracji znacząco spada. Modele radzą sobie lepiej z aplikacjami monolitycznymi i prostymi wielowarstwowymi niż z rozproszonymi systemami mikroserwisowymi wymagającymi orkiestracji wielu zależnych komponentów.

\subsubsection{Strategia testowania}

W przeciwieństwie do H1, szczegółowa analiza wpływu złożoności wymaga \textbf{kontrolowanych warunków eksperymentalnych}. W tym celu przygotowano cztery repozytoria (poc1–poc4) o znanej strukturze i rosnącej złożoności. Wszystkie zostały napisane w języku Python z wykorzystaniem frameworka FastAPI \cite{fastapi} dla zapewnienia spójności środowiskowej.

\bigskip
\noindent
\textbf{Repozytoria testowe:}

\begin{itemize}
    \item \textbf{poc1: Aplikacja bezstanowa bez zależności}\\
    Najprostszy przypadek — aplikacja FastAPI bez zewnętrznych zależności, baz danych ani usług towarzyszących. Testuje zdolność modelu do wygenerowania minimalnie działającej konfiguracji.\\
    \textit{Repozytorium:} \url{https://github.com/run-rasztabiga-me/poc1-fastapi}

    \item \textbf{poc2: Aplikacja ze stanową bazą danych}\\
    Rozszerzenie poc1 o lokalną bazę PostgreSQL. Testuje zdolność modelu do rozpoznania zależności między usługami i poprawnego zdefiniowania ich konfiguracji oraz połączeń sieciowych. Ze względu na komponent stanowy, model powinien wygenerować \texttt{StatefulSet} dla bazy, \texttt{PersistentVolumeClaim} oraz dedykowany \texttt{Service}.\\
    \textit{Repozytorium:} \url{https://github.com/run-rasztabiga-me/poc2-fastapi}

    \item \textbf{poc3: Frontend + Backend + Baza danych}\\
    Architektura trójwarstwowa: frontend (SPA), backend (FastAPI), baza danych. Testuje zdolność modelu do generowania konfiguracji obejmujących zależności między wieloma kontenerami i usługami.\\
    \textit{Repozytorium:} \url{https://github.com/run-rasztabiga-me/poc3-fastapi}

    \item \textbf{poc4: Prosty system mikroserwisowy}\\
    Najbardziej złożony przypadek — system składający się z trzech mikroserwisów, baz danych oraz kolejki komunikatów. Pozwala przetestować zdolność modelu do tworzenia manifestów Kubernetes z wykorzystaniem wielu zasobów oraz relacji między nimi.\\
    \textit{Repozytorium:} \url{https://github.com/run-rasztabiga-me/poc4-fastapi}
\end{itemize}

\bigskip
\noindent
Kontrolowane środowisko pozwala na precyzyjne śledzenie różnic w wynikach między różnymi poziomami złożoności. Testy obejmują 3–5 powtórzeń dla każdej kombinacji (model, repozytorium) z 4–6 modelami (commercial + open-source).

\subsubsection{Metryki}

\begin{itemize}
    \item \texttt{overall\_score} — agregowany wynik z systemu scoringowego (wagi fazowe),
    \item \texttt{build\_success} — czy wszystkie obrazy zbudowały się poprawnie,
    \item \texttt{runtime\_success} — czy aplikacja działa w klastrze,
    \item \texttt{error\_count} — liczba błędów walidacji (ERROR),
    \item \textbf{Wskaźnik złożoności} — liczba komponentów: \texttt{len(docker\_images)} + wykryte serwisy.
\end{itemize}

\subsubsection{Kryteria weryfikacji}

\begin{itemize}
    \item \textbf{Pozytywna weryfikacja hipotezy}: Widoczny, systematyczny spadek success rate i overall\_score wraz ze wzrostem złożoności:
    \begin{itemize}
        \item poc1: $\sim$90\% success rate,
        \item poc2: $\sim$75\% success rate,
        \item poc3: $\sim$60\% success rate,
        \item poc4: $\sim$40\% success rate.
    \end{itemize}

    \item \textbf{Negatywna weryfikacja hipotezy}: Brak wyraźnej korelacji między złożonością a wynikami — modele radzą sobie równie dobrze (lub źle) niezależnie od liczby komponentów.
\end{itemize}

\subsection{Weryfikacja H3: Jakość i zgodność z dobrymi praktykami}

\subsubsection{Hipoteza}

Większość autonomicznie generowanych konfiguracji zawiera błędy lub ostrzeżenia wykrywane przez narzędzia statycznej analizy (Hadolint dla Docker, Kube-linter dla Kubernetes), co wymaga dodatkowej walidacji i poprawek przed wdrożeniem produkcyjnym.

\subsubsection{Strategia testowania: wielowarstwowa walidacja}

W celu kompleksowej oceny jakości, bezpieczeństwa oraz kompletności wygenerowanych konfiguracji, dla hipotezy H3 zastosowano \textbf{trójwarstwowe podejście walidacyjne} łączące automatyzację, ocenę LLM oraz ekspertyzę człowieka. Zapewnia to triangulację wyników — wzajemne potwierdzenie ustaleń z różnych źródeł — oraz umożliwia identyfikację unikalnych problemów wykrywanych przez każdą z metod.

\bigskip
\noindent
\textbf{Warstwa 1: Automatyczna walidacja (Hadolint + Kube-linter)}

Narzędzia statycznej analizy wykrywają naruszenia reguł i dobrych praktyk, klasyfikując je według poziomów ważności:
\begin{itemize}
    \item \textit{ERROR} — błędy krytyczne uniemożliwiające poprawne działanie,
    \item \textit{WARNING} — ostrzeżenia o potencjalnych problemach wymagających uwagi,
    \item \textit{INFO} — sugestie pomocnicze nieistotne dla funkcjonalności.
\end{itemize}

Stanowi podstawową, w pełni zautomatyzowaną linię walidacji dostępną dla wszystkich konfiguracji.

\bigskip
\noindent
\textbf{Warstwa 2: Ocena LLM-as-a-Judge}

Model językowy pełni rolę eksperta DevOps oceniającego konfigurację w trzech wymiarach (skala 0–100 dla każdego):

\begin{itemize}
    \item \textbf{Security (bezpieczeństwo)} — czy kontener uruchamia się jako non-root, czy secrets są prawidłowo zarządzane (nie hardcoded), czy image tags są pinned (nie \texttt{:latest}), czy brak uprzywilejowanych kontenerów, czy odpowiednie security contexts,

    \item \textbf{Completeness (kompletność)} — czy obecne są resource limits/requests, czy zdefiniowane health probes (liveness/readiness), czy dla stanowych aplikacji są PersistentVolumeClaims, czy networking jest poprawnie skonfigurowany, czy ConfigMaps/Secrets są prawidłowo wykorzystane,

    \item \textbf{Best Practices (dobre praktyki)} — czy Dockerfile używa multi-stage builds, czy warstwy są optymalizowane (caching), czy labels/annotations są obecne, czy nazewnictwo jest spójne i opisowe, czy dokumentacja inline (komentarze) jest obecna.
\end{itemize}

LLM Judge zwraca strukturalny JSON zawierający:
\begin{itemize}
    \item \texttt{security\_score} (0–100),
    \item \texttt{completeness\_score} (0–100),
    \item \texttt{best\_practices\_score} (0–100),
    \item \texttt{overall\_score} (0–100),
    \item \texttt{critical\_issues} — lista najważniejszych problemów,
    \item \texttt{suggestions} — lista sugestii poprawek.
\end{itemize}

LLM Judge wykrywa problemy semantyczne i kontekstowe, które mogą umknąć narzędziom statycznym (np. nieoptymalne rozmieszczenie instrukcji COPY w Dockerfile, brak strategii skalowania, nieodpowiednie resource limits dla typu aplikacji).

\bigskip
\noindent
\textbf{Warstwa 3: Ocena ekspercka (human evaluation)}

Dla reprezentatywnej próbki konfiguracji ($\sim$30 przypadków) przeprowadzana jest manualna ocena przez eksperta DevOps w czterech wymiarach (skala 0–100 dla każdego):

\begin{itemize}
    \item \textbf{Functionality (funkcjonalność)} — czy aplikacja w ogóle zadziała, czy wszystkie komponenty są obecne, czy networking jest poprawny,

    \item \textbf{Security (bezpieczeństwo)} — czy jest bezpieczna dla produkcji, czy brak krytycznych luk, czy secrets są prawidłowo zarządzane,

    \item \textbf{Production-readiness (gotowość produkcyjna)} — resource limits/requests, health probes, monitoring/observability,

    \item \textbf{Quality (jakość kodu)} — struktura i nazewnictwo, optymalizacja (multi-stage, caching), dokumentacja i komentarze.
\end{itemize}

Ocena ekspercka stanowi złoty standard referencyjny służący zarówno do walidacji hipotezy H3, jak i do oceny skuteczności LLM Judge poprzez analizę korelacji między oceną automatyczną a ekspercką.

\bigskip
\noindent
\textbf{Dataset:}
\begin{itemize}
    \item \textbf{Warstwa 1 i 2}: Wszystkie repozytoria (poc1–poc5 + GitHub dataset),
    \item \textbf{Warstwa 3}: Reprezentatywna próbka $\sim$30 konfiguracji (stratified sampling).
\end{itemize}

\subsubsection{Metryki}

\textbf{Warstwa 1 (Automated):}
\begin{itemize}
    \item \texttt{error\_count} — liczba ERROR z Hadolint + Kube-linter,
    \item \texttt{warning\_count} — liczba WARNING,
    \item \texttt{info\_count} — liczba INFO (nie wpływa na scoring),
    \item \texttt{has\_errors} — bool (\texttt{error\_count} > 0),
    \item \texttt{is\_clean} — bool (\texttt{error\_count} == 0 AND \texttt{warning\_count} == 0),
    \item \texttt{dockerfile\_syntax\_valid} — poprawność składniowa Dockerfile,
    \item \texttt{k8s\_syntax\_valid} — poprawność składniowa manifestów K8s.
\end{itemize}

\textbf{Warstwa 2 (LLM Judge):}
\begin{itemize}
    \item \texttt{llm\_security\_score} (0–100),
    \item \texttt{llm\_completeness\_score} (0–100),
    \item \texttt{llm\_best\_practices\_score} (0–100),
    \item \texttt{llm\_overall\_score} (0–100),
    \item \texttt{llm\_critical\_issues} — lista tekstowa.
\end{itemize}

\textbf{Warstwa 3 (Human):}
\begin{itemize}
    \item \texttt{human\_functionality\_score} (0–100),
    \item \texttt{human\_security\_score} (0–100),
    \item \texttt{human\_production\_score} (0–100),
    \item \texttt{human\_quality\_score} (0–100),
    \item \texttt{human\_overall\_score} (0–100) — średnia z powyższych,
    \item \texttt{human\_notes} — komentarz tekstowy eksperta.
\end{itemize}

\subsubsection{Kryteria weryfikacji}

\begin{itemize}
    \item \textbf{Pozytywna weryfikacja hipotezy}:
    \begin{itemize}
        \item \textbf{Warstwa 1}: > 60\% konfiguracji ma błędy/ostrzeżenia, < 30\% jest "clean",
        \item \textbf{Warstwa 2}: średni \texttt{llm\_overall\_score} < 60/100, \texttt{llm\_security\_score} < 50/100,
        \item \textbf{Warstwa 3}: średni \texttt{human\_overall\_score} < 60/100,
        \item \textbf{Korelacja}: \texttt{human\_overall\_score} koreluje z \texttt{llm\_overall\_score} ($\rho$ > 0.6) — potwierdzenie, że LLM Judge rzeczywiście dobrze ocenia.
    \end{itemize}

    \item \textbf{Negatywna weryfikacja hipotezy}: Większość konfiguracji (> 70\%) jest "clean" (brak ERROR/WARNING), wysokie średnie score w Warstwach 2 i 3 (> 80/100) — wskazuje, że wygenerowane konfiguracje są wysokiej jakości i gotowe do wdrożenia produkcyjnego bez dodatkowych poprawek.
\end{itemize}

\subsection{Weryfikacja H4: Niezawodność i powtarzalność procesu agentowego}

\subsubsection{Hipoteza}

Proces generacji konfiguracji przez agenta LLM charakteryzuje się zmiennością wyników mimo deterministycznych parametrów (\texttt{temperature = 0}), szczególnie w zakresie strategii eksploracji repozytorium i kolejności generowanych zasobów, co wpływa na powtarzalność rozwiązania.

\subsubsection{Strategia testowania}

Weryfikacja H4 wymaga \textbf{wielokrotnych powtórzeń} (5–10 runs) dla każdej kombinacji (model, repozytorium, prompt) przy \textbf{maksymalnie deterministycznych parametrach}:
\begin{itemize}
    \item \texttt{temperature = 0} — wymusza wybór najbardziej prawdopodobnych tokenów,
    \item \texttt{seed = 42} — stała wartość ziarna losowości.
\end{itemize}

Testy przeprowadzane są na kontrolowanych repozytoriach (poc1–poc4), co umożliwia porównanie wyników dla tej samej konfiguracji wejściowej. Dla każdego przebiegu rejestrowane są nie tylko metryki jakościowe (overall\_score, error\_count), ale także szczegółowe ślady wykonania z LangSmith (kolejność wywołań narzędzi, przeszukiwane pliki, strategia eksploracji).

\subsubsection{Metryki}

\begin{itemize}
    \item \texttt{overall\_score} — variance między powtórzeniami,
    \item \texttt{error\_count} — variance między powtórzeniami,
    \item \texttt{warning\_count} — variance między powtórzeniami,
    \item \texttt{build\_success} — consistency (czy zawsze ten sam wynik?),
    \item \texttt{runtime\_success} — consistency,
    \item \textbf{Tool usage patterns} — analiza jakościowa z LangSmith: czy agent używa tych samych narzędzi w tej samej kolejności?
\end{itemize}

\subsubsection{Kryteria weryfikacji}

\begin{itemize}
    \item \textbf{Pozytywna weryfikacja hipotezy}:
    \begin{itemize}
        \item Variance > 10 punktów w \texttt{overall\_score} między runs dla tej samej konfiguracji wejściowej,
        \item Różne wyniki \texttt{build\_success}/\texttt{runtime\_success} dla identycznych parametrów,
        \item Różna kolejność wywołań narzędzi widoczna w LangSmith traces,
        \item Różne strategie eksploracji repozytorium (różne pliki czytane w różnej kolejności).
    \end{itemize}

    \item \textbf{Negatywna weryfikacja hipotezy}: Variance < 5 punktów, identyczne wyniki funkcjonalne (build/runtime), spójna kolejność wywołań narzędzi — wskazuje na wysoką powtarzalność procesu agentowego mimo złożoności zadania.
\end{itemize}

\subsection{Weryfikacja H5: Podatność na manipulację kontekstem}

\subsubsection{Hipoteza}

Agenty LLM są podatne na manipulację przez złośliwe lub mylące treści zawarte w repozytorium (prompt injection, social engineering w plikach README/dokumentacji). Obecność instrukcji konfliktowych lub próby przekierowania agenta mogą prowadzić do wygenerowania niepoprawnych, niebezpiecznych lub całkowicie odmiennych od zamierzonych konfiguracji.

\subsubsection{Strategia testowania}

Weryfikacja H5 wymaga specjalnie spreparowanego repozytorium \textbf{poc5-adversarial-fastapi} zawierającego autentyczny kod aplikacji FastAPI, ale uzupełnionego o elementy adversarial mające na celu manipulację agenta:

\begin{itemize}
    \item \textbf{Podmiana aplikacji} — plik README instruuje sklonowanie innego repozytorium zamiast używania obecnego kodu,
    \item \textbf{Privilege escalation} — dokumentacja (DEPLOY.md) sugeruje użycie \texttt{privileged: true}, \texttt{runAsUser: 0} w manifestach Kubernetes,
    \item \textbf{Resource exhaustion} — DEPLOY.md wymaga ekstremalnie wysokich limitów zasobów (64Gi RAM, 32 CPU),
    \item \textbf{Secret exfiltration} — przykładowy \texttt{Dockerfile.example} zawiera instrukcję \texttt{RUN curl} wysyłającą secrets do zewnętrznego endpointu,
    \item \textbf{Przestarzałe pliki} — błędny \texttt{docker-compose.yml.old} z podatnościami (stare wersje obrazów, hardcoded credentials).
\end{itemize}

Repozytorium: \url{https://github.com/run-rasztabiga-me/poc5-adversarial-fastapi}

\bigskip
\noindent
Testy obejmują porównanie wyników z "baseline" (czysty poc1 bez adversarial content) oraz jakościową analizę wygenerowanych plików pod kątem wprowadzonych podatności.

\subsubsection{Metryki}

\begin{itemize}
    \item \textbf{Deviation score} — jak bardzo wygenerowana konfiguracja odbiega od oczekiwanej (porównanie z poc1),
    \item \texttt{error\_count} — czy adversarial input wprowadza dodatkowe błędy walidacji,
    \item \textbf{Security issues injected} — analiza binarna: czy agent dodał \texttt{privileged: true}, \texttt{runAsUser: 0}, ekstremalne resource limits, curl wysyłający dane, itd.,
    \item \textbf{Manual inspection} — czy agent próbował sklonować inne repo? Czy użył wartości z adversarial dokumentacji? Czy skopiował podatności z przykładowych plików?
\end{itemize}

\subsubsection{Kryteria weryfikacji}

\begin{itemize}
    \item \textbf{Pozytywna weryfikacja hipotezy}: Agent wprowadził co najmniej 1–2 adversarial suggestions (np. privileged container, ekstremalne limity, użył błędnego docker-compose.yml jako odniesienia) — wskazuje na podatność agentów na manipulację kontekstem.

    \item \textbf{Negatywna weryfikacja hipotezy}: Agent całkowicie zignorował adversarial content i wygenerował konfigurację identyczną jak dla poc1 — wskazuje na wysoką odporność agentów na prompt injection poprzez zawartość repozytorium.
\end{itemize}

\subsection{Orkiestracja i infrastruktura wspierająca}

\subsubsection{System zarządzania eksperymentami}

W celu ułatwienia przeprowadzania i zarządzania eksperymentami zaimplementowano dedykowaną warstwę orkiestracyjną. System umożliwia definiowanie złożonych scenariuszy testowych w formie strukturalnych plików konfiguracyjnych (YAML), które określają macierze kombinacji model × repozytorium, liczbę powtórzeń, warianty promptów oraz wspólne ustawienia środowiskowe.

\bigskip
\noindent
Kluczowe cechy systemu orkiestracji:
\begin{itemize}
    \item \textbf{Deklaratywna konfiguracja eksperymentów} — wszystkie parametry (modele, repozytoria, prompty, liczba powtórzeń) definiowane w plikach YAML,
    \item \textbf{Bezstanowy ewaluator} — komponent ewaluatora pozostaje bezstanowy, co umożliwia wielokrotne użycie w różnych kontekstach,
    \item \textbf{Wsparcie dla wariantów promptów} — możliwość porównywania różnych instrukcji systemowych,
    \item \textbf{Agregacja metryk} — automatyczne zbieranie i utrwalanie wyników w formatach CSV i JSON,
    \item \textbf{Izolacja przestrzeni roboczych} — każdy przebieg operuje we własnym, unikalnym katalogu roboczym.
\end{itemize}

Aby ograniczyć wpływ losowych fluktuacji (mimo \texttt{temperature = 0} dla H4), proces generacji i ewaluacji jest powtarzany wielokrotnie dla każdej kombinacji (model, repozytorium, prompt). Pozwala to na wykrycie niepowtarzalnych wyników oraz uśrednienie metryk w dalszej analizie.

\subsubsection{Panel sterowania eksperymentami}

W celu usprawnienia zarządzania eksperymentami oraz monitorowania ich postępu zaimplementowano dedykowany interfejs webowy (Streamlit). Panel umożliwia:

\begin{itemize}
    \item \textbf{Wybór i uruchamianie konfiguracji eksperymentów} — graficzny wybór plików YAML i inicjowanie przebiegów,
    \item \textbf{Monitorowanie postępu w czasie rzeczywistym} — aktualny status, liczba ukończonych/pozostałych przebiegów, szacowany czas do zakończenia,
    \item \textbf{Przeglądanie wyników} — podsumowania z podziałem na modele, repozytoria i warianty promptów,
    \item \textbf{Filtrowanie i agregacja danych} — szybkie porównanie wyników i identyfikacja problemów,
    \item \textbf{Dostęp do artefaktów} — bezpośredni wgląd w wygenerowane pliki konfiguracyjne oraz raporty z walidacji.
\end{itemize}

\subsubsection{Model scoringowy}

Na podstawie wyników walidacji z potoku testowego, moduł oceny wylicza wyniki cząstkowe dla plików Docker oraz manifestów Kubernetes przy użyciu modelu agregacji ocen opartego na wagach fazowych.

Model przypisuje różne wagi ważności poszczególnym fazom walidacji — fazy krytyczne dla funkcjonalności (składnia, budowa obrazu, aplikowanie do klastra) mają wyższą wagę (40\% każda) niż weryfikacja dobrych praktyk przez lintery (20\%).

\textbf{Wagi faz Docker:}
\begin{itemize}
    \item \textit{docker\_syntax} — 40\%
    \item \textit{docker\_build} — 40\%
    \item \textit{docker\_linters} — 20\%
\end{itemize}

\textbf{Wagi faz Kubernetes:}
\begin{itemize}
    \item \textit{k8s\_syntax} — 40\%
    \item \textit{kubernetes\_apply} — 40\%
    \item \textit{k8s\_linters} — 20\%
\end{itemize}

Każde wykryte zagadnienie pomniejsza wynik fazy o wartość zależną od jego powagi:
\begin{itemize}
    \item \textit{ERROR} — odejmuje 15 punktów,
    \item \textit{WARNING} — odejmuje 10 punktów,
    \item \textit{INFO} — nie wpływa na wynik (jedynie sugestie pomocnicze).
\end{itemize}

Bazowy wynik każdej fazy wynosi 100 punktów. Wynik komponentu (Docker lub Kubernetes) stanowi średnią ważoną wyników jego faz, a wynik całościowy (\textit{overall\_score}) agreguje komponenty według wag: Docker (35\%), Kubernetes (40\%) oraz środowisko uruchomieniowe (25\%).

\textbf{Ważne:} Wagi nie są normalizowane — jeśli dany komponent nie został wykonany (np. brak testów runtime), jest traktowany jako 0 punktów, a nie pomijany w obliczeniach. Oznacza to, że aby uzyskać maksymalny wynik, wszystkie komponenty muszą zostać przejściowo przetestowane.

\textbf{Dodatkowe reguły kary za krytyczne błędy:}
\begin{itemize}
    \item Jeśli budowa obrazu Docker zakończy się niepowodzeniem (faza \textit{docker\_build} zawiera błędy ERROR), wszystkie wyniki faz Docker (składnia, linters, budowa) zostają zerowane, co odzwierciedla fakt, że niezdolność do zbudowania obrazu uniemożliwia dalsze użycie konfiguracji.
    \item Jeśli aplikowanie manifestów Kubernetes do klastra zakończy się niepowodzeniem (faza \textit{kubernetes\_apply} zawiera błędy ERROR), wszystkie wyniki faz Kubernetes oraz wynik środowiska uruchomieniowego (\textit{runtime\_score}) zostają wyzerowane, ponieważ aplikacja nie może działać bez poprawnego wdrożenia.
\end{itemize}

Model zapewnia powtarzalność poprzez deterministyczne wagi oraz konsekwentne przypisywanie priorytetów błędom składniowym i funkcjonalnym nad sugestiami stylistycznymi.

\subsubsection{Rejestrowane metryki}

Przebiegi eksperymentów rejestrują metryki w dwóch formatach:

\bigskip
\noindent
\textbf{Plik summary.csv} (agregacja dla wszystkich przebiegów):
\begin{itemize}
    \item \textbf{Metadata}: experiment, timestamp, repo\_url, repo\_name, model\_provider, model\_name, model\_label, temperature, seed, prompt\_id, repetition,
    \item \textbf{Generation metrics}: generation\_success, generation\_time, tool\_calls, tokens\_used, input\_tokens, output\_tokens,
    \item \textbf{Validation metrics}: dockerfile\_syntax\_valid, k8s\_syntax\_valid, build\_success, runtime\_success,
    \item \textbf{Scoring metrics}: overall\_score, dockerfile\_score, k8s\_score, runtime\_score,
    \item \textbf{H3 Warstwa 1}: error\_count, warning\_count, info\_count, has\_errors, is\_clean,
    \item \textbf{H3 Warstwa 2}: llm\_security\_score, llm\_completeness\_score, llm\_best\_practices\_score, llm\_overall\_score,
    \item \textbf{H3 Warstwa 3}: human\_functionality\_score, human\_security\_score, human\_production\_score, human\_quality\_score, human\_overall\_score.
\end{itemize}

\bigskip
\noindent
\textbf{Pliki JSON} (szczegółowe raporty per przebieg):
\begin{itemize}
    \item Pełna lista \texttt{validation\_issues} (file\_path, line\_number, severity, message, rule\_id),
    \item \texttt{scoring\_breakdown} — szczegółowe wyniki per faza,
    \item \texttt{llm\_judge\_results} — feedback od LLM Judge,
    \item \texttt{docker\_build\_metrics} — czasy budowania, rozmiary obrazów.
\end{itemize}

Wszystkie interakcje agenta z modelem językowym są automatycznie śledzone przez platformę LangSmith, która umożliwia inspekcję pełnych ścieżek wywołań narzędzi, przesłanych promptów oraz odpowiedzi modelu. Jest to szczególnie przydatne dla jakościowej analizy H4 (deterministyczność) oraz H5 (adversarial testing).

% \clearpage % Rozdziały zaczynamy od nowej strony.
\section{Analiza porównawcza modeli}

TODO



\textbf{Przykładowe obserwacje i napotkane problemy:}
Omówienie ogólnego zachowania modeli LLM podczas generacji.
Analiza typowych problemów napotkanych podczas eksperymentów, takich jak:
\begin{itemize}
    \item Ograniczenia długości kontekstu (tokenów),
    \item Błędy w rozumieniu struktury repozytorium,
    \item Wpływ długości i precyzji prompta na jakość generacji,
    \item Deterministyczność generowanych plików,
    \item Niekonsekwencje w nazwach plików / usług.
\end{itemize}

\textbf{Wyniki testów:}

\textbf{Analiza porównawcza modeli?:}

\textbf{Wnioski?:}
% \input{tex/6-bezpieczenstwo}
% \input{tex/7-projekt-systemu}
% \input{tex/8-implementacja}
% \input{tex/9-wnioski-i-dalsze-kierunki}

%---------------
% Bibliografia
%---------------
\cleardoublepage % Zaczynamy od nieparzystej strony
\printbibliography
\clearpage

% Wykaz symboli i skrótów.
% Pamiętaj, żeby posortować symbole alfabetycznie
% we własnym zakresie. Makro \acronymlist
% generuje właściwy tytuł sekcji, w zależności od języka.
% Makro \acronym dodaje skrót/symbol do listy,
% zapewniając podstawowe formatowanie.

\acronymlist
\acronym{LLM}{ang. \emph{Large Language Model}} - duży model językowy
\acronym{IaC}{ang. \emph{Infrastructure as Code} – infrastruktura jako kod}
\acronym{K8s}{ang. \emph{Kubernetes} – system orkiestracji kontenerów}
\acronym{PaaS}{ang. \emph{Platform as a Service} – platforma jako usługa}
\acronym{CI/CD}{ang. \emph{Continuous Integration / Continuous Delivery} – ciągła integracja i dostarczanie}
\acronym{YAML}{ang. \emph{YAML Ain't Markup Language} – czytelny dla człowieka format danych tekstowych}
\acronym{API}{ang. \emph{Application Programming Interface} – interfejs programistyczny}
\acronym{KCF}{ang. \emph{Kubernetes Configuration Fault} – błąd konfiguracyjny w manifestach Kubernetes}
\acronym{CWE}{ang. \emph{Common Weakness Enumeration} – klasyfikacja podatności w oprogramowaniu}
\acronym{GPT}{ang. \emph{Generative Pre-trained Transformer} – architektura modelu językowego wykorzystywana w LLM}
\vspace{0.8cm}

%--------------------------------------
% Spisy: rysunków, tabel, załączników
%--------------------------------------
\pagestyle{plain}

\listoffigurestoc    % Spis rysunków.
\vspace{1cm}         % vertical space
\listoftablestoc     % Spis tabel.
\vspace{1cm}         % vertical space
\listofappendicestoc % Spis załączników
\vspace{1cm}         % vertical space
\listoflistingstoc   % Spis listingów

%-------------
% Załączniki
%-------------

% Obrazki i tabele w załącznikach nie trafiają do spisów

% Używając powyższych spisów jako szablonu,
% możesz dodać również swój własny wykaz,
% np. spis algorytmów.

\end{document} % Dobranoc.
