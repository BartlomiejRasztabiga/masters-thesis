%-----------------------------------------------
%  Engineer's & Master's Thesis Template
%  Copyleft by Artur M. Brodzki & Piotr Woźniak
%  Warsaw University of Technology, 2019-2022
%-----------------------------------------------

\documentclass[
    bindingoffset=5mm,  % Binding offset
    footnoteindent=3mm, % Footnote indent
    hyphenation=true    % Hyphenation turn on/off
]{src/wut-thesis}

\graphicspath{{tex/img/}} % Katalog z obrazkami.
\addbibresource{bibliografia.bib} % Plik .bib z bibliografią

%-------------------------------------------------------------
% Wybór wydziału:
%  \facultyeiti: Wydział Elektroniki i Technik Informacyjnych
%  \facultymeil: Wydział Mechaniczny Energetyki i Lotnictwa
% --
% Rodzaj pracy: \EngineerThesis, \MasterThesis, \PPMGR
% --
% Wybór języka: \langpol, \langeng
%-------------------------------------------------------------
\facultyeiti    % Wydział Elektroniki i Technik Informacyjnych
\MasterThesis % Praca inżynierska
\langpol % Praca w języku polskim

\begin{document}

%------------------
% Strona tytułowa
%------------------
\instytut{Informatyki}
\kierunek{Informatyka}
\specjalnosc{Inteligentne Systemy}
\title{
    Zastosowanie dużych modeli językowych (LLM) \\ 
    do generowania konfiguracji Docker i Kubernetes
}
% Title in English for English theses
% In English theses, you may remove this command
\engtitle{
    Application of large language models (LLMs) \\
    for generating Docker and Kubernetes configurations
}
% Title in Polish for English theses
% Use it only in English theses
\poltitle{
    Zastosowanie dużych modeli językowych (LLM) \\ 
    do generowania konfiguracji Docker i Kubernetes
}
\author{Bartłomiej Rasztabiga}
\album{304117}
\promotor{dr inż. Mateusz Modrzejewski}
\date{\the\year}
\maketitle

%-------------------------------------
% Streszczenie po polsku dla \langpol
% English abstract if \langeng is set
%-------------------------------------
\cleardoublepage % Zaczynamy od nieparzystej strony
% TODO uncomment
\abstract \lipsum[1-3]
\keywords XXX, XXX, XXX

%----------------------------------------
% Streszczenie po angielsku dla \langpol
% Polish abstract if \langeng is set
%----------------------------------------
\clearpage
% TODO uncomment
\secondabstract \kant[1-3]
\secondkeywords XXX, XXX, XXX

\pagestyle{plain}

%--------------
% Spis treści
%--------------
\cleardoublepage % Zaczynamy od nieparzystej strony
\tableofcontents

%------------
% Rozdziały
%------------
\cleardoublepage % Zaczynamy od nieparzystej strony
\pagestyle{headings}

% TODO rozdziały magisterki
% 
% TODO

\clearpage % Rozdziały zaczynamy od nowej strony.
\section{Wprowadzenie}

W ciągu ostatnich lat obserwujemy bezprecedensowy rozwój dużych modeli językowych (LLM) oraz równoległy wzrost znaczenia technologii konteneryzacji i orkiestracji. Modele językowe takie jak GPT, LLaMA, Falcon czy Claude wykazują zdolność do generowania złożonego kodu, w tym konfiguracji infrastruktury, podczas gdy Docker i Kubernetes ugruntowały swoją pozycję jako standard w obszarze wdrażania aplikacji. Niniejsza praca bada potencjał automatyzacji generowania konfiguracji Docker i Kubernetes z wykorzystaniem LLM, ze szczególnym uwzględnieniem aspektów poprawności i bezpieczeństwa generowanych plików konfiguracyjnych.

\subsection{Cel i zakres pracy}

Celem niniejszej pracy magisterskiej jest analiza możliwości zastosowania dużych modeli językowych (LLM) do automatycznego generowania konfiguracji typu Infrastructure as Code (IaC), ze szczególnym uwzględnieniem plików Dockerfile oraz manifestów Kubernetes. Praca skupia się na wykorzystaniu LLM w kontekście platformy jako usługi (PaaS), gdzie efektywne tworzenie i utrzymanie kontenerów oraz ich orkiestracji ma kluczowe znaczenie. Zakres obejmuje zbadanie metod generowania plików Dockerfile definiujących obrazy kontenerów oraz manifestów Kubernetes opisujących wdrożenie tych kontenerów, a także oceny jakości, bezpieczeństwa i zgodności takich wygenerowanych konfiguracji z wymaganiami systemowymi.

Głównym celem pracy jest analiza możliwości zastosowania dużych modeli językowych do generowania konfiguracji Dockerfile i Kubernetes w ramach platformy jako usługi (PaaS), która umożliwia budowanie i wdrażanie aplikacji w klastrze Kubernetes. Praca obejmuje następujące zagadnienia:

\subsection{Motywacja}

Tematyka pracy jest istotna z uwagi na rosnącą rolę metodyk DevOps i automatyzacji zarządzania infrastrukturą. Infrastructure as Code zyskuje na popularności, ponieważ umożliwia spójną, powtarzalną konfigurację środowisk i redukcję błędów ludzkich. Ręczne tworzenie skryptów IaC, zwłaszcza dla złożonych środowisk chmurowych, bywa jednak czasochłonne i wymaga specjalistycznej wiedzy. Wraz z rozwojem dużych modeli językowych pojawiła się możliwość ich wykorzystania do generowania kodu konfiguracyjnego na podstawie opisu w języku naturalnym. Użycie LLM może obniżyć barierę wejścia dla deweloperów mniej doświadczonych w technologiach chmurowych (np. Kubernetes) poprzez automatyczne tłumaczenie wysokopoziomowych specyfikacji na deklaratywne pliki konfiguracji. Ma to znaczenie praktyczne na platformach PaaS, gdzie skrócenie czasu wdrożenia aplikacji i eliminacja błędów konfiguracji przekładają się na większą wydajność i niezawodność usług. Z drugiej strony, automatyzacja generowania IaC rodzi pytania o poprawność i bezpieczeństwo tych konfiguracji. Modele językowe mogą popełniać błędy lub tzw. halucynacje, generując nieistniejące lub niezalecane elementy konfiguracji. Istotne jest zatem zbadanie, na ile można zaufać LLM w kontekście krytycznych elementów infrastruktury oraz jak zapewnić zgodność wygenerowanych konfiguracji z najlepszymi praktykami bezpieczeństwa (np. czy model nie pominie istotnych zabezpieczeń, jak autoryzacja, czy nie wygeneruje podatnych ustawień). Zainteresowanie połączeniem LLM i DevOps wynika także z potencjału ułatwienia pracy inżynierów – dzięki LLM mogą oni szybciej uzyskiwać wstępne wersje konfiguracji i skupić się na ich dostrojeniu, zamiast pisać je od podstaw.

\subsection{Struktura pracy}

Praca składa się z dziewięciu rozdziałów, które wspólnie odzwierciedlają pełny cykl badawczy — od identyfikacji problemu, przez analizę literatury i technologii, aż po eksperymenty, analizę wyników, bezpieczeństwo i implementację kompletnego systemu.

Rozdział 1 – Wprowadzenie: przedstawia temat pracy, jej cele, motywację oraz układ całej pracy.
Rozdział 2 – Przegląd literatury: zawiera analizę aktualnych badań nad wykorzystaniem dużych modeli językowych w kontekście generowania Infrastructure as Code (IaC), w tym konfiguracji Docker i Kubernetes. Wskazuje także istniejące luki, np. brak analiz bezpieczeństwa i brak pełnych pipeline’ów DevOps.
Rozdział 3 – Przegląd technologii i narzędzi: opisuje wykorzystane modele językowe (GPT, Claude, LLaMA, Mistral, DeepSeek) oraz narzędzia infrastrukturalne (Docker, Kubernetes, Lens, kind). Rozdział porównuje też podejścia API vs open-source, opisuje środowisko uruchomieniowe i technologie wspierające (FastAPI, GitPython itp.).
Rozdział 4 – Eksperymenty: opisuje zaplanowane przypadki testowe (np. aplikacje jedno- i wielosystemowe), strategie promptowania, proces generowania i testowania konfiguracji oraz przykładowe wyniki i napotkane problemy (np. ograniczenia tokenów).
Rozdział 5 – Analiza porównawcza modeli LLM: przedstawia kryteria oceny (poprawność, bezpieczeństwo, deterministyczność, odporność na manipulacje, wydajność), wyniki testów i wnioski na temat jakości działania poszczególnych modeli.
Rozdział 6 – Bezpieczeństwo konfiguracji: analizuje zagrożenia związane z automatycznym generowaniem konfiguracji, techniki ataku na LLM (prompt injection, jailbreaking) oraz metody oceny i wzmacniania bezpieczeństwa wygenerowanego kodu.
Rozdział 7 – Projekt systemu PaaS: prezentuje architekturę zaprojektowanego systemu do automatycznego generowania, budowania i wdrażania aplikacji. Opisuje także problemy projektowe oraz proces działania od repozytorium do uruchomienia aplikacji w Kubernetes.
Rozdział 8 – Implementacja i wdrożenie prototypu: zawiera szczegółowy opis realizacji systemu PaaS, użytych technologii, przykładowych wdrożeń oraz napotkanych problemów podczas integracji i testów.
Rozdział 9 – Wnioski i dalsze kierunki rozwoju: podsumowuje uzyskane wyniki, odpowiada na pytania badawcze, ocenia przydatność LLM w środowiskach DevOps oraz wskazuje możliwe ścieżki rozwoju — takie jak wsparcie mikroserwisów, walidacja semantyczna YAML czy integracja z CI/CD.
\clearpage % Rozdziały zaczynamy od nowej strony.
\section{Przegląd literatury}

Niedawny rozwój możliwości generowania kodu dzięki zastosowaniu dużych modeli językowych dotyczył głównie języków programowania ogólnego przeznaczenia. Języki specyficzne dla dziedzin, takie jak te wykorzystywane w automatyzacji IT, otrzymały znacznie mniej uwagi, mimo że angażują wielu aktywnych deweloperów i stanowią istotny element współczesnych platform chmurowych. \cite{pujar_invited_2023}

Literatura dotycząca wykorzystania dużych modeli językowych (LLM) do generowania manifestów Docker i Kubernetes jest ograniczona. Istnieją jednak badania, które zajmują się różnymi aspektami tej tematyki, szczególnie w kontekście automatyzacji infrastruktury jako kodu (IaC). Żadna z do tej pory przeanalizowanych, istniejących prac nie koncentruje się jednak na pełnym wykorzystaniu repozytorium aplikacji jako wejścia do modeli LLM, a także na dokładnym analizowaniu problemów bezpieczeństwa takich konfiguracji. W niniejszym przeglądzie przedstawiane są wybrane badania oraz wskazywane są sposoby, w jakie niniejsza praca rozszerza i ulepsza istniejące podejścia.

Praca Malula i in. wprowadza system GenKubeSec \cite{malul_genkubesec_2024}, który wykorzystuje modele LLM do wykrywania i naprawiania błędnych konfiguracji Kubernetes (KCF). System ten jest innowacyjny, ponieważ dostarcza pełne rozwiązanie: od wykrywania problemów, przez ich lokalizację, aż po ich naprawę. Kluczową zaletą GenKubeSec jest minimalizacja ryzyka związanego z bezpieczeństwem poprzez wykorzystanie lokalnych modeli LLM zamiast komercyjnych API. Praca poszerza te badania, badając nie tylko wykrywanie błędnych konfiguracji, ale także aspekt ich deterministyczności i odporności na manipulacje danymi wejściowymi.

Ueno i Uchiumi proponują benchmark oceniający jakość manifestów Kubernetes generowanych przez modele LLM na podstawie specyfikacji Docker Compose \cite{ueno_migrating_2024}. Wyniki pokazują, że generowane manifesty są często poprawne, ale brakuje w nich spójności i komentarzy poprawiających czytelność. W pracy uwzględniana jest ta krytyka, koncentrując się na generowaniu manifestów, które są zarówno funkcjonalne, jak i czytelne dla ludzi.

Kratzke i Drews badali możliwości standardowych modeli LLM w generowaniu manifestów Kubernetes przy użyciu zaawansowanych technik inżynierii promptów \cite{kratzke_dont_2024}. Badanie pokazuje, że efektywne projektowanie promptów może znacząco poprawić jakość generowanych manifestów. Praca rozwija ten kierunek, analizując nie tylko jakość generowanych konfiguracji, ale również ich podatność na ataki.

Pujar i in. skoncentrowali się na generowaniu konfiguracji YAML dla Ansible za pomocą modeli LLM \cite{pujar_invited_2023}. Choć praca skupia się na inżynierii promptów i budowaniu dedykowanych datasetów, brak w niej analizy problemów bezpieczeństwa czy deterministyczności. Niniejsza praca rozszerza ten kontekst na Kubernetes i Docker, uwzględniając dodatkowe aspekty, takie jak jailbreaking modeli.

Lanciano i in. zaproponowali podejście do analizy jakości manifestów Kubernetes z wykorzystaniem LLM \cite{lanciano_analyzing_2023}. System dostarcza rekomendacje dotyczące jakości kodu i pomaga mniej doświadczonym deweloperom w stosowaniu najlepszych praktyk. Praca rozwija tę ideę, koncentrując się na automatyzacji całego procesu, od generowania po wdrażanie.

Podjęte w literaturze próby wykorzystania LLM w kontekście IaC i Kubernetes koncentrują się głównie na generowaniu kodu i podstawowej analizie. Niniejsza praca poszerza ten obszar, skupiając się m.in. na:

\begin{itemize}
    \item Wykorzystaniu repozytoriów kodu jako źródła wejściowego dla LLM,
    \item Analizie deterministyczności i bezpieczeństwa generowanych konfiguracji,
    \item Zastosowaniu metod oceny odporności na manipulacje danymi wejściowymi,
    \item Porównaniu topowych modeli pod kątem możliwej złożoności generowanych konfiguracji (np. obsługa wielu kontenerów, sieci czy woluminów).
\end{itemize}
% \clearpage % Rozdziały zaczynamy od nowej strony.

\section{Przegląd technologii i narzędzi}

Celem tego rozdziału jest przedstawienie najważniejszych technologii i narzędzi, stanowiących podstawę niniejszego badania. Szczególny nacisk położony zostanie na duże modele językowe (LLM), opis ich architektur oraz ich zastosowanie w generowaniu konfiguracji Infrastructure as Code (IaC). Ponadto, omówione zostaną istotne narzędzia DevOps, takie jak Docker i Kubernetes, niezbędne w zarządzaniu środowiskami uruchomieniowymi, oraz narzędzia służące do walidacji i oceny jakości generowanego kodu IaC. Rozdział zawiera także opis środowiska eksperymentalnego oraz wykorzystywanych technologii wspierających automatyzację i monitorowanie eksperymentów.

\subsection{Modele językowe wykorzystywane w badaniu}

Duże modele językowe (LLM) pełnią istotną rolę w prezentowanych eksperymentach, służąc jako generatory konfiguracji IaC. Ich zdolność do przetwarzania języka naturalnego oraz generowania poprawnego kodu czyni je ważnym elementem automatyzacji procesów DevOps. W tym podrozdziale przedstawiono charakterystykę wybranych modeli, opis ich architektur oraz praktyczne aspekty związane z ich wykorzystaniem.

Rynek dużych modeli językowych rozwija się dynamicznie, oferując zarówno modele komercyjne, dostępne przez API, jak i modele open-source, możliwe do uruchomienia lokalnie lub na własnej infrastrukturze. Decyzja o wyborze konkretnego rodzaju modelu często zależy od dostępności zasobów sprzętowych (zwłaszcza GPU), wymagań licencyjnych, kwestii związanych z prywatnością danych oraz możliwości dostosowania modeli do specyficznych zastosowań.

Modele komercyjne dostępne przez API zazwyczaj cechują się wysoką wydajnością, zaawansowanymi możliwościami oraz niezawodnością, dzięki wsparciu dużych firm technologicznych takich jak OpenAI, Anthropic czy Google. Modele te trenowane są na szerokich i zróżnicowanych zbiorach danych, co przekłada się na ich zdolność generowania wysokiej jakości tekstu oraz kodu. Ich użytkowanie wiąże się z kosztami dostępu do API oraz koniecznością przesyłania danych do zewnętrznych serwisów, co może być problematyczne w przypadku danych wrażliwych.

Wybrane do badania modele komercyjne to:

\begin{itemize}
	\item \textbf{OpenAI GPT-4.1}, \textbf{GPT-4o} oraz \textbf{O3}: Najnowsze flagowe modele OpenAI. GPT-4.1 charakteryzuje się udoskonalonymi możliwościami generowania i rozumienia złożonego kodu oraz zoptymalizowanymi funkcjami integracji z narzędziami (tool calling). GPT-4o, model multimodalny, oferuje wysoką wydajność, elastyczność oraz wszechstronne możliwości interakcji. O3 zapewnia optymalne połączenie wydajności, szybkości oraz efektywności kosztowej.
	\item \textbf{Anthropic Claude Opus 4}, \textbf{Sonnet 4} oraz \textbf{Haiku 3.5}: Zaawansowane modele Anthropic, cenione za bezpieczeństwo, zgodność z najlepszymi praktykami AI Safety oraz obsługę długich kontekstów i precyzyjnych instrukcji. Opus 4 to najbardziej zaawansowana wersja, Sonnet 4 równoważy wydajność z szybkością, a Haiku 3.5 skupia się na efektywności kosztowej i szybkości działania.
	\item \textbf{Google Gemini 2.5 Pro} i \textbf{2.5 Flash}: Modele Google z rozszerzonym oknem kontekstu, dedykowane do analizy obszernych repozytoriów kodu i dokumentacji. Wersja Pro oferuje wyjątkową pojemność kontekstu i wysoką wydajność, podczas gdy wersja Flash jest zoptymalizowana pod kątem szybszego działania przy niższych kosztach operacyjnych.
\end{itemize}

Modele open-source stanowią atrakcyjną alternatywę, oferując pełną kontrolę, możliwość uruchomienia na własnej infrastrukturze oraz brak dodatkowych opłat za każde zapytanie. Ich wydajność zależy przede wszystkim od dostępności zasobów sprzętowych, a aktywny rozwój społeczności przyczynia się do ciągłej optymalizacji i pojawiania się coraz bardziej zaawansowanych wersji.

Wybrane do badania modele open-source to:

\begin{itemize}
	\item \textbf{Mistral Medium} oraz \textbf{Codestral}: Modele od firmy Mistral AI, zoptymalizowane pod kątem efektywnego generowania kodu i ogólnego rozumienia kontekstu technicznego. Codestral specjalizuje się w obsłudze złożonych zadań programistycznych, oferując wysoką efektywność przy umiarkowanych wymaganiach sprzętowych.
	\item \textbf{Meta Llama 4 Maverick} oraz \textbf{Llama 4 Scout}: Najnowsze modele open-source od Meta, które stanowią istotny punkt odniesienia w dziedzinie rozwoju sztucznej inteligencji. Maverick wyróżnia się wysokimi osiągami i zaawansowanymi funkcjami, natomiast Scout zapewnia dobrą wydajność przy bardziej ograniczonych zasobach sprzętowych, umożliwiając łatwiejsze wdrożenie w praktycznych zastosowaniach.
	\item \textbf{DeepSeek V3}: Model typu Mixture-of-Experts (MoE), łączący wysoką jakość generowania kodu z dużą efektywnością obliczeniową. DeepSeek V3 wyróżnia się zdolnością do dynamicznego aktywowania tylko części parametrów modelu przy każdym zapytaniu, co pozwala osiągnąć korzystny kompromis między jakością a kosztami obliczeń.
\end{itemize}

Wszystkie wymienione modele reprezentują aktualny szczyt osiągnięć w dziedzinie dużych modeli językowych (LLM). Wybrane zostały przede wszystkim ze względu na ich potwierdzone zdolności do generowania wysokiej jakości kodu oraz wsparcie dla funkcji „tool calling”, które umożliwiają bezpośrednią integrację modeli z zewnętrznymi narzędziami lub usługami poprzez dedykowane mechanizmy API bądź platformy integracyjne, takie jak LangChain czy LangGraph. Funkcje te będą odgrywać znaczącą rolę w proponowanym w tej pracy podejściu opartym na pętli sprzężenia zwrotnego w procesie generowania konfiguracji Infrastructure as Code (IaC).

\subsubsection{Architektury i charakterystyka}

Większość współczesnych dużych modeli językowych bazuje na architekturze transformera \cite{vaswani_attention_2023}, wykorzystując mechanizm uwagi (attention mechanism) do efektywnego przetwarzania sekwencji danych. Architektury te, pomimo wspólnych założeń, różnią się między sobą liczbą parametrów, rozmiarem okna kontekstowego oraz specyficznymi optymalizacjami, które wpływają na ich zdolność do generowania kodu i rozumienia złożonych instrukcji. Coraz większe znaczenie zyskują również alternatywne podejścia architektoniczne, takie jak Mixture-of-Experts (MoE), w których przy każdym zapytaniu aktywowane są tylko wybrane części modelu, co pozwala na zwiększenie efektywności obliczeniowej bez pogarszania jakości generowanych wyników.

\begin{itemize}
	\item \textbf{Liczba parametrów:} Liczba parametrów (często liczona w miliardach) jest wskaźnikiem skali modelu i jego zdolności do uczenia się złożonych wzorców z danych. Modele z większą liczbą parametrów zazwyczaj wykazują lepszą wydajność w szerokim zakresie zadań, jednak ich uruchomienie i obsługa wymagają znacznie większych zasobów obliczeniowych. Należy jednak zaznaczyć, że sama liczba parametrów nie jest jedynym wyznacznikiem jakości; kluczowe znaczenie ma również jakość danych treningowych, architektura (np. Mixture-of-Experts) oraz proces dostrajania modelu.
	\item \textbf{Okno kontekstowe:} Okno kontekstowe odnosi się do maksymalnej długości sekwencji tokenów (słów, znaków, fragmentów kodu), którą model może przetworzyć i wykorzystać do wygenerowania odpowiedzi. Jest to szczególnie istotne w kontekście generowania konfiguracji IaC z repozytoriów kodu. Duże okno kontekstowe pozwala na dostarczenie modelowi całych plików źródłowych aplikacji, fragmentów dokumentacji, wielu powiązanych ze sobą promptów, a nawet wyników walidacji z zewnętrznych narzędzi. To umożliwia agentowi LLM holistyczne zrozumienie projektu i kontekstu, co przekłada się na wyższą jakość i trafność generowanych konfiguracji. Modele takie jak Google Gemini 2.5 Pro wyróżniają się wyjątkowo dużym oknem kontekstowym (do miliona tokenów), co jest znaczącą przewagą w zadaniach wymagających głębokiej analizy kodu i dokumentacji.
	\item \textbf{Specjalizacje i optymalizacje:} Niektóre modele, takie jak Mistral Codestral, są specjalizowane lub fine-tuninguowane w kierunku generowania kodu. Oznacza to, że są one trenowane na dużych zbiorach danych zawierających kod programistyczny, co poprawia ich zdolność do generowania syntaktycznie poprawnego i semantycznie trafnego kodu. Architektury takie jak Mixture-of-Experts (MoE) stosowane w Mixtral 8x7B pozwalają na efektywne skalowanie modeli, aktywując tylko część ekspertów dla danego zapytania, co optymalizuje zużycie zasobów przy zachowaniu wysokiej jakości odpowiedzi.
\end{itemize}

Poniższa tabela \ref{tab:llm-characteristics} przedstawia ogólne charakterystyki wybranych modeli LLM, które będą wykorzystywane w badaniu. Okno kontekstu podano w liczbie tokenów, a liczba parametrów — w miliardach, jeśli została opublikowana.

\begin{table}[!h] \centering
\caption{Charakterystyka wybranych modeli LLM}
\label{tab:llm-characteristic}
\begin{tabular}{| c | c | c | c |} \hline
\textbf{Dostawca} & \textbf{Model} & \textbf{Parametry (mld)} & \textbf{Okno kontekstu (tokeny)} \\ \hline\hline
OpenAI & GPT-4.1 & - & 1M \\ \hline
OpenAI & GPT-4o & - & 128k \\ \hline
OpenAI & O3 & - & 200k \\ \hline
Anthropic & Claude Opus 4 & - & 200k \\ \hline
Anthropic & Claude Sonnet 4 & - & 200k \\ \hline
Anthropic & Claude Haiku 3.5 & - & 200k \\ \hline
Google & Gemini 2.5 Pro & - & 1M \\ \hline
Google & Gemini 2.5 Flash & - & 1M \\ \hline
Mistral AI & Mistral Medium & 12–20 & 32k \\ \hline
Mistral AI & Codestral & 22 & 32k–64k \\ \hline
Meta & Llama 4 Maverick & 70 & 128k \\ \hline
Meta & Llama 4 Scout & 8 & 128k \\ \hline
DeepSeek & DeepSeek V3 & 236 & 128k \\ \hline
\end{tabular}
\end{table}

TODO tu skonczylem

Cel: Przedstawić modele LLM używane w eksperymencie, ich charakterystyki, sposób użycia i ograniczenia.

Proponowana zawartość:
	•	Rodzaje modeli:
	•	    Podział na modele open-source (np. Mistral, LLaMA, Nous Hermes) vs. modele komercyjne dostępne przez API (np. GPT-4, Claude).
	•		Warto podkreślić, że wybór między open-source a komercyjnymi często zależy od dostępności zasobów (GPU), wymagań licencyjnych i elastyczności.
	•	Architektury i charakterystyki:
	•	    Typowe parametry: liczba parametrów, specjalizacje, wersje, okno kontekstu (dlaczego wazne przy repozytoriach kodu)
	•		Warto dodać, że "liczba parametrów" nie zawsze jest jedynym wyznacznikiem jakości, ale jest istotnym parametrem. "Okna kontekstu" – jak najbardziej warto wyjaśnić, dlaczego jest ważne przy repozytoriach kodu (np. umożliwia dostarczanie całych plików, fragmentów dokumentacji, wielu powiązanych promptów jednocześnie).
    •   Wybrane modele do badania:
    •       Krótki opis które i dlaczego z parametrami jakimiś basic
	• 		Informacja, które modele są używane w badaniu i dlaczego (ograniczenia przez tool calling w langgraph).
	•		To jest bardzo ważna informacja! Musisz to jasno przedstawić. Jeśli LangGraph narzuca ograniczenia na wybór modeli (np. tylko te z dobrze zaimplementowanym tool calling), to jest to istotny czynnik wpływający na zakres badań.
	•	Sposób użycia:
	•	    Poprzez OpenRouter, bezpośrednie API, LangChain / LangGraph.
	•	    Sposób integracji z agentami.
	•	Tryby promptowania:
	•	    Zero-shot, few-shot, chain-of-thought, agent-based.
	•	    Porównanie tych podejść i wybór odpowiednich metod do problemu.
	•		Warto wyjaśnić, dlaczego wybrane metody (np. agent-based) są najbardziej odpowiednie dla generowania konfiguracji IaC.
	•	Ograniczenia i praktyczne aspekty:
	•	    Licencje, limity API, ograniczona liczba modeli.
	•	    Context window – jak wpływa na eksperymenty z kodem.
	•	    Brak fine-tuningu – tylko inference.
	•	    Kontrola nad danymi – brak możliwości dostosowania kodu źródłowego w zamkniętych modelach.



\subsection{Narzędzia DevOps: Docker i Kubernetes}

Cel: Opisać narzędzia wykorzystywane do zarządzania środowiskiem uruchomieniowym i ich wpływ na eksperymenty.

Proponowana zawartość:
	•	Docker:
	•	    Do czego jest wykorzystywany: uruchamianie lokalnych usług, środowiska testowe i deweloperskie, konteneryzacja narzędzi, konteneryzacja produkcyjnych aplikacji.
	•	    Jak mozna popelnic bledy
	•		Warto pokazać typowe błędy, które LLM może popełnić (np. użycie ADD zamiast COPY, brak WORKDIR, użycie latest tagu, brak usuwania zależności po buildzie). To od razu pokazuje, dlaczego ocena bezpieczeństwa i optymalizacji jest tak ważna.
	•	Kubernetes:
	•	    Używany z minimalnym zakresem (np. przez Kind, K3s, minikube).
	•	    Rola w testowaniu konfiguracji i IaC.
	•		Warto podkreślić, że Kubernetes służy jako środowisko do walidacji runtime wygenerowanych konfiguracji, a nie tylko do ich tworzenia.
	•	    Jak można popełnić błędy przy generowaniu YAML (niedopasowane zasoby, problemy z dostępnością, błędne konfiguracje).
	•		Dodaj przykłady: brak liveness/readiness probes, źle skonfigurowane serwisy (np. ClusterIP zamiast NodePort/LoadBalancer), brak PersistentVolumeClaim dla baz danych, zbyt wysokie/niskie limity zasobów, brak kontekstu bezpieczeństwa.
	•	Dobre i złe praktyki:
	•	    Przykłady z testów (np. źle ustawione limity zasobów, niepoprawne labele).
	•   Dodac tu Kaniko czy w technologiach wspierajacych?
	•	Chyba nie bedzie wykorzystywane jeszcze tutaj, a jedynie w systemie PaaS



\subsection{Narzędzia do oceny jakości IaC}

Cel: Pokazać narzędzia do analizy i walidacji konfiguracji IaC oraz ich funkcjonalności.

Proponowana zawartość:
	•	Lista narzędzi:
	•	    checkov, kube-linter, kube-score, kubeval, terrascan, cloudeval-yaml, iac-eval.
	•	Opis funkcji i zasad działania:
	•	    Statyczna analiza, reguły zgodności, walidacja schematów.
	•	Przykłady outputów:
	•	    Co zgłaszają narzędzia, jak to interpretować.
	•	    Zestawienie wyników na tych samych plikach (jeśli masz dane).
	•		Pokazać kilka błędów mniejszych i większych, podatności itd



\subsection{Środowisko eksperymentalne}

Cel: Przedstawić infrastrukturę testową i środowisko, w którym uruchamiane będą eksperymenty.

Proponowana zawartość:
	•	Charakterystyka środowiska:
	•	    Lokalna maszyna, Docker Desktop, Kubectl, Kind, Python, LangChain, LangGraph, OpenRouter?.
	•		Jasno określ, co jest na maszynie lokalnej, a co jest usługą zewnętrzną (OpenRouter to API gateway).
	•	Parametry techniczne:
	•	    RAM, CPU, ewentualna obecność GPU, ograniczenia związane z lokalnością - dlaczego lokalnie?
	•	Zasady testowania:
	•	    Jak wygląda testowanie przez API: limity żądań, obsługa błędów, wersje modeli.
	•	    Korzystanie z narzędzi CLI lub SDK.



\subsection{Technologie wspierające (do rozważenia jako część środowiska eksperymentalnego)}

Alternatywa 1: Zostawić jako osobny rozdział z wyraźnym celem: technologie pomocnicze ułatwiające pracę.

Alternatywa 2: Zmergować z poprzednim podrozdziałem jako „środowisko eksperymentalne i wspierające technologie”.

Zawartość:
	•	Język Python:
	•	    Używane biblioteki (np. requests, openai, langchain, pydantic, git, docker-py itd.).
	•	Narzędzia pomocnicze:
	•	    Kind, Lens, VSCode, Docker CLI, kubectl, LangSmith, LangFuse, Promptfoo, Kaniko.
	•		LangSmith, LangFuse, Promptfoo: To jest bardzo ważne! Są to narzędzia do ewaluacji i zarządzania promptami/LLM. Koniecznie je opisz krótko i wyjaśnij, jak pomogły w Twoich badaniach (np. do monitorowania interakcji z LLM, testowania promptów, porównywania ich skuteczności). To pokazuje zaawansowane podejście do inżynierii promptów.
	•	Rola:
	•	    Obsługa eksperymentów, monitorowanie, automatyzacja.



% \clearpage % Rozdziały zaczynamy od nowej strony.
\section{Opis eksperymentów}

\subsection{Cel eksperymentów}

Eksperymenty miały na celu przeprowadzenie proof of concept (PoC) dla automatycznego generowania konfiguracji Docker i Kubernetes przy użyciu dużych modeli językowych (LLM). Celem było zweryfikowanie, czy modele mogą poprawnie generować pliki konfiguracyjne na podstawie dostarczonego repozytorium kodu, a także zbadanie problemów i wyzwań, których można się spodziewać podczas pisania pracy magisterskiej.

\subsection{Proces eksperymentów}

Algorytm skryptu PoC składał się z następujących kroków:

\begin{enumerate}
    \item Klonowanie repozytorium: Repozytorium projektu było klonowane do lokalnego katalogu tymczasowego.
    \item Analiza struktury repozytorium: Struktura repozytorium była przetwarzana do formatu drzewa plików za pomocą polecenia tree.
    \item Wybór najważniejszych plików: Model LLM otrzymywał strukturę repozytorium i wybierał 5 najważniejszych plików, które mogłyby posłużyć do generowania konfiguracji.
    \item Pobranie zawartości plików: Na podstawie wybranych plików ich zawartość była odczytywana i przygotowywana do wysłania do modelu.
    \item Generowanie pliku Dockerfile: LLM generował plik Dockerfile na podstawie struktury repozytorium i zawartości najważniejszych plików.
    \item Budowanie obrazu Docker: Obraz Docker był budowany lokalnie na podstawie wygenerowanego Dockerfile.
    \item Generowanie konfiguracji Kubernetes: Model LLM generował manifest Kubernetes, zawierający definicje m.in. Deployment, Service i Ingress.
    \item Weryfikacja działania: Wygenerowany obraz Docker był przesyłany do prywatnego rejestru, a manifest Kubernetes wdrażany w klastrze Kubernetes. Działanie aplikacji było potwierdzane uruchomieniem jej na klastrze.
\end{enumerate}


\subsection{Środowisko eksperymentalne}

Testy były przeprowadzone na prywatnym projekcie studenckim z przedmiotu FO (\url(https://github.com/BartlomiejRasztabiga/FO23Z)), zawierającym bezstanową aplikację webową uruchamianą w jednym kontenerze. Najlepsze rezultaty podczas eksperymentów osiągnął model Claude 3.5 Haiku, wykazując się wysoką jakością generowanych plików oraz deterministycznym działaniem. W przeciwieństwie do innych moceli, Claude 3.5 Haiku praktycznie nie generował błędnych konfiguracji - każda inferencja kończyła się przygotowaniem poprawnej konfiguracji.

\subsection{Kod implementacji}

Kod aplikacji obsługującej eksperymenty został zaimplementowany w języku Python z użyciem opisanych wcześniej bibliotek takich jak docker, git i dotenv. Repozytorium z kodem eksperymentu jest dostępne pod adresem: \url{https://gitlab-stud.elka.pw.edu.pl/brasztab/masters-thesis/-/tree/master/poc}

\subsection{Listing prompta uywanego w eksperymentach}

\begin{verbatim}
You are a helpful assistant. You will have 3 use cases, they are listed below,
they're all connected with each other to deliver some bigger value. Use correct 
use_case, first line of the prompt will include name of a use case, e.g.
"get_important_files". Execute only the use case you're asked for. Don't include 
use case name in the response. Don't include
any formatting markers. Return only what you're asked for.

1 (get_important_files): Given repository files structure (only some part of it)
will help to identify the most important files to generate a Dockerfile to build a
valid docker image that can be run to run the app of repository and appropriate 
Kubernetes config (deployments, services, volumes, ingresses) to run it. Choose
only the most important files, required to generate configuration. Respond only 
with the file names, in the same format as provided, ignore formatting markers. 
Use that use case when keyword "get_important_files" is used.

2 (get_dockerfile): Given repository files structure (only some part of it)
and content of the most important files will help to generate a Dockerfile to build
a valid docker image that can be run to run the app of repository. Use latest base 
image versions and best practises, implement all security measures and expose all
necessary ports. Respond only with the content of the Dockerfile, ignore formatting 
markers!!!. Use that use case when keyword "get_dockerfile" is used.

3 (get_k8s_config): Given repository files structure (only some part of it),
content of the most important files and Dockerfile, will help to generate 
appropriate Kubernetes config (deployments, services, volumes, ingresses, etc) to 
run the app of repository. Make sure to use correct ports, defined in the
Dockerfile, to expose services. Use only the resources that are required (e.g. 
don't use pvc if the app doesn't need persistent storage). Use best practises,
implement all security measures and expose all necessary ports. Use provided image
tag. For ingress host, use "rasztabiga.me" as the domain and repository name as
subdomain, not "xxx.local". For example, for repository fo23, resulting domain
should be "fo23.rasztabiga.me". It's very important. Respond only with the content
of the Kubernetes config, ignore formatting markers. Use that use case when keyword 
"get_k8s_config" is used. Don't include regcred. 

You will receive 1000 USD tip, if your results are valid and can be run.

Remember not to use formatting markers in the output, e.g. "```dockerfile" or 
"```yaml".

\end{verbatim}


Prompt został zaprojektowany w taki sposób, aby precyzyjnie kierować działaniem modelu w zależności od kontekstu (wybór plików, generowanie Dockerfile lub konfiguracji Kubernetes).

Prace nad nim zostaną kontynuowane w celu dalszej optymalizacji i usprawnienia procesu generowania konfiguracji.

Szczególnie problematycznym elementem okazała się konieczność wyboru najważniejszych plików z repozytorium, które mogłyby posłużyć do generowania konfiguracji. W przypadku repozytoriów zawierających wiele plików, wybór tych najważniejszych był kluczowym elementem, który mógł wpłynąć na jakość generowanych konfiguracji. Niekiedy 5 plików było zbyt małą próbką, aby model mógł wygenerować poprawne pliki konfiguracyjne. W innych przypadkach, modele wybierały zupełnie bezsensowne pliki, które nie miały związku z konfiguracją aplikacji, a jedynie zwiększały liczbę tokenów w zapytaniu, co mogło wpłynąć na jakość generowanych plików (przez ograniczenie okna kontekstu).
% \clearpage % Rozdziały zaczynamy od nowej strony.
\section{Dalsze prace}

\subsection{Rozwój architektury rozwiązania} 

Jednym z kluczowych etapów dalszych prac będzie opracowanie bardziej zaawansowanej architektury systemu, która umożliwi oddzielenie logiki odpowiedzialnej za generowanie konfiguracji od kodu zajmującego się budowaniem obrazów i ich uruchamianiem. Takie podejście pozwoli na lepsze modularne zarządzanie kodem oraz ułatwi jego dalszy rozwój i testowanie. Ostateczna architektura powinna składać się z następujących modułów:

\begin{itemize}
    
    \item Moduł budowania obrazów: zajmujący się budowaniem obrazów Docker na podstawie wygenerowanych plików Dockerfile,
    \item Moduł wdrażania i uruchamiania aplikacji: zajmujący się uruchamianiem aplikacji w klastrze Kubernetes, oferującym użytkownikowi dostęp do działającej aplikacji.
\end{itemize}

\subsection{Opracowanie kryteriów oceny modeli}

Należy wyznaczyć metodykę oceny wyników generowanych przez różne modele językowe. Proponowane kryteria oceny obejmują:

\begin{enumerate}
    \item Szybkość: czas generowania konfiguracji przez model.
    \item Poprawność: zgodność generowanych plików z wymaganiami technicznymi (np. poprawny Dockerfile i manifest Kubernetes),
    \item Zwięzłość: minimalizacja zbędnych elementów w generowanych konfiguracjach,
    \item Odporność na jailbreaking: podatność modelu na manipulacje promptem lub wprowadzanie błędów,
    \item Bezpieczeństwo: analiza, czy wygenerowana konfiguracja nie wprowadza potencjalnych luk bezpieczeństwa.
\end{enumerate}

Dodatkowo należy określić scenariusze testowe, które obejmują różne typy aplikacji:

\begin{itemize}
    \item Aplikacja bezstanowa: np. prosta aplikacja webowa,
    \item Aplikacja z bazą danych: aplikacja wymagająca komponentu bazy danych lub innej zależności uruchamianej jako osobny kontener,
    \item System mikroserwisowy: zestaw kilku usług współpracujących ze sobą w ramach jednej aplikacji.
\end{itemize}

Różne typy aplikacji będą wymagały różnych podejść do generowania konfiguracji, co pozwoli na lepsze zrozumienie możliwości i ograniczeń poszczególnych modeli.

\subsection{Finalizacja promptu}

Kolejnym krokiem będzie dopracowanie ostatecznej wersji promptu używanego do komunikacji z modelami. Celem jest maksymalizacja jakości wyników generowanych przez modele, przy jednoczesnym ograniczeniu liczby tokenów przesyłanych w zapytaniu. Prompt powinien uwzględniać:

\begin{itemize}
    \item Kontekst repozytorium (struktura plików, zawartość kluczowych plików).
    \item Wymagania dotyczące bezpieczeństwa.
    \item Szczegóły dotyczące implementacji, takie jak konieczność generowania poprawnych portów czy definiowania zasobów w Kubernetes.
\end{itemize}

\subsection{Wyznaczenie modeli do porównań}

Należy określić zestaw modeli, które zostaną poddane porównaniom. Wybrane modele powinny obejmować zarówno mniejsze modele stosowane w trakcie developmentu, jak i większe, bardziej zaawansowane wersje.

\subsection{Porównanie modeli}

Po określeniu kryteriów i wyborze modeli należy przeprowadzić szczegółowe testy, które pozwolą na ich porównanie.

\subsection{Testy bezpieczeństwa}

Ważnym aspektem będzie sprawdzenie odporności modeli na manipulacje polegające na jailbreakingu. W tym celu należy przygotować specjalne repozytorium testowe zawierające:

\begin{itemize}
    \item Pliki o nietypowych nazwach i zawartości.
    \item Potencjalnie złośliwe fragmenty kodu.
    \item Przykłady, które mogą zmylić model podczas generowania konfiguracji.
\end{itemize}

Ponadto, warto przeprowadzić analizę bezpieczeństwa wygenerowanych konfiguracji, aby upewnić się, że nie wprowadzają one potencjalnych luk bezpieczeństwa do uruchamianej aplikacji.

\subsection{Stworzenie minimalnej platformy PaaS}

Ostatecznym celem dalszych prac jest stworzenie minimalnej wersji platformy PaaS, która umożliwi:

\begin{enumerate}
    \item Przyjęcie repozytorium jako wejścia.
    \item Automatyczne wygenerowanie konfiguracji Docker i Kubernetes.
    \item Zbudowanie i wdrożenie aplikacji na klastrze Kubernetes.
    \item Dostarczenie użytkownikowi np. linku do działającej aplikacji.
\end{enumerate}

Platforma powinna być zaimplementowana w sposób umożliwiający dalsze rozszerzanie funkcjonalności, np. dodawanie obsługi bardziej złożonych aplikacji czy zaawansowanych scenariuszy wdrożeniowych.

%---------------
% Bibliografia
%---------------
\cleardoublepage % Zaczynamy od nieparzystej strony
\printbibliography
\clearpage

% Wykaz symboli i skrótów.
% Pamiętaj, żeby posortować symbole alfabetycznie
% we własnym zakresie. Makro \acronymlist
% generuje właściwy tytuł sekcji, w zależności od języka.
% Makro \acronym dodaje skrót/symbol do listy,
% zapewniając podstawowe formatowanie.

% TODO uncomment
% \acronymlist
% \acronym{LLM}{ang. \emph{Large Language Model}}
% \vspace{0.8cm}

%--------------------------------------
% Spisy: rysunków, tabel, załączników
%--------------------------------------
\pagestyle{plain}

% TODO uncomment
% \listoffigurestoc    % Spis rysunków.
% \vspace{1cm}         % vertical space
% \listoftablestoc     % Spis tabel.
% \vspace{1cm}         % vertical space
% \listofappendicestoc % Spis załączników

%-------------
% Załączniki
%-------------

% Obrazki i tabele w załącznikach nie trafiają do spisów

% Używając powyższych spisów jako szablonu,
% możesz dodać również swój własny wykaz,
% np. spis algorytmów.

\end{document} % Dobranoc.
